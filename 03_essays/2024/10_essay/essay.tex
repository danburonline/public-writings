\documentclass[10pt]{article}
% Packages
\usepackage{makecell} % For thicker lines
\usepackage{setspace}  % Controls line spacing
\usepackage{hhline}   % For double horizontal lines
\usepackage{colortbl}  % Add colour to LaTeX tables
\usepackage[T1]{fontenc}  % Choice of font encodings
\usepackage{tgtermes}  % Loads the TeX Gyre Termes font
\usepackage{siunitx}  % A comprehensive (SI) units package
\usepackage{tabularx, booktabs} % For advanced table layout
\usepackage{url}  % Verbatim with URL-sensitive line breaks
\usepackage{authblk}  % For author and affiliation management
\usepackage{natbib}  % A package for bibliographies and citations
\usepackage{graphicx}  % Enhances LaTeX's built-in graphics features
\usepackage{listings}  % Typeset programming code within the document
\usepackage{amssymb}  % Mathematical symbols
\usepackage[nottoc]{tocbibind}  % Adds entries to the table of contents
\usepackage{xcolor}  % Provides easy driver-independent access to colors
\usepackage{microtype}  % Improves the spacing between words and letters
\usepackage{enumitem}  % Control layout of itemize, enumerate, description
\usepackage{tocloft}  % Controls the typographic design of table of contents, etc.
\usepackage[breaklinks,linktocpage]{hyperref}  % Creates hyperlinks in your document
\usepackage[font=small,skip=7pt,labelfont=bf]{caption}  % Customising captions in floating envs

% Adjust the page margins in the bibliography
\let\oldthebibliography=\thebibliography
\let\endoldthebibliography=\endthebibliography
\renewenvironment{thebibliography}[1]{%
  \begin{oldthebibliography}{#1}%
    \raggedright%
    }{%
  \end{oldthebibliography}%
}

\setlength\bibindent{0pt}

% Optional options
% \usepackage{background} % Creates a DRAFT background image on all pages
% \backgroundsetup{contents=Preprint, opacity=0.25, color=gray} % Adds a watermark to the document
% This command changes the line spacing to double.
% ? Needed for reviews/drafts
% \doublespacing

% Custom colours
\definecolor{codegreen}{rgb}{0,0.5,0}
\definecolor{codegray}{rgb}{0.4,0.4,0.4}
\definecolor{codepurple}{rgb}{0.58,0,0.82}
\definecolor{backcolour}{rgb}{0.96,0.96,0.96}
\definecolor{lightgray}{gray}{0.8}

\lstdefinelanguage{JavaScript}{
  keywords={break, case, catch, continue, debugger, default, delete, do, else, finally, for, function, if, in, instanceof, new, return, switch, this, throw, try, typeof, var, void, while, with},
  morecomment=[l]{//},
  morecomment=[s]{/*}{*/},
  morestring=[b]',
  morestring=[b]",
  sensitive=true
}

% Listing styles
\lstdefinestyle{mystyle}{
  frame=tb,
  tabsize=2,
  captionpos=b,
  numbers=left,
  framerule=0pt,
  numbersep=5pt,
  showtabs=false,
  breaklines=true,
  keepspaces=true,
  showspaces=false,
  framextopmargin=6pt,
  framexbottommargin=6pt,
  showstringspaces=false,
  breakatwhitespace=false,
  keywordstyle=\color{purple},
  commentstyle=\color{codegreen},
  stringstyle=\color{codepurple},
  numberstyle=\tiny\color{codegray},
  basicstyle=\ttfamily\footnotesize,
  backgroundcolor=\color{backcolour}}
\lstset{style=mystyle}

% ! Custom template commands
% Add a vertical space after section numbers in ToC
\renewcommand\cftsecafterpnum{\vskip8pt}
% Changes the title of the list of listings
\renewcommand{\lstlistlistingname}{List of \lstlistingname s}
% Changes the title of the bibliography
\renewcommand{\bibname}{References}
% Changes the title of the table of contents
\renewcommand{\contentsname}{Table of Contents}
% Changes the leader between section and page numbers in ToC
\renewcommand{\cftsecleader}{\cftdotfill{\cftdotsep}}
\newcommand{\floatcaption}[2]{\caption[#1.]{#1~#2.}}

% Custom template settings
\captionsetup{justification=centering}  % All captions will be centered
\hypersetup{
  colorlinks = true,
  urlcolor = blue,
  linkcolor = blue,
  citecolor = blue,
  breaklinks = true
}
\PassOptionsToPackage{hyphens}{url}
\urlstyle{same}
\def\Urlmuskip{0mu plus 1mu}
\def\UrlBreaks{\do\/\do-}
\def\UrlBigBreaks{\do\/\do-\do:\do.}
\setlist[itemize]{noitemsep, topsep=0pt, partopsep=0pt}



\begin{document}
% Changing the initial page numbering to uppercase Roman
\pagenumbering{roman}
% Resetting the page counter to 1
\counterwithin{lstlisting}{section}
\counterwithin{figure}{section}
\counterwithin{table}{section}
% Sets the distance between the bottom and the footer
\setlength{\footskip}{65pt}

% ! ===============================
% ! Start of the title page content
% ! ===============================

\title{\textbf{Stem Cells and Epilepsy:} \\ Modelling the Brain with Organoids}
\author[ ]{Daniel Burger}
% \author[ ]{K23003985}
\affil[ ]{\textbf{King’s College London}}
\affil[ ]{\href{mailto:public@danielburger.online}{public@danielburger.online}}
\date{\textit{13. February 2024}}
\maketitle
% Resetting the page style for the first page
\thispagestyle{empty}

% The sloppypar environment adjusts the spaces between words such that each line fits into the text width
\begin{sloppypar}
  \begin{abstract}
    % NOT THE FINAL ABSTRACT: This essay examines the role of stem-cell-derived models, such as human induced pluripotent stem cells (hiPSCs) and brain organoids, in epilepsy research. Traditional models in vivo and in silico have provided foundational insights but often fail to mimic complex human brain dynamics. Stem-cell-derived models offer a closer approximation of human neurodevelopmental processes in a controlled environment, enabling detailed studies of epilepsy's pathophysiology. The author highlights key advancements, including the modeling of developmental epilepsies and the exploration of therapeutic targets. Case studies illustrate the practical applications of these models in uncovering the mechanisms of epilepsy. The essay also touches on ethical considerations inherent to stem cell research.
  \end{abstract}

  \pagebreak
  % Changing the page numbering back to uppercase Roman
  \pagenumbering{Roman}
  \tableofcontents
  \pagebreak
  \listoffigures
  \pagebreak
  % \listoftables
  % \pagebreak
  % \addcontentsline{toc}{section}{\lstlistlistingname} % Add to the TOC
  % \lstlistoflistings
  % \pagebreak
  % Changing the page numbering back to Arabic
  \pagenumbering{arabic}

  % ! ====================================
  % ! Start of the actual document content
  % ! ====================================

  \section{Introduction}
  \label{sec:introduction}

  TODO: Short introduction on what epilepsy is and why it is important to study it. What is the main neuropathology behind this disease etc. – like how it works with excitatory and inhibitory neurons and how the balance between these two is disrupted in epilepsy. Mention the role of GABAergic interneurons and the importance of understanding their development and function in epilepsy. Also, mention what type of epilepsy is the most common and what type of epilepsy is the most difficult to treat due to the lack of understanding of the underlying mechanisms or the lack of effective treatments.

  Personal note: 50 million people worldwide have epilepsy (WHO). One third of these people live with uncontrollable seizures because no available treatment works for them. This is why it is important to study epilepsy and find new treatments and potentially cures. People need to quit their jobs because of epilepsy, they can't drive as most countries forbid people with a diagnosis of epilepsy to obtain a drivings license, they can't live a normal life due to the constant risk of seizures. Also, show a EEG recording of a seizure and healthy brain activity to illustrate the difference. Also, maybe shortly explain how a person with epilepsy feels during a seizure and how it affects their life and how a person acquires epilepsy: genetic, brain injury, infection, etc. (in combination with the most common types of epilepsy).

  \subsection{Traditional Models for Epilepsy Research}
  \label{sec:traditional-models-for-epilepsy-research}
  TODO: Briefly mention the traditional models for epilepsy research, such as animal models and in silico models and their limitations.

  Personal note: Animal models are not always a good representation of human brain activity and in silico models are not always accurate due to the complexity of the human brain, as a model is only as good as the data it is based on (criticism of in silico and large brain simulation needed, mention Markram and Blue Brain Project and the energy-need for these simulations). This is why stem-cell-derived models are so important, as they provide a closer approximation of human neurodevelopmental processes in a controlled environment, which is unique and very valuable for epilepsy research. Not only do they provide a closer approximation of human neurodevelopmental processes, but they also enable detailed studies of epilepsy's pathophysiology, which is crucial for understanding the disease and finding new treatments (explain this in more detail). Mention that most likely we will always need a combination of different models to understand the human brain and its diseases, but that stem-cell-derived models are a very important part of this combination and a very new and promising field of research, not only for studying but also for treating epilepsy (referencing future chapter).

  \subsection{Stem-Cell-Derived Models for the new Era}
  \label{sec:stem-cell-derived-models-for-the-new-era}

  TODO: Introducing stem-cell-derived models (e.g., hiPSCs, brain organoids, etc.) and their relevance in studying neurodevelopmental disorders like epilepsy. Explain how stem cells are reprogrammed to model epilepsy in vitro and how this is a new and promising field of research. Mention how the research wasn't possible if we didn't have the technology to reprogram stem cells and how we used to rely on cells from embryonic tissue, which is not only unethical but also not as effective as using stem cells. This opened up a new field of research and made it possible to study epilepsy in a way that wasn't possible before. Explain how incubating works, and the process of creating neural stem cells from stem cells. How we use model version of healthy and epileptic brain tissue to study the differences and how we can use this to find new treatments for epilepsy as we can apply drugs to the tissue and see how it reacts. Not only this, but also interfacing with it via electrodes and other techniques.

  \subsection{Current State of the Art}
  \label{sec:current-state-of-the-art}

  TODO: Discussing the current state of research in this field, referencing Nieto-Estévez and Hsieh (2020) and Wang (2018) for recent advancements in modelling developmental epilepsies and neurological diseases using brain organoids.

  Personal note: Mostly brain tissue is studied in a 2D environment, but with stem-cell-derived models, we can study brain tissue in a 3D environment, which is much closer to the real human brain and its activity, so-called brain organoids or cerebral organoids. This is a new and promising field of research, as it allows us to study the human brain in a way that wasn't possible before. We can grow or even 3D print brain tissue around electrodes, or on 3D electrode matrices in order to stimulate and measure the brain organoids. Other techniques include also genetically modifying brain organoids to fine-grain control neuronal firing via a method called optogenetics (just quickly explain it). Say that e.g. when growing brain organoids in a 3D structure (also explain how this works e.g. via a shaker) that they can develop epileptic activity, which is very interesting for epilepsy research. Also, mention that we can use brain organoids to study the development of GABAergic interneurons and how they are affected in epilepsy, as they are crucial for the balance between excitatory and inhibitory neurons in the brain.

  \section{Discussion}
  \label{sec:discussion}

  TODO: Short introduction to the discussion section.

  \subsection{Stem-Cell-Derived Models for Epilepsy}
  \label{sec:stem-cell-derived-models-for-epilepsy}

  TODO:
  - Discussing reprogramming stem cells to model epilepsy in vitro, using Parent and Anderson (2015) and Tidball and Parent (2015) as references.
  - Highlighting recent findings in the field, such as Thodeson, Brulet, and Hsieh's (2017) work on neural stem cells and epilepsy and how these models have enhanced our understanding of epilepsy.
  - Evaluating the effectiveness and limitations of using stem-cell-derived models, possibly drawing on the comparative analysis provided by Kandemir et al. (2022) between different epilepsy models.

  \subsection{Case Studies and Practical Applications}
  \label{sec:case-studies-and-practical-applications}

  TODO:
  - Presenting specific case studies from the references, like the work by Samarasinghe et al. (2021) on identifying neural oscillations in brain organoids and their implications for understanding epilepsy.
  - KEY PART OF THE ESSAY: Discussing practical applications of these models in understanding and treating epilepsy, referencing Steinberg et al. (2020) and their modelling of genetic epileptic encephalopathies.

  Personal note: Main key takeaway from this section is that studying neural tissue in as-close-to-the-real-thing as possible is key. There are more and more methods to really try to get the tissue structure right in order to be able to study the brain in a very controlled environment without ethical issues or lots of efforts compared to finding in vivo tissue. Not only will this research benefit studying it by modelling it, but also by studying it directly via applying medication onto brain organoids and seeing how they react. Other approaches might be creating individualised brain organoids from patients with epilepsy and studying their brain tissue in order to find individualised treatments for epilepsy with their specific case. Also, going beyond just studying, one can go one further and then start to treat epilepsy with brain organoids, e.g. by implanting them into the brain of a patient with epilepsy in order to produce GABAergic interneurons in the parts of the brain where they are needed – generally just genetically modifying the implants for the specific patient's disease (referencing upcoming chapter).

  \subsection{Other Approaches and Advancements}
  \label{sec:other-approaches-and-advancements}

  TODO:
  - Exploring the transplantation of hiPSCs/brain organoids into living beings, referencing Hunt and Baraban (2015) for their work on interneuron transplantation and others. Also mention the work of NRTX-1001 and how they implanted stem cells into the brain of a patient with epilepsy in order to produce GABAergic interneurons in the parts of the brain where they are needed.
  - Discuss the role and potential of neuroprosthetics in epilepsy treatment and research, drawing on insights from current studies or reviews.

  Personal note: Generally the field of neuroprosthetics with synthetic biological neural tissue is quite interesting and also a personal interest of the author. Not only does it go into the field of personalised treatment and medicine but also into the field of synthetic biology and bioengineering. Cite text from "Augmenting Cognition" book from Markram, where Mijail Demian Serruya writes: "3.9 Expanding the neural substrate  Instead of using pairs of recording and stimulating arrays to reconnect one cortical area to another within a patient’s brain, one could  consider routing the signals through an artificial model of the cortex.  If this were possible, one could provide patients with additional neural substrate.Just as regions of the brain may be become unusable due  to stroke, injury or degenerative conditions, so too there might be the  possibility to add new virtual or ectopic cortex to compensate for lost  tissue. Recordings of units throughout the brain could be fed into a  software model, or into actual ectopic neural tissue, and then activity from this model or neural tissue could be used to trigger stimulation back into the patient’s brain (Fig. 3.3c). Initially, such additional cortex could exist as computational models in software programs  bidirectionally linked to a patient through wireless connections. Eventually, software models could be rendered in hardware as an encapsulated silicon chip, e.g., neuromorphic very-large-scale-integrated  (VLSI) microchips, that in turn could be implanted in the body."

  \subsection{Ethical Considerations}
  \label{sec:ethical-considerations}

  TODO:
  - Discussing ethical issues surrounding stem cell research and brain organoid models, citing Farahany et al. (2018) for a comprehensive view of the ethics of experimenting with human brain tissue.
  - Address specific ethical questions, such as the consciousness of brain organoids and the moral implications of in vitro experimentation.

  Personal note: Generally focus on the aspect of scaling brain organoids. Currently limited by the blood supply (still, ongoing research and most likely being solved in the coming years) and the issues that we don't know how to accurately model the tissue structure of an actual human brain, which is most likely necessary to create a consciousness human brain. However, as mentioned before, both of these issues are being worked on as their promises are huge. Therefore the issue of creating sentient/conscious brain organoids in the labs is a topic of ongoing debate and research. Also mention the company FinalSpark in Switzerland, that e.g. uses brain organoids for running AI algorithms and how this is a very promising field of research and how it is important to keep an eye on the ethical implications of this research, as we don't want to create sentient beings in the lab without knowing it and as slaves for AI algorithms (e.g. creating a sentient being that is constantly in pain and suffering, as it is used for running a specific AI algorithm).

  \section{Conclusion}
  \label{sec:conclusion}

  TODO:
  - Summarising the key points, emphasising the impact of stem-cell-derived models compared to other models, especially e.g. in silico or animal testing.
  - Discussing future prospects of stem-cell-derived models in neuroscience, considering technological advancements and potential breakthroughs.
  - Addressing remaining challenges, including technical, ethical, and funding-related issues, to give a balanced view of the field's future.

  Personal note: Basically conclude and have a key focus on that this field needs to get more funding, as brain organoids coupled with personalised medicine and gene-editing is a very promising field of research and has the potential to cure epilepsy and other neurological diseases. Also, mention somehow neuroprosthetics again the the whole point from Mijail Demian Serruya's part of the "Augmenting Cognition" book from Markram.

  % ! ==============================================
  % ! Start of the references and appendices content
  % ! ==============================================

  \pagebreak
  \bibliographystyle{../../templates/custom-apa}
  \bibliography{references/bibliography}

\end{sloppypar}
\end{document}
