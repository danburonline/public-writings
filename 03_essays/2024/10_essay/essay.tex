\documentclass[10pt]{article}
% Packages
\usepackage{makecell} % For thicker lines
\usepackage{setspace}  % Controls line spacing
\usepackage{hhline}   % For double horizontal lines
\usepackage{colortbl}  % Add colour to LaTeX tables
\usepackage[T1]{fontenc}  % Choice of font encodings
\usepackage{tgtermes}  % Loads the TeX Gyre Termes font
\usepackage{siunitx}  % A comprehensive (SI) units package
\usepackage{tabularx, booktabs} % For advanced table layout
\usepackage{url}  % Verbatim with URL-sensitive line breaks
\usepackage{authblk}  % For author and affiliation management
\usepackage{natbib}  % A package for bibliographies and citations
\usepackage{graphicx}  % Enhances LaTeX's built-in graphics features
\usepackage{listings}  % Typeset programming code within the document
\usepackage{amssymb}  % Mathematical symbols
\usepackage[nottoc]{tocbibind}  % Adds entries to the table of contents
\usepackage{xcolor}  % Provides easy driver-independent access to colors
\usepackage{microtype}  % Improves the spacing between words and letters
\usepackage{enumitem}  % Control layout of itemize, enumerate, description
\usepackage{tocloft}  % Controls the typographic design of table of contents, etc.
\usepackage[breaklinks,linktocpage]{hyperref}  % Creates hyperlinks in your document
\usepackage[font=small,skip=7pt,labelfont=bf]{caption}  % Customising captions in floating envs

% Adjust the page margins in the bibliography
\let\oldthebibliography=\thebibliography
\let\endoldthebibliography=\endthebibliography
\renewenvironment{thebibliography}[1]{%
  \begin{oldthebibliography}{#1}%
    \raggedright%
    }{%
  \end{oldthebibliography}%
}

\setlength\bibindent{0pt}

% Optional options
% \usepackage{background} % Creates a DRAFT background image on all pages
% \backgroundsetup{contents=Preprint, opacity=0.25, color=gray} % Adds a watermark to the document
% This command changes the line spacing to double.
% ? Needed for reviews/drafts
% \doublespacing

% Custom colours
\definecolor{codegreen}{rgb}{0,0.5,0}
\definecolor{codegray}{rgb}{0.4,0.4,0.4}
\definecolor{codepurple}{rgb}{0.58,0,0.82}
\definecolor{backcolour}{rgb}{0.96,0.96,0.96}
\definecolor{lightgray}{gray}{0.8}

\lstdefinelanguage{JavaScript}{
  keywords={break, case, catch, continue, debugger, default, delete, do, else, finally, for, function, if, in, instanceof, new, return, switch, this, throw, try, typeof, var, void, while, with},
  morecomment=[l]{//},
  morecomment=[s]{/*}{*/},
  morestring=[b]',
  morestring=[b]",
  sensitive=true
}

% Listing styles
\lstdefinestyle{mystyle}{
  frame=tb,
  tabsize=2,
  captionpos=b,
  numbers=left,
  framerule=0pt,
  numbersep=5pt,
  showtabs=false,
  breaklines=true,
  keepspaces=true,
  showspaces=false,
  framextopmargin=6pt,
  framexbottommargin=6pt,
  showstringspaces=false,
  breakatwhitespace=false,
  keywordstyle=\color{purple},
  commentstyle=\color{codegreen},
  stringstyle=\color{codepurple},
  numberstyle=\tiny\color{codegray},
  basicstyle=\ttfamily\footnotesize,
  backgroundcolor=\color{backcolour}}
\lstset{style=mystyle}

% ! Custom template commands
% Add a vertical space after section numbers in ToC
\renewcommand\cftsecafterpnum{\vskip8pt}
% Changes the title of the list of listings
\renewcommand{\lstlistlistingname}{List of \lstlistingname s}
% Changes the title of the bibliography
\renewcommand{\bibname}{References}
% Changes the title of the table of contents
\renewcommand{\contentsname}{Table of Contents}
% Changes the leader between section and page numbers in ToC
\renewcommand{\cftsecleader}{\cftdotfill{\cftdotsep}}
\newcommand{\floatcaption}[2]{\caption[#1.]{#1~#2.}}

% Custom template settings
\captionsetup{justification=centering}  % All captions will be centered
\hypersetup{
  colorlinks = true,
  urlcolor = blue,
  linkcolor = blue,
  citecolor = blue,
  breaklinks = true
}
\PassOptionsToPackage{hyphens}{url}
\urlstyle{same}
\def\Urlmuskip{0mu plus 1mu}
\def\UrlBreaks{\do\/\do-}
\def\UrlBigBreaks{\do\/\do-\do:\do.}
\setlist[itemize]{noitemsep, topsep=0pt, partopsep=0pt}



\begin{document}
% Changing the initial page numbering to uppercase Roman
\pagenumbering{roman}
% Resetting the page counter to 1
\counterwithin{lstlisting}{section}
\counterwithin{figure}{section}
\counterwithin{table}{section}
% Sets the distance between the bottom and the footer
\setlength{\footskip}{65pt}

% ! ===============================
% ! Start of the title page content
% ! ===============================

\title{\textbf{Stem Cells and Epilepsy:} \\ Modelling the Brain with Organoids}
\author[ ]{Daniel Burger}
% \author[ ]{K23003985}
\affil[ ]{\textbf{King’s College London}}
\affil[ ]{\href{mailto:public@danielburger.online}{public@danielburger.online}}
\date{\textit{13. February 2024}}
\maketitle
% Resetting the page style for the first page
\thispagestyle{empty}

% The sloppypar environment adjusts the spaces between words such that each line fits into the text width
\begin{sloppypar}
  \begin{abstract}
    % This essay examines the role of stem-cell-derived models, such as human induced pluripotent stem cells (hiPSCs) and brain organoids, in epilepsy research. Traditional models in vivo and in silico have provided foundational insights but often fail to mimic complex human brain dynamics. Stem-cell-derived models offer a closer approximation of human neurodevelopmental processes in a controlled environment, enabling detailed studies of epilepsy's pathophysiology. The author highlights key advancements, including the modeling of developmental epilepsies and the exploration of therapeutic targets. Case studies illustrate the practical applications of these models in uncovering the mechanisms of epilepsy. The essay also touches on ethical considerations inherent to stem cell research.
  \end{abstract}

  \pagebreak
  % Changing the page numbering back to uppercase Roman
  \pagenumbering{Roman}
  \tableofcontents
  \pagebreak
  \listoffigures
  \pagebreak
  % \listoftables
  % \pagebreak
  % \addcontentsline{toc}{section}{\lstlistlistingname} % Add to the TOC
  % \lstlistoflistings
  % \pagebreak
  % Changing the page numbering back to Arabic
  \pagenumbering{arabic}

  % ! ====================================
  % ! Start of the actual document content
  % ! ====================================

  \section{Introduction}
  \label{sec:introduction}

  TODO: Short introduction on what epilepsy is and why it is important to study it. What is the main neuropathology behind this disease etc.

  \subsection{Traditional Models for Epilepsy Research}
  \label{sec:traditional-models-for-epilepsy-research}
  TODO: Briefly mention the traditional models for epilepsy research, such as animal models and in silico models and their limitations.

  \subsection{Stem-Cell-Derived Models for the new Era}
  \label{sec:stem-cell-derived-models-for-the-new-era}

  TODO: Introducing stem-cell-derived models (e.g., hiPSCs, brain organoids, etc.) and their relevance in studying neurodevelopmental disorders like epilepsy

  \subsection{Current State of the Art}
  \label{sec:current-state-of-the-art}

  TODO: Discussing the current state of research in this field, referencing Nieto-Estévez and Hsieh (2020) and Wang (2018) for recent advancements in modelling developmental epilepsies and neurological diseases using brain organoids.

  \section{Discussion}
  \label{sec:discussion}

  TODO: Short introduction to the discussion section.

  \subsection{Stem-Cell-Derived Models for Epilepsy}
  \label{sec:stem-cell-derived-models-for-epilepsy}

  TODO:
  - Discussing reprogramming stem cells to model epilepsy in vitro, using Parent and Anderson (2015) and Tidball and Parent (2015) as references.
  - Highlighting recent findings in the field, such as Thodeson, Brulet, and Hsieh's (2017) work on neural stem cells and epilepsy and how these models have enhanced our understanding of epilepsy.
  - Evaluating the effectiveness and limitations of using stem-cell-derived models, possibly drawing on the comparative analysis provided by Kandemir et al. (2022) between different epilepsy models.

  \subsection{Case Studies and Practical Applications}
  \label{sec:case-studies-and-practical-applications}

  TODO:
  - Presenting specific case studies from the references, like the work by Samarasinghe et al. (2021) on identifying neural oscillations in brain organoids and their implications for understanding epilepsy.
  - KEY PART OF THE ESSAY: Discussing practical applications of these models in understanding and treating epilepsy, referencing Steinberg et al. (2020) and their modelling of genetic epileptic encephalopathies.

  \subsection{Other Approaches and Advancements}
  \label{sec:other-approaches-and-advancements}

  TODO:
  - Exploring the transplantation of hiPSCs/brain organoids into living beings, referencing Hunt and Baraban (2015) for their work on interneuron transplantation and others.
  - Discuss the role and potential of neuroprosthetics in epilepsy treatment and research, drawing on insights from current studies or reviews.

  \subsection{Ethical Considerations}
  \label{sec:ethical-considerations}

  TODO:
  - Discussing ethical issues surrounding stem cell research and brain organoid models, citing Farahany et al. (2018) for a comprehensive view of the ethics of experimenting with human brain tissue.
  - Address specific ethical questions, such as the consciousness of brain organoids and the moral implications of in vitro experimentation.

  \section{Conclusion}
  \label{sec:conclusion}

  TODO:
  - Summarising the key points, emphasising the impact of stem-cell-derived models compared to other models, especially e.g. in silico or animal testing.
  - Discussing future prospects of stem-cell-derived models in neuroscience, considering technological advancements and potential breakthroughs.
  - Addressing remaining challenges, including technical, ethical, and funding-related issues, to give a balanced view of the field's future.

  % ! ==============================================
  % ! Start of the references and appendices content
  % ! ==============================================

  \pagebreak
  \bibliographystyle{../../templates/custom-apa}
  \bibliography{references/bibliography}

\end{sloppypar}
\end{document}
