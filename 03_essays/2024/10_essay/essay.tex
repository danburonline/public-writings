\documentclass[10pt]{article}
% ! ========================
% ! # MARK: Template Control
% ! ========================

% Defines the template variant. Can be 'essay' (default) or 'paper'.
% This command must come before the document class definition in the main file.
\providecommand{\templatevariant}{essay}


% ! =======================
% ! # MARK: Package Loading
% ! =======================

% Common Packages
\usepackage[T1]{fontenc} % Font encoding
\usepackage{tgtermes} % TeX Gyre Termes font (Times clone)
\usepackage{geometry} % Page layout control
\usepackage{setspace} % Line spacing control
\usepackage{graphicx} % Enhanced graphics
\usepackage{xcolor} % Color definitions
\usepackage{colortbl} % Color in tables
\usepackage{hhline} % Double horizontal lines in tables
\usepackage{makecell} % Thicker lines in tables
\usepackage{tabularx, booktabs} % Advanced table layouts
\usepackage{enumitem} % Custom list environments
\usepackage{amsmath} % Advanced math environments
\usepackage{amssymb} % Math symbols
\usepackage{siunitx} % SI units package
\usepackage{listings} % Code listings
\usepackage{natbib} % Bibliography and citations
\usepackage{authblk} % Author and affiliation blocks
\usepackage{tocloft} % ToC, LoF, LoT styling
\usepackage[nottoc]{tocbibind} % Add Bib, Index, etc., to ToC
\usepackage{microtype} % Improved typography (justification, spacing)
\usepackage{url} % URL formatting
\usepackage[breaklinks,linktocpage]{hyperref} % Hyperlinks (must be loaded late)

% Variant-Specific Packages
% Define variant names for comparison
\def\papervariant{paper}
\def\essayvariant{essay}

% Load packages only required for the 'paper' template
\ifx\templatevariant\papervariant
  \usepackage{multicol}     % For multi-column layout
  \usepackage{dblfloatfix}  % Fixes for double-column floats
\fi


% ! ==============================
% ! # MARK: Page Layout & Geometry
% ! ==============================

\ifx\templatevariant\papervariant
  % Paper Template Layout (Two-Column)
  \geometry{
    left=3cm,
    right=3cm,
    top=2.5cm,
    bottom=3cm
  }
  % Settings for two-column layout
  \setlength{\columnseprule}{0pt} % No vertical rule between columns
  \setlength{\columnsep}{0.7cm}   % Space between columns
\else
  % Essay Template Layout (Single-Column, Default)
  \geometry{
    left=4cm,
    right=4cm,
    top=3cm,
    bottom=3.5cm
  }
\fi


% ! ==========================
% ! # MARK: Typography & Fonts
% ! ==========================

% Page Numbering
\renewcommand{\thepage}{\footnotesize\arabic{page}}

% List Spacing
\setlist[itemize]{noitemsep, topsep=7pt, partopsep=0pt, leftmargin=*, label=\textendash}


% ! ================================
% ! # MARK: Floats, Figures & Tables
% ! ================================

% Float Placement Parameters
\renewcommand{\topfraction}{0.9}
\renewcommand{\bottomfraction}{0.8}
\renewcommand{\textfraction}{0.1}
\renewcommand{\floatpagefraction}{0.8}
% Additional float settings for two-column paper template
\ifx\templatevariant\papervariant
  \renewcommand{\dbltopfraction}{0.9}
  \renewcommand{\dblfloatpagefraction}{0.8}
\fi

% Caption Styling
\usepackage[font=footnotesize,skip=7pt,labelfont=bf]{caption}
\captionsetup{justification=raggedright} % Left-align all captions
\newcommand{\floatcaption}[2]{\caption[#1.]{#1~#2.}} % Custom caption command


% ! ================================
% ! # MARK: Bibliography & Citations
% ! ================================

\renewcommand{\bibname}{References} % Change bibliography title
\setlength\bibindent{0pt}           % No indentation for bibliography entries

% Adjust layout and font size in the bibliography environment
\let\oldthebibliography=\thebibliography
\let\endoldthebibliography=\endthebibliography
\renewenvironment{thebibliography}[1]{%
  \begin{oldthebibliography}{#1}%
    \raggedright%
    \footnotesize%
    \setlength{\itemsep}{3pt}%
    \setlength{\parsep}{0pt}%
    \setlength{\parskip}{0pt}%
    }{%
  \end{oldthebibliography}%
}


% ! =====================
% ! # MARK: Code Listings
% ! =====================

% Custom Colors for Code
\definecolor{codegreen}{rgb}{0,0.5,0}
\definecolor{codegray}{rgb}{0.4,0.4,0.4}
\definecolor{codepurple}{rgb}{0.58,0,0.82}
\definecolor{backcolour}{rgb}{0.96,0.96,0.96}
\definecolor{lightgray}{gray}{0.8}

% Language Definition (Example: JavaScript)
\lstdefinelanguage{JavaScript}{
  keywords={break, case, catch, continue, debugger, default, delete, do, else, finally, for, function, if, in, instanceof, new, return, switch, this, throw, try, typeof, var, void, while, with},
  morecomment=[l]{//},
  morecomment=[s]{/*}{*/},
  morestring=[b]',
  morestring=[b]",
  sensitive=true
}

% Listing Style Definition
\lstdefinestyle{mystyle}{
  backgroundcolor=\color{backcolour},
  commentstyle=\color{codegreen},
  keywordstyle=\color{purple},
  numberstyle=\tiny\color{codegray},
  stringstyle=\color{codepurple},
  basicstyle=\ttfamily\footnotesize,
  breakatwhitespace=false,
  breaklines=true,
  captionpos=b,
  frame=tb,
  framerule=0pt,
  framextopmargin=6pt,
  framexbottommargin=6pt,
  keepspaces=true,
  numbers=left,
  numbersep=5pt,
  showspaces=false,
  showstringspaces=false,
  showtabs=false,
  tabsize=2
}
\lstset{style=mystyle} % Apply the defined style globally


% ! ===================================
% ! # MARK: Document Structure & Titles
% ! ===================================

% Table of Contents (ToC)
\renewcommand{\contentsname}{Table of Contents}
\renewcommand\cftsecafterpnum{\vskip8pt}              % Vertical space after section numbers
\renewcommand{\cftsecleader}{\cftdotfill{\cftdotsep}} % Dotted leaders for sections

% List of Listings (LoL)
\renewcommand{\lstlistlistingname}{List of \lstlistingname s}


% ! ==========================
% ! # MARK: Hyperlinks & URLs
% ! ==========================

\hypersetup{
  colorlinks = true,
  urlcolor   = blue,
  linkcolor  = blue,
  citecolor  = blue,
  breaklinks = true
}
% URL Line Breaking
\PassOptionsToPackage{hyphens}{url}
\urlstyle{same}
\def\Urlmuskip{0mu plus 1mu}
\def\UrlBreaks{\do\/\do-}
\def\UrlBigBreaks{\do\/\do-\do:\do.}


% ! =======================
% ! # MARK: Draft Watermark
% ! =======================

% Uncomment the following lines to add a "DRAFT" watermark on every page.
% \usepackage{background}
% \backgroundsetup{contents=DRAFT, opacity=0.25, color=gray}
% Line Spacing
% \doublespacing % Uncomment for review drafts

\usepackage{enumitem}


\begin{document}
% Changing the initial page numbering to uppercase Roman
\pagenumbering{roman}
% Resetting the page counter to 1
\counterwithin{lstlisting}{section}
\counterwithin{figure}{section}
\counterwithin{table}{section}
% Sets the distance between the bottom and the footer
\setlength{\footskip}{65pt}

% ! ===============================
% ! Start of the title page content
% ! ===============================

\title{\textbf{Stem Cells and Epilepsy:} \\ Modelling the Brain with Organoids}
\author[ ]{Daniel Burger}
% \author[ ]{K23003985}
\affil[ ]{\textbf{King’s College London}}
\affil[ ]{\href{mailto:public@danielburger.online}{public@danielburger.online}}
\date{\textit{13. February 2024}}
\maketitle
% Resetting the page style for the first page
\thispagestyle{empty}
% \doublespacing % TODO Uncomment before exporting to MS Word

% The sloppypar environment adjusts the spaces between words such that each line fits into the text width
\begin{sloppypar}
  \begin{abstract}
    % TODO Create the final abstract
    % NOT THE FINAL ABSTRACT: This essay examines the role of stem-cell-derived models, such as human induced pluripotent stem cells (hiPSCs) and brain organoids, in epilepsy research. Traditional models in vivo and in silico have provided foundational insights but often fail to mimic complex human brain dynamics. Stem-cell-derived models offer a closer approximation of human neurodevelopmental processes in a controlled environment, enabling detailed studies of epilepsy's pathophysiology. The author highlights key advancements, including the modeling of developmental epilepsies and the exploration of therapeutic targets. Case studies illustrate the practical applications of these models in uncovering the mechanisms of epilepsy. The essay also touches on ethical considerations inherent to stem cell research.
  \end{abstract}

  \pagebreak
  % Changing the page numbering back to uppercase Roman
  \pagenumbering{Roman}
  \tableofcontents
  \pagebreak
  \listoffigures
  \pagebreak
  % \listoftables
  % \pagebreak
  % \addcontentsline{toc}{section}{\lstlistlistingname} % Add to the TOC
  % \lstlistoflistings
  % \pagebreak
  % Changing the page numbering back to Arabic
  \pagenumbering{arabic}

  % ! ====================================
  % ! Start of the actual document content
  % ! ====================================

  \section{Introduction}
  \label{sec:introduction}

  Epilepsy, a neurological disorder afflicting around 50 million individuals worldwide \citep{world2019epilepsy}, manifests through recurrent, unprovoked seizures, which can lead to symptoms such as convulsions (i.e. uncontrolled shaking), loss of consciousness, and unusual sensations or behaviours. The development of seizures in epilepsy is mainly due to a complex interaction between the excitatory and inhibitory activities of neurons in the brain. When there is an imbalance that leads to more excitation than inhibition, it can cause sudden and abnormal electrical activity in localised or entire regions of the brain \citep{robinson_propagation_1997}.

  Temporal lobe epilepsy (TLE) is the most common form \citep{epilepsy_foundation_temporal_2019}—yet it is the drug-resistant epilepsies, especially those without an identifiable epileptogenic zone, that present the most significant treatment challenges \citep{iwasaki_non-invasive_2016, guery_clinical_2021}. Current interventions, such as neuromodulation \citep{fisher_electrical_2014}, offer some hope, but their effectiveness is variable and can be marred by severe side effects, highlighting the urgent need for novel research directions.

  The impact of epilepsy extends far beyond its physiological symptoms, affecting every facet of an individual's life. A large number of people who are diagnosed suffer from uncontrollable seizures, rendering them incapable of performing routine tasks, securing employment, or even driving—a prohibition enforced in many countries, such as e.g. Switzerland, due to safety concerns \citep{schweizerische_epilepsieliga_driving_2021}. This pervasive uncertainty cultivates a lifestyle fraught with limitations, emphasising the disease's profound societal and personal toll.

  \begin{figure}[ht]
    \centering
    \frame{\includegraphics[width=\textwidth]{figures/seizure-eeg.jpg}}
    \caption[EEG Waveforms: Epileptic vs. Normal Brain Activity]{\textbf{EEG Waveforms: Epileptic vs. Normal Brain Activity.} Comparative display of EEG waveforms showing brain electrical activity over 10 seconds. The top graph represents EEG data during an epileptic seizure, characterised by high amplitude and frequent spikes, indicative of abnormal neuronal activity. The bottom graph illustrates a normal EEG with regular wave patterns and lower amplitude, reflecting typical brain function. The x-axis measures time in seconds, and the y-axis measures the amplitude of brain waves in microvolts (µV). Image taken from \cite{espinosa_feedforward_2020}.}
    \label{fig:seizure-eeg}
  \end{figure}

  To fully grasp the essence of epilepsy, one must consider not only its physical manifestations but also the profound psychological toll it exacts on sufferers. The unpredictability of seizures can instil a persistent fear and sense of helplessness, significantly affecting mental health. The origins of epilepsy are diverse, encompassing genetic predispositions (genotype) and observable characteristics (phenotype), including brain injuries and infections, underscoring the complexity of its etiology.
  The dichotomy between normal and epileptic brain function is starkly illustrated in EEG recordings, as exemplarily shown in \autoref{fig:seizure-eeg}. These visual representations not only highlight the aberrant electrical activity characteristic of seizures but also underscore the critical need for a deeper understanding of the underlying mechanisms of epilepsy. Such insights are essential for paving the way toward more targeted and efficacious treatments, addressing the disease's genotype and phenotype.

  \subsection{Traditional Models for Epilepsy Research}
  \label{sec:traditional-models-for-epilepsy-research}

  Traditional epilepsy research models, such as in vivo animal subjects and in silico simulations, have been invaluable yet present notable limitations. Animal models, often e.g. employing rodents \citep{wang_animal_2022}, provide insights but may not fully translate to human epilepsy due to differences in brain structure and function \citep{kandratavicius_animal_2014}. Similarly, in silico models, like those developed by the Blue Brain Project \citep{markram_blue_2006}, offer detailed simulations of neural networks but are limited by current computational capabilities and understanding of the brain \citep{mirza_integrative_2016}.

  Stem-cell-derived models, mainly from human induced pluripotent stem cells (hiPSCs), emerge as a promising solution, offering a more accurate and ethical approach to studying epilepsy. Unlike animal models, which face translational hurdles due to species-specific differences in brain architecture and function, hiPSCs can be derived from human cells, ensuring a closer representation of human pathophysiology. Furthermore, the use of hiPSCs sidesteps the ethical quandaries associated with embryonic stem cell research, as they can be obtained from adult cells (e.g. skin cells) without harm to the donor, thereby aligning with ethical standards for human research \citep{takahashi_induction_2006}. These models circumvent traditional methods’ limitations by providing a renewable source of human neural tissue and enabling the exploration of epilepsy’s neurodevelopmental and pathophysiological aspects in a patient-specific context while also allowing the precise analysis and image of the tissues in a highly controlled environment.

  \subsection{Current State of the Art}
  \label{sec:current-state-of-the-art}

  The advances in stem cell technology have given rise to another groundbreaking approach in the study of epilepsy: the cultivation of three-dimensional brain organoids derived from hiPSCs. Compared to the first approaches where neuronal cells were cultivated in a two-dimensional manner (sometimes called dish brains), researchers can now generate these cerebral organoids that mimic the brain’s complex architectures in terms of tissue structures and the arrangement of cellular types, cell-to-cell interactions, and synaptic connectivity, surpassing two-dimensional dish brains that fall short in replicating the physiological interactions, regional specificity, and microenvironment gradients observed in vivo \citep{clevers_modeling_2016, wang_modeling_2018}.

  Another benefit of cultivating brain organoids is that it allows researchers to grow or 3D-print neural tissue, e.g. around electrically conductive matrices or electrodes \citep{yao_3d_2023}, engaging with and measuring the organoids in ways reminiscent of living brains. Additionally, combining multiple brain organoids as an assembloid offers a sophisticated method to study the interactions between different brain regions \citep{sloan_generation_2018}. These assembloids recreate the complex neural networks and facilitate investigations into the inter-regional synaptic connections critical for higher-order brain functions, often disrupted in epilepsy.

  These state-of-the-art techniques offer a transformative avenue for epilepsy research and beyond. Using hiPSCs to reflect a patient’s unique genetic makeup, brain organoids serve as personalised models to decipher the complex interplay of factors driving epileptogenesis. The insights garnered through such increasingly precise 3D models promise to accelerate the discovery of innovative treatments, aiming to improve the quality of life for individuals with neurological diseases.


  \section{Discussion}
  \label{sec:discussion}

  % TODO: Write short intro to the discussion section (in public essay).

  \subsection{Stem-Cell-Derived Models for Epilepsy}
  \label{sec:stem-cell-derived-models-for-epilepsy}

  %   Yu et al. (2007): This study was one of the pioneering works demonstrating the generation of induced pluripotent stem cells (iPSCs) from human somatic cells. The significance of this work lies in its demonstration of reprogramming adult cells to a pluripotent state, akin to embryonic stem cells, without using embryos. This breakthrough has profound implications for regenerative medicine, including epilepsy research, as it allows for the creation of patient-specific neural cells to study disease mechanisms and potential therapies.

  % Takahashi et al. (2007): This seminal paper, published by Shinya Yamanaka's lab, reported the generation of human iPSCs from adult fibroblasts by introducing four specific genes. This work laid the foundation for using patient-derived iPSCs in disease modeling, including neurological disorders like epilepsy, by enabling the study of disease processes in cell types that are otherwise difficult to obtain from living patients.

  % Jiao et al. (2013): This research focused on modeling Dravet Syndrome, a severe form of genetic epilepsy, using iPSCs derived from patients. The study demonstrated that neurons generated from these iPSCs showed abnormal electrophysiological properties that are characteristic of the disease, providing a valuable model for understanding the underlying mechanisms of Dravet Syndrome and for screening potential therapeutic compounds.

  % Nadadhur et al. (2019): This paper presented a study on Tuberous Sclerosis Complex (TSC), a genetic disorder that often leads to epilepsy, using patient-derived iPSCs. The researchers used iPSCs to model the neural manifestations of TSC, showing defects in neuronal differentiation and maturation that could contribute to the development of epilepsy in TSC patients. This model offers insights into the disease's pathogenesis and a platform for testing therapeutic interventions.

  % Lancaster et al. (2013): This study introduced the use of brain organoids, which are 3D cultures that can mimic the structure and function of the human brain, derived from iPSCs. These organoids have the potential to model complex brain disorders, including epilepsy, in a more physiologically relevant context than 2D cultures, allowing for the study of neural development, disease progression, and drug response in a system that recapitulates the architecture of the brain.

  % Di Lullo & Kriegstein (2017): This review discussed the advances and applications of cerebral organoids in modeling human brain development and diseases. It highlighted the potential of organoids to provide insights into the cellular and molecular mechanisms of neurological disorders, including epilepsy, and stressed the importance of developing more refined organoid models to accurately replicate human brain complexity.

  % Cunningham et al. (2014): This work explored the integration of human iPSC-derived neural progenitor cells into the brains of rodents. The study demonstrated that these human cells could integrate into the host brain, form functional synaptic connections, and mature into neurons and glial cells, offering a valuable in vivo model for studying human brain development, disease processes, and the potential for cell-based therapies in neurological conditions like epilepsy.

  % Marchetto et al. (2010): This research used iPSCs derived from patients with Rett Syndrome, a neurodevelopmental disorder that often includes epilepsy as a symptom, to model the disease in vitro. The study found abnormalities in neuronal development, such as reduced soma size and fewer synapses in iPSC-derived neurons, which could contribute to the neurological symptoms of Rett Syndrome, providing insight into the cellular basis of the disorder and potential therapeutic targets.

  % Parent et al. (2007): This review discussed the role of neurogenesis in epilepsy, particularly focusing on the generation of new neurons in the adult brain's hippocampal region, which can be altered in epilepsy. The review highlighted the potential of manipulating neurogenesis as a therapeutic approach for epilepsy and the need for further research to understand the complex relationship between neurogenesis and seizure activity.

  One of the most compelling aspects of hiPSC technology is its ability to model genetic forms of epilepsy. By deriving iPSCs from patients with known genetic mutations, researchers can observe the direct effects of these mutations on neuronal development, function, and network formation. This approach has been instrumental in elucidating the pathophysiological mechanisms underlying syndromes such as Dravet Syndrome and Tuberous Sclerosis Complex, where specific gene mutations lead to distinct epileptic phenotypes \citep{jiao_modeling_2013, nadadhur_neuronal_2019}.

  Despite these advances, stem-cell-derived models of epilepsy face several challenges. The differentiation of iPSCs into fully mature, functional neurons and glial cells that accurately represent the diversity and complexity of the human brain remains a daunting task. This is particularly pertinent in modelling age-related epilepsies, where the disease phenotype may only manifest or worsen over time. As mentioned earlier, techniques such as 3D brain organoids have emerged as a promising solution to mimic the brain's architecture and cellular heterogeneity more closely, though they are not without their own set of limitations, including the lack of vascularization and an immune system \citep{lancaster_cerebral_2013, di_lullo_cerebral_2017}.

  Moreover, the integration of iPSC-derived neural progenitor cells (NPCs) into in vivo models, such as grafting into embryonic rodent brains, has opened new avenues for studying the interaction between human neurons and their environment. This approach facilitates the maturation of human cells within a living brain, allowing researchers to observe how human neurons integrate and function within an established neural network, thereby offering valuable insights into the processes of synaptogenesis, plasticity, and network formation in the context of epilepsy \citep{cunningham_integration_2014}.

  Recent studies, such as those by \citeauthor{thodeson_neural_2017} \citeyearpar{thodeson_neural_2017}, underscore the potential of neural stem cells in epilepsy research. These investigations reveal that iPSC models can uncover novel insights into diseases with epileptic phenotypes, despite challenges such as variable expression profiles and differentiation potential among iPSC lines. For example, iPSC models of Rett syndrome have illuminated critical aspects of epileptogenesis by demonstrating decreases in neuronal soma size, neurite outgrowth, and synapse formation compared to controls, thereby highlighting the intricate interplay between neuronal and astrocytic contributions to the disease \citep{marchetto_modeling_2010}.

  The evaluation of stem-cell-derived models' effectiveness and limitations, as discussed by \citeauthor{kandemir_investigation_2022} \citeyearpar{kandemir_investigation_2022}, points to the nuanced relationship between neurogenesis and epilepsy. These models not only provide deep insights into the pathological underpinnings of epilepsy but also illuminate potential therapeutic avenues. However, the specificity of neurogenesis markers and the role of mature astrocytes in epilepsy remain areas of ongoing research, emphasizing the need for continued innovation and refinement in stem-cell-derived epilepsy models \citep{parent_neurogenesis_2007}.


  \subsection{Case Studies and Practical Applications}
  \label{sec:case-studies-and-practical-applications}

  Building on the foundational concepts discussed earlier, we now shift our focus to two pivotal case studies and practical examples. These instances illuminate the real-world application of stem-cell-derived models in epilepsy research, showcasing the depth of insights gained and the potential pathways to therapeutic advancements.

  \vspace{0.5cm} % Add empty line

  \begin{itemize}[leftmargin=*]
    \item \textbf{Samarasinghe et al. (2021)} delved into the complex neural dynamics of brain organoids, uncovering significant epileptiform activities within organoids modeling Rett syndrome, which is a genetic disorder that typically affects females and is characterised by impairments in language and coordination, repetitive movements, slower growth, difficulty walking and so on. Complications can include seizures, scoliosis, and sleeping problems. However, \citeauthor{samarasinghe_identification_2021} groundbreaking work demonstrated not only the presence of sophisticated physiological activities within these organoids but also the potential for therapeutic intervention. Remarkably, they observed a substantial reduction in epileptiform activities upon administration of pifithrin-$\alpha$, a TP53 inhibitor, which points towards new directions in epilepsy treatment, especially for conditions exhibiting resistance to conventional therapies.

    \item \textbf{Steinberg et al. (2020)} embarked on an ambitious project to model epileptic encephalopathies using a combination of CRISPR-engineered human ES cells and patient-derived iPSCs, with a focus on the devastating WOREE syndrome. Their meticulous approach unveiled significant cellular and molecular CNS abnormalities, including alterations in GABAergic markers which suggest a disruption in the development of normal and balanced neuronal networks. This work not only underscores the intricate pathophysiology of epileptic encephalopathies but also provides a proof-of-concept for potential therapeutic interventions, such as the modulation of GABAergic responses, which are crucial in the dynamics of developmental epilepsies.
  \end{itemize}

  \vspace{0.5cm} % Add empty line
  These case studies emphasize the critical importance of accurately mimicking human brain tissue in research to provide a platform for studying epilepsy in a controlled, ethical manner. The advancements in stem-cell-derived models, particularly organoids, open up new avenues for direct therapeutic research and personalized treatment strategies. By enabling the study of individualized brain organoids from patients with epilepsy, researchers can explore tailored treatments for specific cases, thereby enhancing the efficacy and precision of epilepsy management. Furthermore, these models hold promise for innovative treatments, such as the potential use of organoids for cell-based therapies in epilepsy, by genetically modifying implants to address the unique needs of each patient's condition.

  \subsection{Other Approaches and Advancements}
  \label{sec:other-approaches-and-advancements}

  TODO:
  - Exploring the transplantation of hiPSCs/brain organoids into living beings, referencing Hunt and Baraban (2015) for their work on interneuron transplantation and others. Also mention the work of NRTX-1001 and how they implanted stem cells into the brain of a patient with epilepsy in order to produce GABAergic interneurons in the parts of the brain where they are needed.
  - Discuss the role and potential of neuroprosthetics in epilepsy treatment and research, drawing on insights from current studies or reviews.
  - Discuss how multiple brain organoids can be fused into an assembloid to study e.g. neuronal migration from e.g. GABAergic and glutamatergic neurons and how this can be used to study epilepsy and other neurological diseases.

  Personal note: Generally the field of neuroprosthetics with synthetic biological neural tissue is quite interesting and also a personal interest of the author. Not only does it go into the field of personalised treatment (also sometimes called precision medicine) and medicine but also into the field of synthetic biology and bioengineering. Cite text from "Augmenting Cognition" book from Markram, where Mijail Demian Serruya writes: "3.9 Expanding the neural substrate  Instead of using pairs of recording and stimulating arrays to reconnect one cortical area to another within a patient’s brain, one could  consider routing the signals through an artificial model of the cortex.  If this were possible, one could provide patients with additional neural substrate.Just as regions of the brain may be become unusable due  to stroke, injury or degenerative conditions, so too there might be the  possibility to add new virtual or ectopic cortex to compensate for lost  tissue. Recordings of units throughout the brain could be fed into a  software model, or into actual ectopic neural tissue, and then activity from this model or neural tissue could be used to trigger stimulation back into the patient’s brain. Initially, such additional cortex could exist as computational models in software programs  bidirectionally linked to a patient through wireless connections. Eventually, software models could be rendered in hardware as an encapsulated silicon chip, e.g., neuromorphic very-large-scale-integrated  (VLSI) microchips, that in turn could be implanted in the body."

  \subsection{Ethical Considerations}
  \label{sec:ethical-considerations}

  TODO:
  - Discussing ethical issues surrounding stem cell research and brain organoid models, citing Farahany et al. (2018) for a comprehensive view of the ethics of experimenting with human brain tissue.
  - Address specific ethical questions, such as the consciousness of brain organoids and the moral implications of in vitro experimentation.

  Personal note: Generally focus on the aspect of scaling brain organoids. Currently limited by the blood supply (still, ongoing research and most likely being solved in the coming years) and the issues that we don't know how to accurately model the tissue structure of an actual human brain, which is most likely necessary to create a consciousness human brain. However, as mentioned before, both of these issues are being worked on as their promises are huge. Therefore the issue of creating sentient/conscious brain organoids in the labs is a topic of ongoing debate and research. Also mention the company FinalSpark in Switzerland, that e.g. uses brain organoids for running AI algorithms and how this is a very promising field of research and how it is important to keep an eye on the ethical implications of this research, as we don't want to create sentient beings in the lab without knowing it and as slaves for AI algorithms (e.g. creating a sentient being that is constantly in pain and suffering, as it is used for running a specific AI algorithm).

  \section{Conclusion}
  \label{sec:conclusion}

  TODO:
  - Summarising the key points, emphasising the impact of stem-cell-derived models compared to other models, especially e.g. in silico or animal testing.
  - Discussing future prospects of stem-cell-derived models in neuroscience, considering technological advancements and potential breakthroughs.
  - Addressing remaining challenges, including technical, ethical, and funding-related issues, to give a balanced view of the field's future.

  Personal note: Basically conclude and have a key focus on that this field needs to get more funding, as brain organoids coupled with personalised medicine (also sometimes called precision medicine) and gene-editing is a very promising field of research and has the potential to cure epilepsy and other neurological diseases. Also, mention somehow neuroprosthetics again the the whole point from Mijail Demian Serruya's part of the "Augmenting Cognition" book from Markram.

  % ! ==============================================
  % ! Start of the references and appendices content
  % ! ==============================================

  \pagebreak
  \bibliographystyle{../../templates/custom-apa}
  \bibliography{references/bibliography}

\end{sloppypar}
\end{document}
