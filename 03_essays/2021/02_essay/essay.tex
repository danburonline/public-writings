\documentclass[10pt]{article}
% Packages
\usepackage{makecell} % For thicker lines
\usepackage{setspace}  % Controls line spacing
\usepackage{hhline}   % For double horizontal lines
\usepackage{colortbl}  % Add colour to LaTeX tables
\usepackage[T1]{fontenc}  % Choice of font encodings
\usepackage{tgtermes}  % Loads the TeX Gyre Termes font
\usepackage{siunitx}  % A comprehensive (SI) units package
\usepackage{tabularx, booktabs} % For advanced table layout
\usepackage{url}  % Verbatim with URL-sensitive line breaks
\usepackage{authblk}  % For author and affiliation management
\usepackage{natbib}  % A package for bibliographies and citations
\usepackage{graphicx}  % Enhances LaTeX's built-in graphics features
\usepackage{listings}  % Typeset programming code within the document
\usepackage{amssymb}  % Mathematical symbols
\usepackage[nottoc]{tocbibind}  % Adds entries to the table of contents
\usepackage{xcolor}  % Provides easy driver-independent access to colors
\usepackage{microtype}  % Improves the spacing between words and letters
\usepackage{enumitem}  % Control layout of itemize, enumerate, description
\usepackage{tocloft}  % Controls the typographic design of table of contents, etc.
\usepackage[breaklinks,linktocpage]{hyperref}  % Creates hyperlinks in your document
\usepackage[font=small,skip=7pt,labelfont=bf]{caption}  % Customising captions in floating envs

% Adjust the page margins in the bibliography
\let\oldthebibliography=\thebibliography
\let\endoldthebibliography=\endthebibliography
\renewenvironment{thebibliography}[1]{%
  \begin{oldthebibliography}{#1}%
    \raggedright%
    }{%
  \end{oldthebibliography}%
}

\setlength\bibindent{0pt}

% Optional options
% \usepackage{background} % Creates a DRAFT background image on all pages
% \backgroundsetup{contents=Preprint, opacity=0.25, color=gray} % Adds a watermark to the document
% This command changes the line spacing to double.
% ? Needed for reviews/drafts
% \doublespacing

% Custom colours
\definecolor{codegreen}{rgb}{0,0.5,0}
\definecolor{codegray}{rgb}{0.4,0.4,0.4}
\definecolor{codepurple}{rgb}{0.58,0,0.82}
\definecolor{backcolour}{rgb}{0.96,0.96,0.96}
\definecolor{lightgray}{gray}{0.8}

\lstdefinelanguage{JavaScript}{
  keywords={break, case, catch, continue, debugger, default, delete, do, else, finally, for, function, if, in, instanceof, new, return, switch, this, throw, try, typeof, var, void, while, with},
  morecomment=[l]{//},
  morecomment=[s]{/*}{*/},
  morestring=[b]',
  morestring=[b]",
  sensitive=true
}

% Listing styles
\lstdefinestyle{mystyle}{
  frame=tb,
  tabsize=2,
  captionpos=b,
  numbers=left,
  framerule=0pt,
  numbersep=5pt,
  showtabs=false,
  breaklines=true,
  keepspaces=true,
  showspaces=false,
  framextopmargin=6pt,
  framexbottommargin=6pt,
  showstringspaces=false,
  breakatwhitespace=false,
  keywordstyle=\color{purple},
  commentstyle=\color{codegreen},
  stringstyle=\color{codepurple},
  numberstyle=\tiny\color{codegray},
  basicstyle=\ttfamily\footnotesize,
  backgroundcolor=\color{backcolour}}
\lstset{style=mystyle}

% ! Custom template commands
% Add a vertical space after section numbers in ToC
\renewcommand\cftsecafterpnum{\vskip8pt}
% Changes the title of the list of listings
\renewcommand{\lstlistlistingname}{List of \lstlistingname s}
% Changes the title of the bibliography
\renewcommand{\bibname}{References}
% Changes the title of the table of contents
\renewcommand{\contentsname}{Table of Contents}
% Changes the leader between section and page numbers in ToC
\renewcommand{\cftsecleader}{\cftdotfill{\cftdotsep}}
\newcommand{\floatcaption}[2]{\caption[#1.]{#1~#2.}}

% Custom template settings
\captionsetup{justification=centering}  % All captions will be centered
\hypersetup{
  colorlinks = true,
  urlcolor = blue,
  linkcolor = blue,
  citecolor = blue,
  breaklinks = true
}
\PassOptionsToPackage{hyphens}{url}
\urlstyle{same}
\def\Urlmuskip{0mu plus 1mu}
\def\UrlBreaks{\do\/\do-}
\def\UrlBigBreaks{\do\/\do-\do:\do.}
\setlist[itemize]{noitemsep, topsep=0pt, partopsep=0pt}



\begin{document}
% Changing the initial page numbering to uppercase Roman
\pagenumbering{roman}
% Resetting the page counter to 1
\counterwithin{lstlisting}{section}
\counterwithin{figure}{section}
\counterwithin{table}{section}
% Sets the distance between the bottom and the footer
\setlength{\footskip}{65pt}

% ! ===============================
% ! Start of the title page content
% ! ===============================

\title{\textbf{WebGPU vs Pixel Streaming:} \\ A View From Afar}
\author[1]{Daniel Burger}
\affil[1]{\textbf{Middlesex University London\thanks{In collaboration with SAE Institute Zürich.}}}
\affil[ ]{\href{mailto:public@danielburger.online}{public@danielburger.online}}
\date{\textit{7. February 2021}}
\maketitle
% Resetting the page style for the first page
\thispagestyle{empty}

% The sloppypar environment adjusts the spaces between words such that each line fits into the text width
\begin{sloppypar}
  \begin{abstract}
    Two completely new technologies to develop modern graphics-focused software are on the rise. WebGPU is the successor to WebGL and offers remarkable performance improvements. However, pixel streaming goes in a completely different direction and is actively used by the gaming industry.

    In this article, we go into the near future and look at a hypothetical 3D application’s top-level architecture and argue the pros and cons of WebGPU vs pixel streaming from a developer’s perspective.
  \end{abstract}

  \pagebreak
  % Changing the page numbering back to uppercase Roman
  \pagenumbering{Roman}
  \tableofcontents
  \pagebreak
  \listoffigures
  \pagebreak
  \listoftables
  \pagebreak
  \addcontentsline{toc}{section}{\lstlistlistingname} % Add to the TOC
  \lstlistoflistings
  \pagebreak
  % Changing the page numbering back to Arabic
  \pagenumbering{arabic}

  % ! ====================================
  % ! Start of the actual document content
  % ! ====================================

  \section{Introduction}
  \label{sec:introduction}

  On the 17th of June 2015, Brendan Eich—the inventor of JavaScript—and the teams behind Mozilla, Chrome, Edge and WebKit presented a new browser standard: \mbox{WebAssembly}, a portable and highly efficient byte-code compilation target for high-level languages such as C++ and Rust \citep{eich_asmjs_2015}.

  % ! ==============================================
  % ! Start of the references and appendices content
  % ! ==============================================

  \pagebreak
  \bibliographystyle{../../templates/custom-apa}
  \bibliography{references/bibliography}

\end{sloppypar}
\end{document}
