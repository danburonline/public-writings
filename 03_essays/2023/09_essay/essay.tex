% Global document settings
\documentclass[10pt]{article}

% Packages
\usepackage{tgtermes}
\usepackage{graphicx}
\usepackage{natbib}
\usepackage{authblk}
\usepackage{array}
\usepackage{colortbl}
\usepackage{tocloft}
\usepackage{xcolor}
\usepackage{siunitx}
\usepackage{setspace}
\usepackage{listings}
\usepackage{caption}
\usepackage[T1]{fontenc}
\usepackage[nottoc]{tocbibind}
\usepackage[breaklinks]{hyperref}
\usepackage[font=small,skip=7pt]{caption}

% Custom colours
\definecolor{codegreen}{rgb}{0,0.6,0}
\definecolor{codegray}{rgb}{0.5,0.5,0.5}
\definecolor{codepurple}{rgb}{0.58,0,0.82}
\definecolor{backcolour}{rgb}{0.95,0.95,0.92}

% Listing styles
\lstdefinestyle{mystyle}{
  backgroundcolor=\color{backcolour},
  commentstyle=\color{codegreen},
  keywordstyle=\color{purple},
  numberstyle=\tiny\color{codegray},
  stringstyle=\color{codepurple},
  basicstyle=\ttfamily\footnotesize,
  breakatwhitespace=false,
  breaklines=true,
  captionpos=b,
  keepspaces=true,
  numbers=left,
  numbersep=5pt,
  showspaces=false,
  showstringspaces=true,
  showtabs=false,
  tabsize=2
  }
  \lstset{style=mystyle}

  % Custom commands
  \renewcommand{\bibname}{References} % Change bibliography title
  \renewcommand\cftsecafterpnum{\vskip8pt}
  \renewcommand{\lstlistlistingname}{List of \lstlistingname s}
  \renewcommand{\bibsection}{\section*{Bibliography}}
  \renewcommand{\contentsname}{Table of Contents}
  \renewcommand{\bibsection}{\section{\bibname}}
  \renewcommand{\cftsecleader}{\cftdotfill{\cftdotsep}}

  % Custom settings
  \captionsetup{justification=centering}
  \PassOptionsToPackage{hyphens}{url}
  \urlstyle{same}
  \def\Urlmuskip{0mu}
  \def\UrlBreaks{\do\/\do-}
  \hypersetup{
    colorlinks = true,
    urlcolor = blue,
    linkcolor = black,
    citecolor = black,
  breaklinks=true,
  pdfpagemode=UseOutlines,
  bookmarksopen=true,
  bookmarksopenlevel=2,
  bookmarksnumbered=true
  }

  \title{\textbf{Comparative Analysis of Structural and Functional Neuroimaging in Alzheimer's Disease: }A Hypothetical Multi-Modal Approach}
  \author[ ]{K23003985}
  % \affil[ ]{\textbf{King’s College London}}
  % \affil[ ]{\href{mailto:public@danielburger.online}{public@danielburger.online}}
  \date{\textit{5. December 2023}}

\begin{document}
% \pagenumbering{roman}
% \counterwithin{lstlisting}{section}
% \counterwithin{figure}{section}
% \counterwithin{table}{section}

\maketitle
% \thispagestyle{empty}

% Double spacing for feedback
\doublespacing

\begin{sloppypar} % For better line breaks
  %   \begin{abstract}
  %     To be created.
  %   \end{abstract}
  %   \pagebreak

  % \pagenumbering{Roman}
  % \tableofcontents
  % \pagebreak

  % \listoffigures
  % \pagebreak

  % \listoftables
  % \pagebreak

  % Back to normal numbering
  \pagenumbering{arabic}

  \section{Introduction}
  \label{sec:introduction}

  Alzheimer's disease (AD) stands as a tough challenge in neurodegenerative research, characterised by its insidious onset and progressive cognitive decline. This essay explores the use of advanced neuroimaging to uncover changes in AD. The main focus is on the correlation between changes and neuropathological markers of Alzheimer's disease, specifically beta-amyloid plaques and tau protein tangles.

  The aim is to critically assess the capabilities and limitations of various neuroimaging modalities, such as positron emission tomography (PET) and magnetic resonance imaging (MRI), in detecting these biomarkers. This investigation is grounded in the hypothesis that specific imaging biomarkers, detectable through these advanced techniques, are intimately linked with AD's pathophysiology. Moreover, a multi-modal imaging approach will provide a more comprehensive understanding of AD, potentially leading to earlier diagnosis and improved therapeutic strategies.

  \section{Understanding Alzheimer's Disease}
  \label{sec:alzheimers-disease}

  Alzheimer's disease (AD) is a neurodegenerative disorder characterised by the presence of neurofibrillary tangles and senile plaques in the brain \citep{petersen_alzheimers_2009}. The neuropathology of AD involves the accumulation of beta-amyloid plaques and tau protein tangles, leading to neuronal death and brain atrophy \citep{petersen_alzheimers_2009}. Clinically, AD manifests as progressive cognitive decline, memory loss, and impairment in daily functioning. The current diagnostic criteria for AD have evolved over time, with a focus on integrating biomarkers such as cerebrospinal fluid (CSF) amyloid-beta levels and neuroimaging findings into the diagnostic process.

  The neuropathological changes in AD, including the accumulation of beta-amyloid and tau proteins, are closely related to the clinical manifestations of the disease, such as memory loss and cognitive decline \citep{jack_serial_2009}. The presence of brain amyloidosis alone is not sufficient to produce cognitive decline; rather, the neurodegenerative component of AD pathology directly correlates with cognitive impairment \citep{jack_serial_2009}. Furthermore, the rate of cognitive decline in AD is driven by the rate of neurodegeneration, highlighting the importance of understanding the structural and functional changes in the brain \citep{jack_serial_2009}.

  In summary, the cardinal features of AD encompass its neuropathology, clinical manifestations, and diagnostic criteria. Understanding these features is crucial for designing neuroimaging studies to investigate the structural and functional changes associated with AD. The next chapter will focus on the objectives of the proposed study, the specific hypotheses being tested, and the rationale for the choice of biomarkers and neuroimaging techniques.

  \section{Objectives and Hypotheses of the Study}
  \label{sec:objectives-and-hypotheses}

  The primary objective of the study is to investigate the structural and functional changes associated with Alzheimer's disease (AD) using neuroimaging techniques. Specifically, the study aims to identify biomarkers that can provide insights into the neuropathological progression of AD and its clinical manifestations. The hypotheses being tested revolve around the association between specific neuroimaging biomarkers and the underlying pathophysiology of AD.

  To achieve these objectives, the study will focus on the selection of appropriate structural and functional biomarkers for each neuroimaging technique. The rationale for the choice of biomarkers lies in their potential to capture the key neuropathological changes characteristic of AD. For instance, the accumulation of beta-amyloid plaques and tau protein tangles in the brain, which are hallmarks of AD neuropathology, can be visualised using positron emission tomography (PET) imaging with radiotracers specific to amyloid and tau proteins \citep{bao_pet_2021}. Additionally, magnetic resonance imaging (MRI) can provide structural biomarkers such as hippocampal volume loss, which is associated with AD-related neurodegeneration \citep{besson_cognitive_2015}.

  Furthermore, the study will explore the potential of multi-modal neuroimaging techniques, integrating data from different imaging modalities to provide a comprehensive understanding of AD pathology. Recent advancements in deep learning approaches have shown promise in combining neuroimaging and genomics data to enhance the diagnosis and prediction of AD \citep{lin_deep_2021}. By leveraging multi-modal neuroimaging, the study aims to uncover synergistic biomarkers that can offer a more comprehensive view of AD pathology.

  The choice of biomarkers and neuroimaging techniques is underpinned by a growing body of evidence that supports their relevance in capturing the complex neuropathological changes in AD. For instance, the Alzheimer's Disease Neuroimaging Initiative (ADNI) has been instrumental in identifying and validating neuroimaging biomarkers, such as hippocampal volume loss and amyloid deposition, as indicators of preclinical AD \citep{saykin_genetic_2015}. Additionally, the integration of genetic data from ADNI has contributed to a deeper understanding of AD pathophysiology \citep{saykin_genetic_2015}.

  In summary, the study aims to leverage advanced neuroimaging techniques to identify structural and functional biomarkers associated with AD neuropathology. The hypotheses being tested revolve around the association between these biomarkers and the underlying pathophysiology of AD, with a focus on multi-modal neuroimaging approaches to provide a comprehensive understanding of the disease.

  \section{Neuroimaging Methods for the Study}
  \label{sec:neuroimaging-methods}

  The selection of neuroimaging methods for the study of Alzheimer's disease (AD) is crucial for capturing the structural and functional changes associated with the disease. The study will focus on identifying and describing the neuroimaging methods best suited for investigating the chosen structural and functional biomarkers, considering the specific research questions and objectives.

  One of the primary neuroimaging methods to be considered is positron emission tomography (PET) imaging with radiotracers specific to amyloid and tau proteins. PET imaging allows for the visualisation and quantification of beta-amyloid plaques and tau protein tangles, which are key neuropathological hallmarks of AD \citep{jack_serial_2009}. This method provides valuable insights into the distribution and accumulation of these proteins in the brain, offering a direct assessment of AD pathology.

  In addition to PET imaging, magnetic resonance imaging (MRI) plays a crucial role in capturing structural biomarkers associated with AD. MRI can provide detailed anatomical information, including measures of brain volume, cortical thickness, and hippocampal atrophy, which are indicative of neurodegenerative changes in AD \citep{cai_magnetic_2020}. Furthermore, recent advancements in MRI texture analysis have shown promise in identifying subtle microstructural alterations in the brain associated with AD pathology \citep{cai_magnetic_2020}.

  Moreover, the study will explore the potential of multi-modal neuroimaging techniques, which integrate data from different imaging modalities to provide a comprehensive understanding of AD pathology. Multi-modal approaches, such as combining PET and MRI data, can offer synergistic insights into the structural and functional changes in AD, enhancing the sensitivity and specificity of biomarker detection \citep{ran_multimodal_2022}. These methods enable the comprehensive assessment of AD-related neuropathological changes, providing a more holistic view of the disease process.

  While PET and MRI are valuable neuroimaging methods, it is essential to consider the strengths and limitations of each technique in relation to the features of the psychiatric condition studied. PET imaging, for instance, offers high sensitivity and specificity in detecting amyloid and tau pathology, but it involves exposure to ionising radiation and may have limited availability in certain clinical settings \citep{bao_pet_2021}. On the other hand, MRI provides excellent spatial resolution and does not involve radiation exposure, making it suitable for longitudinal studies and clinical applications \citep{cai_magnetic_2020}. However, MRI may have limitations in detecting specific molecular pathology compared to PET imaging.

  In summary, the selection of neuroimaging methods for the study of AD is critical for capturing the structural and functional changes associated with the disease. PET and MRI, along with multi-modal approaches, offer valuable insights into AD pathology, and understanding their strengths and limitations is essential for designing a comprehensive and effective neuroimaging study.

  \section{Justification of Study Design}
  \label{sec:justification-of-study-design}

  The study design for investigating structural and functional changes in Alzheimer's disease (AD) using neuroimaging techniques requires a robust justification based on classical and recent published work. Elaborating on how specific variables will be controlled in the study design, including detailing specific inclusion and exclusion criteria or statistical methods, will strengthen this section.

  To control potential sources of bias and confounding, the study will implement rigorous inclusion and exclusion criteria. Specifically, participants will undergo comprehensive clinical assessments, including cognitive testing, medical history evaluation, and neuroimaging scans, to ensure accurate diagnosis and characterization of AD pathology. Inclusion criteria will encompass individuals with confirmed AD pathology based on established diagnostic criteria, while exclusion criteria will account for comorbid conditions, medication effects, and other neurodegenerative disorders that may confound neuroimaging findings. Additionally, the study will match participants based on demographic variables, such as age, sex, and education level, to minimize potential confounding effects.

  Furthermore, the study will employ advanced statistical methods to address potential confounders and minimize bias. Propensity score matching will be utilized to balance the distribution of covariates between groups, ensuring that the comparison of neuroimaging biomarkers is not confounded by differences in participant characteristics. Additionally, multivariate regression models will be employed to adjust for relevant clinical and demographic variables, providing a more nuanced understanding of the associations between neuroimaging biomarkers and AD pathology.

  The limitations of the neuroimaging techniques used in the study design will also be addressed. Standardized acquisition and processing methods will be implemented across study sites to minimize measurement error and variability in imaging protocols. Quality control procedures will be established to ensure the reliability and reproducibility of neuroimaging data. Moreover, the interpretation of neuroimaging findings in the context of AD neuropathology will consider the dynamic nature of the disease process and the potential for atypical presentations, enhancing the validity of the study outcomes.

  In summary, the study design will control for potential sources of bias and confounding through rigorous inclusion and exclusion criteria, advanced statistical methods, and standardized neuroimaging protocols. By integrating classical and recent published work, the study will adopt a comprehensive approach to ensure the validity and reliability of the findings related to structural and functional changes in AD.

  \section{Conclusion}
  \label{sec:conclusion}

  In conclusion, the design of a controlled experiment using neuroimaging techniques to investigate structural and functional changes in Alzheimer's disease (AD) holds significant promise for advancing our understanding of this complex neurodegenerative condition. However, it is important to acknowledge the limitations of the study, as doing so can provide a balanced view and suggest areas for future research.

  One limitation of the study design is the potential for selection bias, as participants may be recruited from specialized clinical settings, which could impact the generalizability of the findings to broader AD populations. Additionally, the cross-sectional nature of the study may limit the ability to infer causality or temporal relationships between neuroimaging biomarkers and disease progression. Longitudinal studies are warranted to address these limitations and provide insights into the dynamic changes in AD pathology over time.

  Furthermore, the reliance on neuroimaging biomarkers as proxies for AD neuropathology may introduce measurement error and variability, which could impact the accuracy of the findings. Future research should aim to integrate multi-modal neuroimaging with other biomarkers, such as cerebrospinal fluid markers and genetic profiling, to enhance the comprehensive characterization of AD pathology.

  Despite these limitations, the insights gained from this research endeavor have the potential to inform the development of novel diagnostic and therapeutic strategies for AD. The study design serves as a model for future investigations into the structural and functional changes associated with other neurodegenerative conditions, paving the way for advancements in the field of neuroimaging and neuroscience.

  Looking ahead, future research should focus on addressing the limitations of the current study by incorporating multi-modal biomarkers, longitudinal assessments, and diverse participant populations. Additionally, the integration of advanced neuroimaging techniques with other omics data, such as genomics and proteomics, holds promise for unraveling the intricate mechanisms underlying AD pathology and identifying novel targets for intervention.

  In conclusion, while the study design has inherent limitations, it represents a significant step forward in the quest to unravel the complexities of AD and holds promise for contributing to the development of effective interventions and treatments for this debilitating condition. Acknowledging these limitations provides a balanced view and suggests areas for future research, ultimately advancing our understanding of AD and paving the way for improved clinical management and therapeutic strategies.

  \pagebreak
  \singlespacing % No need for double spacing in the references
  \bibliographystyle{references/custom-apa}
  \bibliography{references/bibliography}

\end{sloppypar}
\end{document}