% Global document settings
\documentclass[10pt]{article}

% Packages
\usepackage{tgtermes}
\usepackage{graphicx}
\usepackage{natbib}
\usepackage{authblk}
\usepackage{array}
\usepackage{colortbl}
\usepackage{tocloft}
\usepackage{xcolor}
\usepackage{siunitx}
\usepackage{setspace}
\usepackage{listings}
\usepackage{caption}
\usepackage[T1]{fontenc}
\usepackage[nottoc]{tocbibind}
\usepackage[breaklinks]{hyperref}
\usepackage[font=small,skip=7pt]{caption}

% Custom colours
\definecolor{codegreen}{rgb}{0,0.6,0}
\definecolor{codegray}{rgb}{0.5,0.5,0.5}
\definecolor{codepurple}{rgb}{0.58,0,0.82}
\definecolor{backcolour}{rgb}{0.95,0.95,0.92}

% Listing styles
\lstdefinestyle{mystyle}{
  backgroundcolor=\color{backcolour},
  commentstyle=\color{codegreen},
  keywordstyle=\color{purple},
  numberstyle=\tiny\color{codegray},
  stringstyle=\color{codepurple},
  basicstyle=\ttfamily\footnotesize,
  breakatwhitespace=false,
  breaklines=true,
  captionpos=b,
  keepspaces=true,
  numbers=left,
  numbersep=5pt,
  showspaces=false,
  showstringspaces=true,
  showtabs=false,
  tabsize=2
  }
  \lstset{style=mystyle}

  % Custom commands
  \renewcommand{\bibname}{References} % Change bibliography title
  \renewcommand\cftsecafterpnum{\vskip8pt}
  \renewcommand{\lstlistlistingname}{List of \lstlistingname s}
  \renewcommand{\bibsection}{\section*{Bibliography}}
  \renewcommand{\contentsname}{Table of Contents}
  \renewcommand{\bibsection}{\section{\bibname}}
  \renewcommand{\cftsecleader}{\cftdotfill{\cftdotsep}}

  % Custom settings
  \captionsetup{justification=centering}
  \PassOptionsToPackage{hyphens}{url}
  \urlstyle{same}
  \def\Urlmuskip{0mu}
  \def\UrlBreaks{\do\/\do-}
  \hypersetup{
    colorlinks = true,
    urlcolor = blue,
    linkcolor = black,
    citecolor = black,
  breaklinks=true,
  pdfpagemode=UseOutlines,
  bookmarksopen=true,
  bookmarksopenlevel=2,
  bookmarksnumbered=true
  }

  \title{\textbf{Comparative Analysis of Structural and Functional Neuroimaging in Alzheimer's Disease: }A Hypothetical Multi-Modal Approach}
  \author[ ]{K23003985}
  % \affil[ ]{\textbf{King’s College London}}
  % \affil[ ]{\href{mailto:public@danielburger.online}{public@danielburger.online}}
  \date{\textit{5. December 2023}}

\begin{document}
% \pagenumbering{roman}
% \counterwithin{lstlisting}{section}
% \counterwithin{figure}{section}
% \counterwithin{table}{section}

\maketitle
% \thispagestyle{empty}

% Double spacing for feedback
\doublespacing

\begin{sloppypar} % For better line breaks
  %   \begin{abstract}
  %     To be created.
  %   \end{abstract}
  %   \pagebreak

  % \pagenumbering{Roman}
  % \tableofcontents
  % \pagebreak

  % \listoffigures
  % \pagebreak

  % \listoftables
  % \pagebreak

  % Back to normal numbering
  \pagenumbering{arabic}

  \section{Introduction}
  \label{sec:introduction}

  Neurodegenerative research faces a difficult task in dealing with Alzheimer's disease (AD), which is known for its gradual onset and decline in cognitive abilities. This essay investigates a multi-modal neuroimaging approach in a hypothetical study design to uncover changes in Alzheimer's disease. The primary focus is on the relationship between changes and neuropathological markers of Alzheimer's disease, specifically beta-amyloid plaques and tau protein tangles.

  The aim is to critically assess the capabilities and limitations of various neuroimaging modalities, such as positron emission tomography (PET) and magnetic resonance imaging (MRI), in detecting these biomarkers. This investigation is grounded in the hypothesis that specific imaging biomarkers, detectable through these advanced techniques, are intimately linked with AD's pathophysiology. Moreover, a multi-modal imaging approach will provide a more comprehensive understanding of AD, potentially leading to earlier diagnosis and improved therapeutic strategies.

  \section{Understanding Alzheimer's Disease}
  \label{sec:alzheimers-disease}

  Alzheimer's disease (AD) is a complex neurodegenerative disorder characterised by a spectrum of neuropathological changes and clinical manifestations. Recent developments in understanding AD's neuropathology have shed light on the role of neuroinflammation and synaptic dysfunction, providing a more comprehensive view of the disease.

  Neuropathologically, AD is characterised by the accumulation of amyloid-beta ($A\beta$) plaques and neurofibrillary tangles, which are associated with synaptic dysfunction and neuroinflammation \citep{marcello_synaptic_2012}. Synaptic dysfunction, including alterations in neurotransmitter systems and synaptic plasticity, has emerged as a critical aspect of AD pathophysiology, contributing to cognitive decline and neuronal network disruption \citep{marcello_synaptic_2012}. Furthermore, neuroinflammation, involving microglial and astrocytic activation, has been implicated in the progression of AD, with recent findings highlighting the role of immune receptors and inflammation mediators in the disease process \citep{heneka_neuroinflammation_2015}.

  Recent studies have also elucidated the molecular mechanisms underlying synaptotoxicity and neuroinflammation in AD, emphasising the impact of these processes on disease progression \citep{marttinen_molecular_2018}. PET imaging studies have demonstrated the association between neuroinflammation and AD neuropathology, highlighting the coexistence of amyloid deposition, neurofibrillary tangles, and neuroinflammation in the brains of individuals with AD \citep{zhou_pet_2021}.

  The interplay between neuroinflammation, synaptic dysfunction, and AD neuropathology has significant implications for understanding disease mechanisms and developing potential therapeutic interventions. Recent research has emphasised the critical role of innate immune genes and the impact of neuroinflammation on synaptic plasticity and neuronal transmission in AD \citep{ransohoff_how_2016}. Additionally, the concept of AD as a synaptopathy has just recently gained attention, emphasising the dysfunction of synapses as a central feature of disease progression \citep{meftah_alzheimers_2023}.

  Moreover, rodent models of neuroinflammation have provided valuable insights into the mechanistic links between neuroinflammation and AD pathology, offering potential avenues for further exploration of therapeutic targets \citep{nazem_rodent_2015}. Recent investigations have shown that there is a clear association between neuroinflammation, synaptic impairment, and cognitive decline in Alzheimer's disease. This highlights the importance of considering the interplay between neuropathological changes, neuroinflammation, and synaptic dysfunction when designing neuroimaging studies to investigate structural and functional changes in AD.

  \section{Objectives and Hypotheses of the Study}
  \label{sec:objectives-and-hypotheses}

  The hypothetical study's primary objective is to investigate the structural and functional changes associated with Alzheimer's disease (AD) using neuroimaging techniques. Specifically, the study aims to identify biomarkers that can provide insights into the neuropathological progression of AD and its clinical manifestations. The hypotheses being tested revolve around the association between specific neuroimaging biomarkers and the underlying pathophysiology of AD.

  To achieve these objectives, the study will focus on selecting appropriate structural and functional biomarkers for each neuroimaging technique. The rationale for choosing biomarkers is their potential to capture AD's key neuropathological change characteristics. For instance, the accumulation of beta-amyloid plaques and tau protein tangles in the brain, which are hallmarks of AD neuropathology, can be visualised using positron emission tomography (PET) imaging with radiotracers specific to amyloid and tau proteins \citep{bao_pet_2021}. Additionally, magnetic resonance imaging (MRI) can provide structural biomarkers such as hippocampal volume loss associated with AD-related neurodegeneration \citep{besson_cognitive_2015}.

  Furthermore, the study will explore the potential of multi-modal neuroimaging techniques, integrating data from different imaging modalities to provide a comprehensive understanding of AD pathology. Recent advancements in deep learning approaches have shown promise in combining neuroimaging and genomics data to enhance the diagnosis and prediction of AD \citep{lin_deep_2021}. By leveraging multi-modal neuroimaging, the study aims to uncover synergistic biomarkers that can offer a more comprehensive view of AD pathology.

  The choice of biomarkers and neuroimaging techniques is underpinned by a growing body of evidence that supports their relevance in capturing the complex neuropathological changes in AD. For instance, the Alzheimer's Disease Neuroimaging Initiative (ADNI) has been instrumental in identifying and validating neuroimaging biomarkers, such as hippocampal volume loss and amyloid deposition, as indicators of preclinical AD \citep{saykin_genetic_2015}. Additionally, integrating genetic data from ADNI has contributed to a deeper understanding of AD pathophysiology \citep{saykin_genetic_2015}.

  \section{Neuroimaging Methods for the Study}
  \label{sec:neuroimaging-methods}

  The selection of neuroimaging methods for the study of Alzheimer's disease (AD) is crucial for capturing the structural and functional changes associated with the disease. The study will identify and describe the neuroimaging methods best suited for investigating the chosen structural and functional biomarkers, considering the specific research questions and objectives.

  One of the primary neuroimaging methods to be considered is positron emission tomography (PET) imaging with radiotracers specific to amyloid and tau proteins. PET imaging allows for the visualisation and quantification of beta-amyloid plaques and tau protein tangles, critical neuropathological hallmarks of AD \citep{jack_serial_2009}. This method provides valuable insights into the distribution and accumulation of these proteins in the brain, offering a direct assessment of AD pathology.

  In addition to PET imaging, magnetic resonance imaging (MRI) is crucial in capturing structural biomarkers associated with AD. MRI can provide detailed anatomical information, including measures of brain volume, cortical thickness, and hippocampal atrophy, which indicate neurodegenerative changes in AD \citep{cai_magnetic_2020}. Furthermore, recent advancements in MRI texture analysis have shown promise in identifying subtle microstructural alterations in the brain associated with AD pathology \citep{cai_magnetic_2020}.

  Moreover, the study will explore the potential of multi-modal neuroimaging techniques, which integrate data from different imaging modalities to provide a comprehensive understanding of AD pathology. Multi-modal approaches, such as combining PET and MRI data, can offer synergistic insights into AD's structural and functional changes, enhancing the sensitivity and specificity of biomarker detection \citep{ran_multimodal_2022}.

  While PET and MRI are valuable neuroimaging methods, it is essential to consider the strengths and limitations of each technique concerning the features of the psychiatric condition studied. PET imaging, for instance, offers high sensitivity and specificity in detecting amyloid and tau pathology. However, it involves exposure to ionising radiation and may have limited availability in specific clinical settings \citep{bao_pet_2021}. On the other hand, MRI provides excellent spatial resolution and does not involve radiation exposure, making it suitable for longitudinal studies and clinical applications \citep{cai_magnetic_2020}. However, MRI may have limitations in detecting specific molecular pathology compared to PET imaging.

  \section{Justification of Study Design}
  \label{sec:justification-of-study-design}

  The study design for investigating structural and functional changes in Alzheimer's disease (AD) using neuroimaging techniques requires a robust justification based on classical and recent published work. Elaborating on how specific variables will be controlled in the study design, including detailing specific inclusion and exclusion criteria or statistical methods, will strengthen this section.

  The study will implement rigorous inclusion and exclusion criteria to control potential sources of bias and confounding. Specifically, participants will undergo comprehensive clinical assessments, including cognitive testing, medical history evaluation, and neuroimaging scans, to ensure accurate diagnosis and characterisation of AD pathology. Inclusion criteria will encompass individuals with confirmed AD pathology based on established diagnostic criteria. In contrast, exclusion criteria will account for comorbid conditions, medication effects, and neurodegenerative disorders that may confound neuroimaging findings. Additionally, the study will match participants based on demographic variables, such as age, sex, and education level, to minimise potential confounding effects.

  Furthermore, the study will employ advanced statistical methods to address potential confounders and minimise bias. Propensity score matching will be utilised to balance the distribution of covariates between groups, ensuring that differences in participant characteristics do not confound the comparison of neuroimaging biomarkers. Additionally, multivariate regression models will be employed to adjust for relevant clinical and demographic variables, providing a more nuanced understanding of the associations between neuroimaging biomarkers and AD pathology.

  The limitations of the neuroimaging techniques used in the study design will also be addressed. Standardised acquisition and processing methods will be implemented across study sites to minimise measurement error and variability in imaging protocols. Quality control procedures will be established to ensure the reliability and reproducibility of neuroimaging data. Moreover, the interpretation of neuroimaging findings in the context of AD neuropathology will consider the dynamic nature of the disease process and the potential for atypical presentations, enhancing the validity of the study outcomes.

  \section{Conclusion}
  \label{sec:conclusion}

  The design of a controlled experiment using neuroimaging techniques to investigate structural and functional changes in Alzheimer's disease (AD) holds significant promise for advancing our understanding of this complex neurodegenerative condition. However, it is essential to acknowledge the study's limitations, as doing so can provide a balanced view and suggest areas for future research.

  One limitation of the study design is the potential for selection bias, as participants may be recruited from specialised clinical settings, which could impact the generalizability of the findings to broader AD populations. Additionally, the study's cross-sectional nature may limit the ability to infer causality or temporal relationships between neuroimaging biomarkers and disease progression. Longitudinal studies are warranted to address these limitations and provide insights into the dynamic changes in AD pathology over time.

  Furthermore, the reliance on neuroimaging biomarkers as proxies for AD neuropathology may introduce measurement error and variability, which could impact the accuracy of the findings. Future research should aim to integrate multi-modal neuroimaging with other biomarkers, such as cerebrospinal fluid markers and genetic profiling, to enhance the comprehensive characterisation of AD pathology.

  Despite these limitations, the insights gained from this research endeavour have the potential to inform the development of novel diagnostic and therapeutic strategies for AD. The study design serves as a model for future investigations into the structural and functional changes associated with other neurodegenerative conditions, paving the way for advancements in neuroimaging and neuroscience.

  Future research should address the current study's limitations by incorporating multi-modal biomarkers, longitudinal assessments, and diverse participant populations. Additionally, integrating advanced neuroimaging techniques with other omics data, such as genomics and proteomics, holds promise for unravelling the intricate mechanisms underlying AD pathology and identifying novel targets for intervention.

  \pagebreak
  \singlespacing % No need for double spacing in the references
  \bibliographystyle{references/custom-apa}
  \bibliography{references/bibliography}

\end{sloppypar}
\end{document}