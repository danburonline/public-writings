\documentclass[11pt]{article}

\usepackage{graphicx}
\usepackage{natbib}
\usepackage{tocloft}
\usepackage[nottoc]{tocbibind}
\usepackage{siunitx}
\renewcommand{\bibsection}{\section*{Bibliography}}
\usepackage[breaklinks]{hyperref}
\usepackage{tocloft}
\usepackage[font=small,skip=7pt]{caption}

\renewcommand{\contentsname}{Table of Contents}
\renewcommand\cftsecafterpnum{\vskip10pt}
\renewcommand{\bibsection}{\section{\bibname}}
\renewcommand{\cftsecleader}{\cftdotfill{\cftdotsep}}


\PassOptionsToPackage{hyphens}{url}
\urlstyle{same}
\def\Urlmuskip{0mu}
\def\UrlBreaks{\do\/\do-}
\hypersetup{
  colorlinks = true,
  urlcolor = blue,
  linkcolor = black,
  citecolor = blue,
  breaklinks=true,
  pdfpagemode=UseOutlines,
  bookmarksopen=true,
  bookmarksopenlevel=2,
  bookmarksnumbered=true
}

\title{Always Bet on The Web — It Will Eat The Rest}
\author{Daniel Burger}
\date{\today}

\begin{document}
\maketitle
\pagebreak

\begin{abstract}
  In recent years, the popularity of 3D on the web has increased significantly. With major players such as Google, Meta, and Apple adopting 3D technologies such as virtual reality (VR) and augmented reality (AR), developers have increasingly turned to the web as a platform for building scalable, mass-market software. One of the key advantages of using the web for this purpose is its open standard nature, which allows for flexibility and compatibility across a range of engines and technologies. In this blog post, we will explore the importance of open standards in the context of building 3D experiences on the web, and how tools like Three.js and React-Three-Fiber can facilitate this process. We will also discuss the potential implications of these developments for the future of web-based 3D experiences.
\end{abstract}
\pagebreak

\tableofcontents
\pagebreak

\section{Introduction}
\label{sec:introduction}

Open standards are crucial for the creation of scalable software that can be accessed by a wide audience. Just as we no longer build websites using proprietary programs, but instead use code, open standards allow us to use a common language for content creation without being tied to a particular technology or engine. For instance, the World Wide Web Consortium (W3C) provides a set of standards that all web browsers must adhere to in order to guarantee compatibility across different platforms. By establishing these standards, developers can create content that can be accessed by anyone on any device, regardless of the browser they are using. Open standards thus facilitate the creation of universal, accessible software.

\section{Declarative 3D}
\label{sec:declarative-3d}

Having open standards is only half of the equation; developers also need tools that allow them to easily create 3D content without having to learn complex coding languages or specialized engines like Unity or Unreal Engine 4. Fortunately, tools like Three.js (and its companion library React-Three-Fiber) make creating 3D experiences on the web an easy task even for those who are unfamiliar with coding or game development principles. By utilizing existing libraries and abstracting away some of the more complex aspects of development, Three.js makes it possible for anyone who knows JavaScript basics to quickly get up and running with their own 3D projects on the web.

\section{Conclusion}
\label{sec:conclusion}

The future of 3D content creation appears bright, partly due to initiatives like WebXR, which gives developers access to APIs from all major VR/AR headsets under a single unified standard. Combined with tools such as Three.js and React-Three-Fiber, developers now have the necessary resources to create compelling 3D experiences on any device with just a few lines of code, without the need for expensive game engines. Open standards grant us freedom from proprietary technologies, while providing access to powerful tools that make it easier than ever to create 3D experiences. This allows us to focus on developing stunning interactive worlds, rather than worrying about compatibility issues between devices or browsers. The potential applications of these advancements are vast, and the future of 3D content creation on the web looks promising.

\pagebreak
\bibliographystyle{references/custom-apa}
\bibliography{references/bibliography}
\end{document}