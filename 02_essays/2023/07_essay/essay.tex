% Global document settings
\documentclass[10pt]{article}

% Packages
\usepackage{tgtermes}
\usepackage{graphicx}
\usepackage{natbib}
\usepackage{authblk}
\usepackage{array}
\usepackage{colortbl}
\usepackage{tocloft}
\usepackage{xcolor}
\usepackage{siunitx}
\usepackage{setspace}
\usepackage{listings}
\usepackage{caption}
\usepackage[T1]{fontenc}
\usepackage[nottoc]{tocbibind}
\usepackage[breaklinks]{hyperref}
\usepackage[font=small,skip=7pt]{caption}

% Custom colours
\definecolor{codegreen}{rgb}{0,0.6,0}
\definecolor{codegray}{rgb}{0.5,0.5,0.5}
\definecolor{codepurple}{rgb}{0.58,0,0.82}
\definecolor{backcolour}{rgb}{0.95,0.95,0.92}

% Listing styles
\lstdefinestyle{mystyle}{
  backgroundcolor=\color{backcolour},
  commentstyle=\color{codegreen},
  keywordstyle=\color{purple},
  numberstyle=\tiny\color{codegray},
  stringstyle=\color{codepurple},
  basicstyle=\ttfamily\footnotesize,
  breakatwhitespace=false,
  breaklines=true,
  captionpos=b,
  keepspaces=true,
  numbers=left,
  numbersep=5pt,
  showspaces=false,
  showstringspaces=true,
  showtabs=false,
  tabsize=2
  }
  \lstset{style=mystyle}

  % Custom commands
  \renewcommand{\bibname}{References} % Change bibliography title
  \renewcommand\cftsecafterpnum{\vskip8pt}
  \renewcommand{\lstlistlistingname}{List of \lstlistingname s}
  \renewcommand{\bibsection}{\section*{Bibliography}}
  \renewcommand{\contentsname}{Table of Contents}
  \renewcommand{\bibsection}{\section{\bibname}}
  \renewcommand{\cftsecleader}{\cftdotfill{\cftdotsep}}

  % Custom settings
  \captionsetup{justification=centering}
  \PassOptionsToPackage{hyphens}{url}
  \urlstyle{same}
  \def\Urlmuskip{0mu}
  \def\UrlBreaks{\do\/\do-}
  \hypersetup{
    colorlinks = true,
    urlcolor = blue,
    linkcolor = black,
    citecolor = black,
  breaklinks=true,
  pdfpagemode=UseOutlines,
  bookmarksopen=true,
  bookmarksopenlevel=2,
  bookmarksnumbered=true
  }

  \title{\textbf{Flicking the Switch: } \\ How Optogenetics Sheds Light \\ on Motor Control Mechanisms}
  % \author[ ]{Daniel Burger}
  \author[ ]{K23003985}
  \affil[ ]{\textbf{King’s College London}}
  \affil[ ]{\href{mailto:K23003985@kcl.ac.uk}{K23003985@kcl.ac.uk}}
  \date{\textit{8. August 2023}}

\begin{document}
\pagenumbering{roman}
\counterwithin{lstlisting}{section}
\counterwithin{figure}{section}
\counterwithin{table}{section}

\maketitle
% \thispagestyle{empty}

\begin{sloppypar} % For better line breaks
  % \begin{abstract}
  %   Abstract coming soon.
  % \end{abstract}
  % \pagebreak

  % \pagenumbering{Roman}
  % \tableofcontents
  % \pagebreak

  % \listoffigures
  % \pagebreak

  % \listoftables
  % \pagebreak


  % Double spacing for feedback
  \doublespacing

  % Back to normal numbering
  \pagenumbering{arabic}

  \section{Introduction}
  \label{sec:introduction}

  Parkinson’s disease, a prevalent neurodegenerative disorder, is manifested primarily through motor symptoms such as rigidity, tremors, and bradykinesia. These symptoms are primarily attributed to the degeneration of dopaminergic neurons in the basal ganglia, a subcortical nucleus instrumental for motor control and decision-making \citep{kravitz_regulation_2010}. The traditional model of basal ganglia function proposes two opposing pathways: the direct pathway, which facilitates movement, and the indirect pathway, which inhibits movement. However, recent anatomical and functional evidence has challenged this classical model, suggesting a more complex interplay between these pathways \citep{dunovan_believer-skeptic_2016}.

  Studies employing optogenetic techniques \citep{kravitz_regulation_2010, cui_concurrent_2013} provide crucial insights into the function of these pathways. This method allows for the precise control of specific neurons using light, offering unparalleled specificity in investigating neuronal activity. Studies by \cite{kravitz_regulation_2010} and \cite{cui_concurrent_2013} have provided empirical support for the classical model of the basal ganglia while also unveiling the complexity of these pathways, suggesting that they function bidirectionally \citep{yttri_opponent_2016} to regulate motor behaviour.

  Recently, studies such as by \cite{hilt_evidence_2016} and \cite{wang_direct_2015} have further expanded our understanding of the role of these pathways in reward-based learning, motor learning, and the regulation of movement speed. These studies, utilising optogenetic methods, have underscored the dynamic interplay between the direct and indirect pathways in the basal ganglia and their impact on motor control, decision-making, reward processing, and motor learning.

  This essay will synthesise the current knowledge and debates surrounding the role of basal ganglia pathways in motor control and decision-making, focusing on the insights provided by studies using optogenetic techniques. The author will explore how dynamic competition between the direct and indirect pathways modulates motor behaviour and decision-making. Furthermore, this essay explores how alterations in these pathways contribute to the motor symptoms of Parkinson’s disease and discuss potential therapeutic strategies for alleviating these symptoms.

  \section{Cui et al. (2013) Study Overview and Findings}
  \label{sec:cui-et-al-2013}

  The study by \cite{cui_concurrent_2013} significantly advanced our understanding of the complex interplay within the basal ganglia pathways. Their research contested the traditional view that the basal ganglia’s direct and indirect pathways function independently, suggesting that these pathways interact dynamically during action selection.

  In this groundbreaking study, \citeauthor{cui_concurrent_2013} employed innovative optogenetic techniques, which allowed them to excite or inhibit specific neurons using light, thereby providing precise control over neuronal activity. They specifically targeted the medium spiny neurons (MSNs) in the dorsomedial striatum of mice. These neurons form the origin points of the direct and indirect pathways.

  The primary finding of \citeauthor{cui_concurrent_2013}’s study was that the direct and indirect pathways in the basal ganglia are concurrently activated during action initiation. They discovered that both direct-pathway MSNs (dMSNs) and indirect-pathway MSNs (iMSNs) increased their firing rates before a movement was initiated. This observation contrasts with the classical model, which suggests that the direct and indirect pathways should be activated opposingly.

  Interestingly, they observed that during movement, there was a reduction in the activity of iMSNs, which was not seen in dMSNs. This observation suggests that while both pathways are involved in action initiation, their roles may diverge during action execution. This finding of concurrent activation and divergence challenges the traditional view of the direct and indirect pathways acting as separate and opposing ‘levers’ for movement facilitation and suppression. Instead, \citeauthor{cui_concurrent_2013}’s findings point towards a more nuanced understanding where these pathways dynamically interact and compete during action decisions.

  A study by \cite{guillaumin_optogenetic_2020} further investigated the role of the direct and indirect pathways in reward-based learning. Their study found that activation of the direct pathway enhanced reward-seeking behaviour, while activation of the indirect pathway suppressed it. This provides additional support for the complex, dynamic interplay between the direct and indirect pathways in the basal ganglia.

  \section{The Role of Optogenetics in Neural Pathways}
  \label{sec:the-role-of-optogenetics-in-neural-pathways}

  Optogenetics is a revolutionary tool in neuroscience that provides unprecedented precision in controlling and observing the activity of specific neurons in living tissue. This technique utilises genetically modified neurons to express light-sensitive proteins known as opsins, enabling the manipulation of neuronal activity with high spatial and temporal resolution using light.

  In understanding the basal ganglia pathways, optogenetics has been a powerful tool in decoding the role of specific neural pathways. For instance, the study conducted by \cite{kravitz_regulation_2010} employed optogenetic techniques to activate the direct and indirect pathways in mice selectively. Their findings provided empirical support for the classical model of basal ganglia function, revealing the opposing roles of these pathways in regulating movement.

  Similarly, \cite{cui_concurrent_2013} used optogenetic techniques to measure the activity of direct and indirect pathways during movement initiation and execution. Their finding of the co-activation of both pathways during movement initiation challenged the traditional understanding of these pathways functioning independently.

  More recent studies have continued to leverage the power of optogenetics to investigate the roles of the direct and indirect pathways in the basal ganglia. \cite{yttri_opponent_2016} used cell-type-specific optogenetic stimulation to probe the bidirectional control of movement velocity by the direct and indirect pathways. \cite{guillaumin_optogenetic_2020} used optogenetics to investigate the roles of these pathways in reward-based learning, while \cite{hilt_evidence_2016} examined their roles in motor learning.

  However, it is also important to note the limitations of optogenetics. For instance, it requires genetic modifications, which can be challenging in certain settings. Additionally, delivering light to deep brain structures can be technically challenging. Despite these limitations, optogenetics has greatly enhanced our understanding of neural circuits and holds promise for future studies and potential therapeutic applications.

  Optogenetics has emerged as a powerful tool in neuroscience, providing novel insights into the functioning of neural pathways, including the direct and indirect pathways of the basal ganglia. As our understanding and usage of optogenetics continue to evolve, it promises to shed more light on the intricate workings of the brain and potentially pave the way for innovative therapeutic strategies.

  \section{Complementary and Contrasting Studies}
  \label{sec:complementary-and-contrasting-studies}

  In addition to the work by \cite{cui_concurrent_2013} and \cite{kravitz_regulation_2010}, several other studies have significantly contributed to understanding the roles of the direct and indirect pathways in the basal ganglia.

  \cite{yttri_opponent_2016} conducted a study that further explored these pathways’ roles in decision-making. They used electrophysiological recordings in monkeys to examine the activity of neurons in the direct and indirect pathways during a perceptual decision task. Their findings indicated that neurons in both pathways exhibited activity related to the decision-making process, suggesting a concurrent role for both pathways in decision formation.

  In another study, \cite{guillaumin_optogenetic_2020} investigated the role of the direct and indirect pathways in reward-based learning. They found that activation of the direct pathway enhanced reward-seeking behaviour, while activation of the indirect pathway suppressed it. These findings indicate that the direct and indirect pathways are involved in motor control and play a role in reward processing, providing further evidence of the complex interplay between these pathways.

  \cite{hilt_evidence_2016} focused on the role of the direct and indirect pathways in motor learning. They found that activation of the direct pathway facilitated motor learning, while activation of the indirect pathway impaired it. These findings suggest that the direct pathway promotes motor skill acquisition. In contrast, the indirect pathway may inhibit motor learning, highlighting the dynamic interplay between these pathways in basal ganglia function.

  Lastly, \cite{wang_direct_2015} explored the role of the direct and indirect pathways in regulating movement speed. They found that activation of the direct pathway increased movement speed, while activation of the indirect pathway decreased it. These findings provide further evidence for the opposing roles of these pathways in motor control and highlight their involvement in regulating movement speed.

  These studies collectively underscore the dynamic interaction between the direct and indirect pathways in the basal ganglia. They highlight the importance of these pathways in motor control, decision-making, reward processing, and motor learning. Furthermore, they emphasise the utility of optogenetics in investigating these neural circuits, shedding light on their functional roles and their contribution to movement disorders such as Parkinson’s disease.

  \section{Conclusion}
  \label{sec:conclusion}

  Understanding the basal ganglia’s direct and indirect pathways and their roles in motor control, decision-making, reward processing, and motor learning has significantly evolved. Recent research has challenged the traditional view of these pathways as independently functioning entities, suggesting a more intricate interplay between these pathways.

  Studies using optogenetic techniques, such as those conducted by \cite{cui_concurrent_2013}, \cite{kravitz_regulation_2010}, \cite{yttri_opponent_2016}, \cite{guillaumin_optogenetic_2020}, \cite{hilt_evidence_2016}, and \cite{wang_direct_2015}, have provided valuable insights into the functioning of these pathways. These studies collectively indicate that both the direct and indirect pathways are concurrently activated during movement initiation, playing complementary roles in decision-making and action selection. Furthermore, they highlight the importance of these pathways in reward-based learning and motor learning.

  However, the exact mechanisms underlying the dynamic interaction between the direct and indirect pathways remain to be fully clarified. Future research, aided by the continued refinement and application of optogenetic techniques, promises to shed further light on these intricate neural circuits. This will enhance our understanding of basal ganglia function and motor control and have significant implications for our understanding of movement disorders such as Parkinson’s disease and the development of potential therapeutic strategies.

  \pagebreak
  \singlespacing % No need for double spacing in the references
  \bibliographystyle{references/custom-apa}
  \bibliography{references/bibliography}

\end{sloppypar}
\end{document}