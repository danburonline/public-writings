% Global document settings
\documentclass[10pt]{article}

% Packages
\usepackage{tgtermes}
\usepackage{graphicx}
\usepackage{natbib}
\usepackage{authblk}
\usepackage{array}
\usepackage{colortbl}
\usepackage{tocloft}
\usepackage{xcolor}
\usepackage{siunitx}
\usepackage{setspace}
\usepackage{listings}
\usepackage{caption}
\usepackage[T1]{fontenc}
\usepackage[nottoc]{tocbibind}
\usepackage[breaklinks]{hyperref}
\usepackage[font=small,skip=7pt]{caption}

% Custom colours
\definecolor{codegreen}{rgb}{0,0.6,0}
\definecolor{codegray}{rgb}{0.5,0.5,0.5}
\definecolor{codepurple}{rgb}{0.58,0,0.82}
\definecolor{backcolour}{rgb}{0.95,0.95,0.92}

% Listing styles
\lstdefinestyle{mystyle}{
  backgroundcolor=\color{backcolour},
  commentstyle=\color{codegreen},
  keywordstyle=\color{purple},
  numberstyle=\tiny\color{codegray},
  stringstyle=\color{codepurple},
  basicstyle=\ttfamily\footnotesize,
  breakatwhitespace=false,
  breaklines=true,
  captionpos=b,
  keepspaces=true,
  numbers=left,
  numbersep=5pt,
  showspaces=false,
  showstringspaces=true,
  showtabs=false,
  tabsize=2
  }
  \lstset{style=mystyle}

  % Custom commands
  \renewcommand{\bibname}{References} % Change bibliography title
  \renewcommand\cftsecafterpnum{\vskip8pt}
  \renewcommand{\lstlistlistingname}{List of \lstlistingname s}
  \renewcommand{\bibsection}{\section*{Bibliography}}
  \renewcommand{\contentsname}{Table of Contents}
  \renewcommand{\bibsection}{\section{\bibname}}
  \renewcommand{\cftsecleader}{\cftdotfill{\cftdotsep}}

  % Custom settings
  \captionsetup{justification=centering}
  \PassOptionsToPackage{hyphens}{url}
  \urlstyle{same}
  \def\Urlmuskip{0mu}
  \def\UrlBreaks{\do\/\do-}
  \hypersetup{
    colorlinks = true,
    urlcolor = blue,
    linkcolor = black,
    citecolor = black,
  breaklinks=true,
  pdfpagemode=UseOutlines,
  bookmarksopen=true,
  bookmarksopenlevel=2,
  bookmarksnumbered=true
  }

  \title{\textbf{Flicking the Switch: } \\ How Optogenetics Sheds Light \\ on Motor Control Mechanisms}
  \author[ ]{Daniel Burger}
  \affil[ ]{\textbf{King’s College London}}
  \affil[ ]{\href{mailto:public@danielburger.online}{public@danielburger.online}}
  \date{\textit{8. August 2023}}

\begin{document}
\pagenumbering{roman}
\counterwithin{lstlisting}{section}
\counterwithin{figure}{section}
\counterwithin{table}{section}

% \maketitle
% \thispagestyle{empty}

\begin{sloppypar} % For better line breaks
  % \begin{abstract}
  %   Abstract coming soon.
  % \end{abstract}
  % \pagebreak

  % \pagenumbering{Roman}
  % \tableofcontents
  % \pagebreak

  % \listoffigures
  % \pagebreak

  % \listoftables
  % \pagebreak


  % Double spacing for feedback
  \doublespacing

  \pagenumbering{arabic}
  \section{Introduction}
  \label{sec:introduction}
  Parkinson’s disease, a neurodegenerative disorder, is primarily characterised by motor symptoms such as rigidity, tremors, and bradykinesia. These symptoms are primarily attributed to the degeneration of dopaminergic neurons in the basal ganglia (BG), a subcortical nuclei crucial for motor control and decision-making \citep {kravitz_regulation_2010}. Traditional models of the basal ganglia propose two opposing pathways: the direct pathway, which facilitates movement, and the indirect pathway, which inhibits movement. However, recent anatomical and functional evidence has challenged this classical model, suggesting a more complex interplay between these pathways \citep {dunovan_believer-skeptic_2016}.

  \cite {kravitz_regulation_2010} used optogenetic techniques to explore the roles of these pathways in animal models. Their findings support the classical model, demonstrating that activation of the direct pathway facilitates movement, while activation of the indirect pathway inhibits it. However, they also highlight the complexity of these pathways, suggesting that they function bidirectionally to regulate motor behaviour.

  On the other hand, \cite {dunovan_believer-skeptic_2016} propose a “Believer-Skeptic” model, arguing that the direct and indirect pathways dynamically compete during action selection, integrating contextual uncertainty with accumulating evidence. This model suggests that the basal ganglia play a crucial role in motor function, decision-making, and reinforcement learning.

  Building upon these findings, this essay will synthesise the current knowledge and debates surrounding the role of basal ganglia pathways in motor control and decision-making. We will explore how dynamic competition between the direct and indirect pathways modulates motor behaviour and decision-making. Furthermore, we will explore how alterations in these pathways contribute to the motor symptoms of Parkinson’s disease and discuss potential therapeutic strategies for alleviating these symptoms.

  \section{Cui et al. (2013) Study Overview and Findings}
  \label{sec:cui-et-al-2013}

  \cite{cui_concurrent_2013} contributed significantly to understanding the complex interactions within the basal ganglia pathways. Their research challenged the traditional view of the basal ganglia’s direct and indirect pathways functioning independently. Instead, they proposed that these pathways interact dynamically during action selection.

  In their study, Cui and colleagues made use of innovative optogenetic techniques, allowing them to excite or inhibit specific neurons using light, thereby offering precise control over neuronal activity. They targeted the medium spiny neurons (MSNs), which form the origin of the direct and indirect pathways in the dorsomedial striatum of mice.

  The primary finding of \citeauthor{cui_concurrent_2013} ’s study was that the direct and indirect pathways in the basal ganglia are co-activated during action initiation. They discovered that both direct-pathway MSNs (dMSNs) and indirect-pathway MSNs (iMSNs) increased their firing rates before a movement was initiated. This contrasts with the classical model, which suggests that the direct and indirect pathways should be activated opposingly.

  Moreover, they observed that during movement, there was a reduction in the activity of iMSNs, which was not seen in dMSNs. This suggests that while both pathways are involved in action initiation, their roles may diverge during action execution.

  This co-activation and divergence challenge the traditional view of the direct and indirect pathways acting as separate and opposing ‘levers’ for movement facilitation and suppression. Instead, \citeauthor{cui_concurrent_2013} ’s findings point towards a more nuanced understanding where these pathways dynamically interact and compete during action decisions.

  In conclusion, the study’s results by \citeauthor{cui_concurrent_2013} provide compelling evidence for the dynamic competition and interaction between the direct and indirect pathways of the basal ganglia. This enhanced understanding of basal ganglia function has significant implications for our insights into normal motor control and decision-making and our understanding of the pathophysiology of movement disorders such as Parkinson’s disease.

  The basal ganglia, a group of subcortical nuclei, play a key role in regulating motor control and decision-making. According to the classical model, the basal ganglia contain two main pathways with opposing functions: the direct pathway, which facilitates movement, and the indirect pathway, which inhibits movement \citep{kravitz_regulation_2010}.

  Activation of the direct pathway leads to disinhibition of the thalamus, releasing an “excitation” signal that promotes movement. Conversely, activation of the indirect pathway increases inhibition of the thalamus, sending a “suppression” signal that inhibits movement. This model suggests that the balance between these two pathways determines the initiation or suppression of movement.

  \section{The Role of Optogenetics in Understanding Neural Pathways}
  \label{sec:the-role-of-optogenetics-in-understanding-neural-pathways}
  Optogenetics is a revolutionary tool in neuroscience, offering unprecedented precision in controlling and observing the activity of specific neurons in living tissue. Optogenetics allows researchers to manipulate neuronal activity using precise bursts of light by genetically modifying neurons to produce light-sensitive proteins known as opsins.

  In understanding the basal ganglia pathways, optogenetics has unveiled the complex interactions within these circuits. For instance, the study conducted by \cite{kravitz_regulation_2010} employed optogenetics to selectively activate the direct and indirect pathways in mice, leading to a deeper understanding of their roles in movement regulation. Their findings provided empirical support for the classical basal ganglia functioning model while suggesting a bidirectional role for these pathways in motor control.

  Similarly, \cite{cui_concurrent_2013} employed optogenetics to observe the activity of direct and indirect pathways during movement initiation and execution. Their finding of co-activation of both pathways during movement initiation challenged the traditional understanding of these pathways functioning independently.

  These studies highlight optogenetics’s power in decoding specific neural pathways’ roles. Through the use of optogenetics, researchers can selectively manipulate individual neural pathways, providing insights into their functions that may not be discernible through traditional experimental techniques.

  Furthermore, optogenetics holds promise for developing novel treatment strategies for neurological disorders. For instance, in Parkinson’s disease, where dysregulation of basal ganglia pathways leads to motor symptoms, optogenetics could potentially be used to restore normal activity within these pathways.

  In conclusion, optogenetics has emerged as a powerful tool in neuroscience, providing novel insights into the functioning of neural pathways. As our understanding and usage of optogenetics continue to evolve, it promises to shed more light on the intricate workings of the brain and potentially pave the way for innovative therapeutic strategies.

  \section{Complementary and Contrasting Studies}
  \label{sec:complementary-and-contrasting-studies}
  The study by \cite{cui_concurrent_2013} challenged the classical model of basal ganglia function, which proposes that the direct pathway facilitates movement while the indirect pathway inhibits it. Using a novel in vivo method, they could precisely measure the activity of both pathways in freely moving mice. The findings showed that both pathways were active when animals initiated actions, suggesting that they do not function independently but work concurrently in action initiation.

  This idea of concurrent activation of the direct and indirect pathways is consistent with the “Believer-Skeptic” framework proposed by \cite{dunovan_believer-skeptic_2016}. They suggest that these pathways represent two opposing arguments in decision-making, with the direct pathway (the “Believer”) promoting the execution of an action and the indirect pathway (the “Skeptic”) suppressing the execution of an action. According to this framework, the direct and indirect pathways work together in a dynamic competition to reach a decision.

  In a related study, \cite{kravitz_regulation_2010} also used optogenetic control to regulate Parkinsonian motor behaviours in mice. Their findings supported the opposing roles of the direct and indirect pathways in motor control, with activation of the direct pathway facilitating movement and activation of the indirect pathway leading to motoric freezing. This study provided significant evidence supporting the classical model of basal ganglia function.

  In conclusion, while the classical model of basal ganglia function suggests opposing roles for the direct and indirect pathways in motor control, more recent studies indicate that these pathways may work concurrently during action initiation and decision-making processes. Optogenetics has been instrumental in these studies, allowing for the selective activation and measurement of specific neural pathways.

  \pagebreak
  \singlespacing % No need for double spacing in the references
  \bibliographystyle{references/custom-apa}
  \bibliography{references/bibliography}

\end{sloppypar}
\end{document}