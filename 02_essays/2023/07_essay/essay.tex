% Global document settings
\documentclass[10pt]{article}

% Packages
\usepackage{tgtermes}
\usepackage{graphicx}
\usepackage{natbib}
\usepackage{authblk}
\usepackage{array}
\usepackage{colortbl}
\usepackage{tocloft}
\usepackage{xcolor}
\usepackage{siunitx}
\usepackage{setspace}
\usepackage{listings}
\usepackage{caption}
\usepackage[T1]{fontenc}
\usepackage[nottoc]{tocbibind}
\usepackage[breaklinks]{hyperref}
\usepackage[font=small,skip=7pt]{caption}

% Custom colours
\definecolor{codegreen}{rgb}{0,0.6,0}
\definecolor{codegray}{rgb}{0.5,0.5,0.5}
\definecolor{codepurple}{rgb}{0.58,0,0.82}
\definecolor{backcolour}{rgb}{0.95,0.95,0.92}

% Listing styles
\lstdefinestyle{mystyle}{
  backgroundcolor=\color{backcolour},
  commentstyle=\color{codegreen},
  keywordstyle=\color{purple},
  numberstyle=\tiny\color{codegray},
  stringstyle=\color{codepurple},
  basicstyle=\ttfamily\footnotesize,
  breakatwhitespace=false,
  breaklines=true,
  captionpos=b,
  keepspaces=true,
  numbers=left,
  numbersep=5pt,
  showspaces=false,
  showstringspaces=true,
  showtabs=false,
  tabsize=2
  }
  \lstset{style=mystyle}

  % Custom commands
  \renewcommand{\bibname}{References} % Change bibliography title
  \renewcommand\cftsecafterpnum{\vskip8pt}
  \renewcommand{\lstlistlistingname}{List of \lstlistingname s}
  \renewcommand{\bibsection}{\section*{Bibliography}}
  \renewcommand{\contentsname}{Table of Contents}
  \renewcommand{\bibsection}{\section{\bibname}}
  \renewcommand{\cftsecleader}{\cftdotfill{\cftdotsep}}

  % Custom settings
  \captionsetup{justification=centering}
  \PassOptionsToPackage{hyphens}{url}
  \urlstyle{same}
  \def\Urlmuskip{0mu}
  \def\UrlBreaks{\do\/\do-}
  \hypersetup{
    colorlinks = true,
    urlcolor = blue,
    linkcolor = black,
    citecolor = black,
  breaklinks=true,
  pdfpagemode=UseOutlines,
  bookmarksopen=true,
  bookmarksopenlevel=2,
  bookmarksnumbered=true
  }

  \title{\textbf{Flicking the Switch: } \\ How Optogenetics Sheds Light \\ on Motor Control Mechanisms}
  % \author[ ]{Daniel Burger}
  \author[ ]{K23003985}
  \affil[ ]{\textbf{King’s College London}}
  \affil[ ]{\href{mailto:K23003985@kcl.ac.uk}{K23003985@kcl.ac.uk}}
  \date{\textit{8. August 2023}}

\begin{document}
\pagenumbering{roman}
\counterwithin{lstlisting}{section}
\counterwithin{figure}{section}
\counterwithin{table}{section}

\maketitle
% \thispagestyle{empty}

\begin{sloppypar} % For better line breaks
  % \begin{abstract}
  %   Abstract coming soon.
  % \end{abstract}
  % \pagebreak

  % \pagenumbering{Roman}
  % \tableofcontents
  % \pagebreak

  % \listoffigures
  % \pagebreak

  % \listoftables
  % \pagebreak


  % Double spacing for feedback
  \doublespacing

  % Back to normal numbering
  \pagenumbering{arabic}

  \section{Introduction}
  \label{sec:introduction}

  Parkinson’s disease, a prevalent neurodegenerative disorder, prominently manifests through motor symptoms such as rigidity, tremors, and bradykinesia. These symptoms are primarily attributed to the degeneration of dopaminergic neurons in the basal ganglia, a subcortical nucleus pivotal for motor control and decision-making \citep{kravitz_regulation_2010}. The conventional model of basal ganglia function posits two antagonistic pathways: the direct pathway, promoting movement, and the indirect pathway, inhibiting movement. Alterations in the balance and function of these pathways are thought to contribute to the motor deficits seen in Parkinson’s disease. However, this binary model has been challenged by recent evidence, suggesting a more nuanced interaction between these pathways with potential concurrent activation and divergence \citep{dunovan_believer-skeptic_2016}.

  The advent of optogenetic techniques, which allow for the precise manipulation of specific neurons using light, has significantly advanced our understanding of these intricate basal ganglia pathways \citep{kravitz_regulation_2010, cui_concurrent_2013}. Studies by \cite{kravitz_regulation_2010} and \cite{cui_concurrent_2013} have not only provided empirical support for the traditional model but have also unveiled the complexity of these pathways, suggesting that they may function bidirectionally \citep{yttri_opponent_2016} to regulate motor behaviour.

  More recently, studies such as those by \cite{hilt_evidence_2016} and \cite{wang_direct_2015} have further expanded our understanding of the role of these pathways in reward-based learning, motor learning, and the regulation of movement speed. These investigations, employing optogenetic methods, have underscored the dynamic interplay between the direct and indirect pathways in the basal ganglia and their impact on motor control, decision-making, reward processing, and motor learning.

  Armed with this context, this essay aims to synthesise the current knowledge and debates surrounding the role of basal ganglia pathways in motor control and decision-making, focusing on the insights provided by optogenetic studies. Furthermore, this essay will explore the implications of alterations in these pathways in the manifestation of motor symptoms in Parkinson’s disease. This exploration will provide a comprehensive understanding of the basal ganglia’s role in motor control, offering insights into potential therapeutic strategies for mitigating the motor symptoms of Parkinson’s disease and proposing areas for future research.

  \section{Overview and Findings of Cui et al. (2013) Study}
  \label{sec:cui-et-al-2013}

  The study by \cite{cui_concurrent_2013} marked a significant step forward in understanding basal ganglia pathways. The researchers contested the traditional model, which posits the direct and indirect pathways of the basal ganglia as independent entities. They suggested a more dynamic interaction between them during action selection.

  To investigate this, \citeauthor{cui_concurrent_2013} utilised optogenetic techniques, a ground-breaking method that allows the manipulation of specific neurons using light, providing unparalleled control over neuronal activity. Their focus was on the medium spiny neurons (MSNs) located in the dorsomedial striatum of mice, which form the origin points of the direct and indirect pathways.

  The most significant revelation from \citeauthor{cui_concurrent_2013} ’s study was the concurrent activation of the direct and indirect pathways during action initiation. They found that both direct-pathway MSNs (dMSNs) and indirect-pathway MSNs (iMSNs) increased their firing rates before initiating a movement, which counters the classical model’s assertion of opposing activation of these pathways.

  Interestingly, they observed a decrease in the activity of iMSNs during movement, a change not seen in dMSNs. This finding indicates that while both pathways are involved in action initiation, their roles may diverge during action execution. Additionally, \cite{cui_concurrent_2013} found that both dMSNs and iMSNs were quiet during inactive states when the mice were not moving, further challenging the traditional model.

  These findings were further supported by a study by \cite{guillaumin_optogenetic_2020}, which delved into the role of the direct and indirect pathways in reward-based learning. The researchers found that activating the direct pathway enhanced reward-seeking behaviour while activating the indirect pathway suppressed it. This lends further credence to the complex, dynamic interaction between the direct and indirect pathways in the basal ganglia.

  \section{Optogenetics in Decoding Neural Pathways}
  \label{sec:the-role-of-optogenetics-in-neural-pathways}

  Optogenetics, a revolutionary tool in neuroscience, employs light-sensitive proteins known as opsins to control and observe specific neurons’ activity in living tissue. This technique, offering unprecedented precision, has greatly enhanced our understanding of the basal ganglia pathways.

  It was \cite{kravitz_regulation_2010} who first employed optogenetics to selectively activate the direct and indirect pathways in mice, providing empirical support for the classical model of basal ganglia function by revealing the opposing roles of these pathways in regulating movement.

  Later, \cite{cui_concurrent_2013} utilised optogenetic techniques to measure the activity of direct and indirect pathways during movement initiation and execution. Their findings contradicted the traditional model by showing that both pathways are concurrently activated during movement initiation.

  More recently, researchers have continued to harness the power of optogenetics to further our understanding of the basal ganglia’s direct and indirect pathways. For example, \cite{yttri_opponent_2016} used cell-type-specific optogenetic stimulation to investigate the bidirectional control of movement velocity and provided evidence that both pathways can bidirectionally regulate velocity. Additionally, \cite{guillaumin_optogenetic_2020} explored the roles of these pathways in reward-based learning and found they have opposing effects on reward-seeking behaviour.

  However, it is crucial to acknowledge the limitations of optogenetics. The need for genetic modifications can pose challenges in certain settings, and delivering light to deep brain structures can be technically challenging. Despite these limitations, optogenetics has greatly enhanced our understanding of neural circuits and holds promise for future studies and potential therapeutic applications.

  In conclusion, optogenetics has emerged as a powerful tool in neuroscience, providing novel insights into the functioning of neural pathways, including the direct and indirect pathways of the basal ganglia. As our understanding and application of optogenetics continue to evolve, we expect it to shed more light on the intricate workings of the brain and potentially pave the way for innovative therapeutic strategies.

  \section{Complementary and Contrasting Studies}
  \label{sec:complementary-and-contrasting-studies}

  Beyond the work of \cite{cui_concurrent_2013} and \cite{kravitz_regulation_2010}, several other studies have significantly contributed to our understanding of the direct and indirect pathways in the basal ganglia.

  \cite{yttri_opponent_2016} further explored these pathways’ roles in decision-making using electrophysiological recordings in monkeys during a perceptual decision task. They discovered that neurons in both pathways were active in the decision-making process, suggesting a concurrent role for both pathways in decision formation. This complements Cui et al.’s finding of concurrent activation during action initiation.

  In a different vein, \cite{guillaumin_optogenetic_2020} investigated the role of these pathways in reward-based learning. They found that while the direct pathway’s activation enhanced reward-seeking behaviour, the indirect pathway’s activation suppressed it. These findings add another layer to our understanding, showing that the direct and indirect pathways are involved in motor control and reward processing. This contrasts with Cui et al.’s observation that both pathways activate prior to movement initiation.

  \cite{hilt_evidence_2016} turned their focus to motor learning. They found that the direct pathway’s activation facilitated motor learning while the indirect pathway’s activation impaired it. This research suggests that the direct pathway promotes motor skill acquisition, while the indirect pathway may inhibit it, demonstrating the dynamic interplay between these pathways in basal ganglia function.

  Lastly, \cite{wang_direct_2015} explored how these pathways regulate movement speed. They found that the direct pathway’s activation increased movement speed while the indirect pathway’s activation decreased it. These findings provide additional evidence for the opposing roles of these pathways in motor control and underscore their involvement in regulating movement speed.

  Collectively, these studies emphasise the dynamic interaction between the basal ganglia’s direct and indirect pathways. They highlight these pathways’ critical role in motor control, decision-making, reward processing, and motor learning and demonstrate the utility of optogenetics in investigating these neural circuits. These studies have shed light on the functional roles of these pathways and their contribution to movement disorders like Parkinson’s disease.

  \section{Conclusion}
  \label{sec:conclusion}

  Our understanding of the basal ganglia’s direct and indirect pathways, including their roles in motor control, decision-making, reward processing, and motor learning, has evolved significantly. Recent research has moved beyond the traditional view of these pathways as independently functioning entities, highlighting a complex, dynamic interplay between them.

  Studies employing optogenetic techniques, such as those conducted by \cite{cui_concurrent_2013}, \cite{kravitz_regulation_2010}, \cite{yttri_opponent_2016}, \cite{guillaumin_optogenetic_2020}, \cite{hilt_evidence_2016}, and \cite{wang_direct_2015}, have provided invaluable insights. These studies collectively suggest that the direct and indirect pathways are concurrently activated during movement initiation and play complementary roles in decision-making and action selection. They also underscore the significance of these pathways in reward-based learning and motor learning.

  Despite these advances, the mechanisms underlying the dynamic interaction between the direct and indirect pathways are yet to be fully elucidated. Key questions remain regarding how these pathways coordinate and compete during action selection and how their activity is modulated during learning. As the refinement and application of optogenetic techniques continue, future research is promising to shed further light on these intricate neural circuits. Such advancements will enhance our understanding of basal ganglia function and motor control and have profound implications for understanding movement disorders like Parkinson’s disease and developing potential therapeutic strategies.

  \pagebreak
  \singlespacing % No need for double spacing in the references
  \bibliographystyle{references/custom-apa}
  \bibliography{references/bibliography}

\end{sloppypar}
\end{document}