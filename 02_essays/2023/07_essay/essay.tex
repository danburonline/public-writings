% Global document settings
\documentclass[10pt]{article}

% Packages
\usepackage{tgtermes}
\usepackage{graphicx}
\usepackage{natbib}
\usepackage{authblk}
\usepackage{array}
\usepackage{colortbl}
\usepackage{tocloft}
\usepackage{xcolor}
\usepackage{siunitx}
\usepackage{setspace}
\usepackage{listings}
\usepackage{caption}
\usepackage[T1]{fontenc}
\usepackage[nottoc]{tocbibind}
\usepackage[breaklinks]{hyperref}
\usepackage[font=small,skip=7pt]{caption}

% Custom colours
\definecolor{codegreen}{rgb}{0,0.6,0}
\definecolor{codegray}{rgb}{0.5,0.5,0.5}
\definecolor{codepurple}{rgb}{0.58,0,0.82}
\definecolor{backcolour}{rgb}{0.95,0.95,0.92}

% Listing styles
\lstdefinestyle{mystyle}{
  backgroundcolor=\color{backcolour},
  commentstyle=\color{codegreen},
  keywordstyle=\color{purple},
  numberstyle=\tiny\color{codegray},
  stringstyle=\color{codepurple},
  basicstyle=\ttfamily\footnotesize,
  breakatwhitespace=false,
  breaklines=true,
  captionpos=b,
  keepspaces=true,
  numbers=left,
  numbersep=5pt,
  showspaces=false,
  showstringspaces=true,
  showtabs=false,
  tabsize=2
  }
  \lstset{style=mystyle}

  % Custom commands
  \renewcommand{\bibname}{References} % Change bibliography title
  \renewcommand\cftsecafterpnum{\vskip8pt}
  \renewcommand{\lstlistlistingname}{List of \lstlistingname s}
  \renewcommand{\bibsection}{\section*{Bibliography}}
  \renewcommand{\contentsname}{Table of Contents}
  \renewcommand{\bibsection}{\section{\bibname}}
  \renewcommand{\cftsecleader}{\cftdotfill{\cftdotsep}}

  % Custom settings
  \captionsetup{justification=centering}
  \PassOptionsToPackage{hyphens}{url}
  \urlstyle{same}
  \def\Urlmuskip{0mu}
  \def\UrlBreaks{\do\/\do-}
  \hypersetup{
    colorlinks = true,
    urlcolor = blue,
    linkcolor = black,
    citecolor = black,
  breaklinks=true,
  pdfpagemode=UseOutlines,
  bookmarksopen=true,
  bookmarksopenlevel=2,
  bookmarksnumbered=true
  }

  \title{\textbf{Flicking the Switch: } \\ Optogenetics and the Interplay of Direct and \\ Indirect Pathways in Motor Control}
  \author[ ]{Daniel Burger}
  \affil[ ]{\textbf{King’s College London}}
  \affil[ ]{\href{mailto:public@danielburger.online}{public@danielburger.online}}
  \date{\textit{8. August 2023}}

\begin{document}
\pagenumbering{roman}
\counterwithin{lstlisting}{section}
\counterwithin{figure}{section}
\counterwithin{table}{section}

\maketitle
\thispagestyle{empty}

% Double spacing for feedback
% \doublespacing

\begin{sloppypar} % For better line breaks
  \begin{abstract}
    This essay presents a comprehensive synthesis of the current understanding of basal ganglia pathways, focusing on direct and indirect neural circuits. It critically evaluates the traditional dichotomous model and examines recent evidence suggesting a nuanced, dynamic interaction between the two pathways in motor control, decision-making, reward processing, and motor learning. Utilising key studies employing optogenetic techniques, the essay contributes to a deeper understanding of these pathways’ concurrent activation and divergence during action initiation and execution.

    Moreover, the essay delves into the implications of alterations in these pathways for the manifestation of Parkinson’s disease symptoms and potential therapeutic strategies for mitigating these symptoms. It also outlines the significant contributions of optogenetics to our knowledge of these pathways, underscoring the technique’s power and precision. Despite significant advancements in understanding basal ganglia circuit dynamics, the essay highlights open questions regarding the precise mechanisms coordinating pathway interactions during action selection and learning, underlining the need for continued research.
  \end{abstract}
  \pagebreak

  \pagenumbering{Roman}
  \tableofcontents
  \pagebreak

  \listoffigures
  \pagebreak

  % \listoftables
  % \pagebreak

  % Back to normal numbering
  \pagenumbering{arabic}

  \section{Introduction}
  \label{sec:introduction}

  Parkinson’s disease, a prevalent neurodegenerative disorder, primarily affects the basal ganglia, a group of subcortical nuclei in the brain. These nuclei, which include key structures such as the striatum, globus pallidus, subthalamic nucleus, and substantia nigra, play crucial roles in motor control, decision-making, and reward processing \citep{zhang_oculomotor_2018,ojagbemi_neuropsychiatric_2013}.

  The motor symptoms of Parkinson’s disease, such as rigidity, tremors, and bradykinesia, are often attributed to the degeneration of dopaminergic neurons in the substantia nigra, a critical component of the basal ganglia \citep{abedini_cooccurrence_2015}. The resultant disruption in the balance and functioning of the basal ganglia’s direct and indirect pathways, which are primarily composed of medium spiny neurons (MSNs), has significant implications for motor control \cite{abedini_cooccurrence_2015,ojagbemi_neuropsychiatric_2013}.

  Traditionally, the direct and indirect pathways of the basal ganglia are thought to function antagonistically: the direct pathway facilitates movement, while the indirect pathway inhibits it \citep{isett_indirect_2022}. However, this binary model has been challenged by recent studies, which suggest a more complex interaction between these pathways. For instance, these pathways may show potential concurrent activation and divergence in certain motor actions, adding complexity to our understanding of their roles in voluntary movement control \citep{perez_striatal_2017}.

  \begin{figure}[ht]
    \centering
    \includegraphics[width=\textwidth]{figures/optogenetics.png}
    \caption[A rat equipped with an optogenetic implant]{\textbf{A rat equipped with an optogenetic implant.} The probe seen here, part of an optogenetic implant, can measure or manipulate neuronal signals, offering precise control over specific neural pathways. Image credit: The New York Times \citep{belluck_risky_2016}.}
    \label{fig:optogenetics}
  \end{figure}

  The advent of optogenetics, a technique allowing precise manipulation of specific neurons using light, has significantly advanced our understanding of the basal ganglia pathways \citep{deisseroth_next-generation_2006}. Optogenetics involves the genetic modification of neurons to express light-sensitive proteins known as opsins. These opsins can be activated or inhibited by exposure to certain wavelengths of light via an implanted device, as depicted in \autoref{fig:optogenetics}. Notably, Kravitz et al. used optogenetics to activate direct and indirect pathway MSNs selectively, providing key insights into their functions and contribution to the traditional model of the basal ganglia \citep{kravitz_regulation_2010}.

  Subsequent optogenetic investigations have proposed that these pathways may function bidirectionally, with the specific function depending on the context or the motor behaviour being executed \citep{yttri_opponent_2016}. This revelation, along with recent studies expanding our understanding of these pathways’ roles in reward-based learning, motor learning, and movement velocity regulation \citep{hilt_evidence_2016, wang_direct_2015}, underscores the complexity of basal ganglia pathways.

  This essay, therefore, aims to synthesise our current knowledge of the basal ganglia pathways’ role in motor control and decision-making, focusing on the insights provided by optogenetic studies. The following section will delve into significant research that has challenged the traditional model of these pathways.

  \section{Challenging the Traditional Model}
  \label{sec:challenging-the-traditional-model}

  One pivotal study that has advanced our understanding of the basal ganglia pathways is by \cite{cui_concurrent_2013}. The researchers proposed a more dynamic interaction between the direct and indirect pathways during action selection, a significant departure from the conventional understanding.

  \cite{cui_concurrent_2013} employed optogenetic techniques to explore this interaction. They focused on the medium spiny neurons (MSNs) in the dorsomedial striatum of mice, the origin points of the direct and indirect pathways, and manipulated neuronal activity precisely to study its effects on action selection.

  \begin{figure}[ht]
    \centering
    \includegraphics[width=\textwidth]{figures/direct-indirect-activation.png}
    \caption[Concurrent activation of direct and indirect pathways during action initiation]{\textbf{Concurrent activation of direct and indirect pathways during action initiation.} This figure, based on data from \cite{cui_concurrent_2013} ’s study, demonstrates the concurrent activation of direct-pathway medium spiny neurons (dMSNs) and indirect-pathway medium spiny neurons (iMSNs) at the start of a lever-pressing session. The fluorescence traces indicate increased activity for both dMSNs and iMSNs at the session start, challenging the classical model’s assertion of opposing activation of these pathways.}
    \label{fig:pathway-activation}
  \end{figure}

  A key revelation from their study was the concurrent activation of both the direct and indirect pathways during the initiation of a movement. As illustrated in \autoref{fig:pathway-activation}, both types of neurons increased their firing rates at the start of a lever-pressing session, contradicting the conventional model’s assertion of opposing activation of these pathways.

  Moreover, they observed a decrease in the activity of iMSNs during the actual movement execution, whereas the activity in dMSNs remained unchanged. This divergence in activity suggests that although both pathways are involved in action initiation, their roles may differ during action execution.

  Interestingly, both types of neurons were inactive when the rats were not moving, suggesting a more nuanced interplay between the direct and indirect pathways than previously thought.

  In addition to these findings, \cite{cui_concurrent_2013} examined the role of these pathways in contralateral vs. ipsilateral movement control. Understanding the differences in these roles is crucial, as the basal ganglia, like many brain structures, control movements on the opposite side of the body (contralateral).

  These groundbreaking findings from \cite{cui_concurrent_2013} were further supported by a study conducted by \cite{guillaumin_experimental_2021}. They explored the role of the direct and indirect pathways in reward-based learning, a process where an individual learns to perform certain actions based on the reward they receive. This differs from operant conditioning, where an individual learns to associate an action with a consequence. Their study found that activating the direct pathway enhanced reward-seeking behaviour while activating the indirect pathway suppressed it, reinforcing the view of a complex, dynamic interaction between the direct and indirect pathways in the basal ganglia.

  \section{Complementary and Contrasting Studies}
  \label{sec:complementary-and-contrasting-studies}

  This chapter explores recent studies that have furthered our understanding of the basal ganglia pathways, each adding a unique perspective to the findings of \cite{cui_concurrent_2013}.

  In a similar vein to Cui et al., \cite{yttri_opponent_2016} used optogenetic stimulation to explore the role of these pathways in the bidirectional control of movement velocity. They found that both pathways can regulate movement velocity and speed, suggesting a capacity for cooperative action that extends beyond the concurrent activation observed by \cite{cui_concurrent_2013} during action initiation.

  \cite{guillaumin_experimental_2021} took a different approach, investigating the role of these pathways in reward-based learning. They found that direct pathway activation enhanced reward-seeking behaviour while indirect pathway activation suppressed it. This indicates that both pathways also participate in reward processing. This cognitive function shares neural circuits with motor control, highlighting the complexity and versatility of these pathways beyond the context of movement initiation.

  \cite{hilt_evidence_2016} focused on motor learning, finding that activation of the direct pathway facilitated motor learning, while activation of the indirect pathway impaired it. This apparent contrast to \cite{cui_concurrent_2013} ’s findings may reflect the different neural dynamics and populations engaged in motor learning and action initiation, underscoring the diverse roles of these pathways in various aspects of motor control.

  Lastly, \cite{wang_direct_2015} explored how these pathways regulate movement speed control. They found that the direct pathway facilitated an increase in speed, while the indirect pathway led to a decrease. This observation buttresses \cite{cui_concurrent_2013} ’s findings of the concurrent activation of both pathways during action initiation, suggesting that these pathways might work together to fine-tune motor parameters such as speed during action execution.

  In summary, these studies provide a broader perspective on the diverse roles of the direct and indirect pathways in the basal ganglia, from motor control to reward processing, enriching our understanding of their complex dynamics and interactions.

  \section{Optogenetics in Decoding Neural Pathways}
  \label{sec:the-role-of-optogenetics-in-neural-pathways}

  Optogenetics, a revolutionary tool in neuroscience, employs light-sensitive proteins known as opsins to control and observe specific neurons’ activity in living tissue. This technique offers unprecedented precision and has greatly enhanced our understanding of the basal ganglia pathways. Importantly, optogenetics operates based on genetic modifications, allowing opsins to be selectively expressed in specific types of neurons. This is typically achieved through viral vectors or transgenic animals, where opsins are introduced under the control of neuron-specific promoters.

  \cite{yttri_opponent_2016} provides an excellent example of how optogenetics can be applied for detailed investigation of the basal ganglia pathways. They employed this technique to explore the complexity of how these pathways can bidirectionally regulate movement velocity, thereby extending our understanding beyond the conventional model of antagonistic pathway function.

  On the other hand, \cite{guillaumin_experimental_2021} used optogenetics to delve into the motivational aspects of behaviour, revealing the involvement of both pathways in reward processing. This shows how optogenetics can broaden our perspective, highlighting the application of these pathways beyond the context of movement initiation.

  Despite the valuable insights optogenetics provides, it is important to acknowledge its limitations. The necessity for genetic modifications can pose challenges, especially in non-model organisms, and delivering light to deep brain structures can be technically challenging. Moreover, neurons’ artificial activation or inhibition may not fully represent the complex dynamics of natural neuronal activity.

  Nonetheless, optogenetics has undeniably revolutionised neuroscience, providing novel insights into the functioning of neural pathways, including the direct and indirect pathways of the basal ganglia. As our understanding and application of optogenetics continue to evolve, we can expect it to shed more light on the brain’s complex workings and potentially contribute to developing treatments for neurological disorders.

  In conclusion, optogenetics has emerged as a powerful tool, providing novel insights into the functioning of neural pathways, including the direct and indirect pathways of the basal ganglia. Future advancements in this technique promise to further our understanding of these pathways and their intricate roles in motor control and decision-making.

  \section{Conclusion}
  \label{sec:conclusion}

  Collectively, these studies emphasise the multifaceted involvement of the direct and indirect pathways in various functions like reward processing, motor learning, and precise motor control. They demonstrate the utility of optogenetics in elucidating the nuanced interactions between these pathways.

  Our understanding of the basal ganglia’s direct and indirect pathways, including their diverse roles in motor control, decision-making, reward processing, and motor learning, has advanced significantly in recent years. While earlier models viewed these pathways in strict dichotomous terms, recent research highlights a more dynamic, nuanced interaction between them.

  However, open questions remain regarding the precise mechanisms coordinating pathway interactions during action selection and learning. As optogenetic techniques continue to be refined, future research can build on these discoveries to further elucidate the complexities of basal ganglia circuit dynamics and functions. Such advancements will have profound implications for understanding diseases involving basal ganglia dysfunction.

  In conclusion, while early models painted a binary picture of direct and indirect pathway opposition, contemporary evidence points to a more collaborative, context-dependent interaction between these pathways in functions like motor control and learning. Continued research into these complex neural circuits promises exciting breakthroughs in comprehending the basal ganglia’s multifaceted roles and neurologic disease.

  \pagebreak
  \singlespacing % No need for double spacing in the references
  \bibliographystyle{references/custom-apa}
  \bibliography{references/bibliography}

\end{sloppypar}
\end{document}