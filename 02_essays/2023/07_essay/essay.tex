% Global document settings
\documentclass[10pt]{article}

% Packages
\usepackage{tgtermes}
\usepackage{graphicx}
\usepackage{natbib}
\usepackage{authblk}
\usepackage{array}
\usepackage{colortbl}
\usepackage{tocloft}
\usepackage{xcolor}
\usepackage{siunitx}
\usepackage{setspace}
\usepackage{listings}
\usepackage{caption}
\usepackage[T1]{fontenc}
\usepackage[nottoc]{tocbibind}
\usepackage[breaklinks]{hyperref}
\usepackage[font=small,skip=7pt]{caption}

% Custom colours
\definecolor{codegreen}{rgb}{0,0.6,0}
\definecolor{codegray}{rgb}{0.5,0.5,0.5}
\definecolor{codepurple}{rgb}{0.58,0,0.82}
\definecolor{backcolour}{rgb}{0.95,0.95,0.92}

% Listing styles
\lstdefinestyle{mystyle}{
  backgroundcolor=\color{backcolour},
  commentstyle=\color{codegreen},
  keywordstyle=\color{purple},
  numberstyle=\tiny\color{codegray},
  stringstyle=\color{codepurple},
  basicstyle=\ttfamily\footnotesize,
  breakatwhitespace=false,
  breaklines=true,
  captionpos=b,
  keepspaces=true,
  numbers=left,
  numbersep=5pt,
  showspaces=false,
  showstringspaces=true,
  showtabs=false,
  tabsize=2
  }
  \lstset{style=mystyle}

  % Custom commands
  \renewcommand{\bibname}{References} % Change bibliography title
  \renewcommand\cftsecafterpnum{\vskip8pt}
  \renewcommand{\lstlistlistingname}{List of \lstlistingname s}
  \renewcommand{\bibsection}{\section*{Bibliography}}
  \renewcommand{\contentsname}{Table of Contents}
  \renewcommand{\bibsection}{\section{\bibname}}
  \renewcommand{\cftsecleader}{\cftdotfill{\cftdotsep}}

  % Custom settings
  \captionsetup{justification=centering}
  \PassOptionsToPackage{hyphens}{url}
  \urlstyle{same}
  \def\Urlmuskip{0mu}
  \def\UrlBreaks{\do\/\do-}
  \hypersetup{
    colorlinks = true,
    urlcolor = blue,
    linkcolor = black,
    citecolor = black,
  breaklinks=true,
  pdfpagemode=UseOutlines,
  bookmarksopen=true,
  bookmarksopenlevel=2,
  bookmarksnumbered=true
  }

  \title{\textbf{Flicking the Switch: } \\ How Optogenetics Sheds Light \\ on Motor Control Mechanisms}
  \author[ ]{Daniel Burger}
  \affil[ ]{\textbf{King’s College London}}
  \affil[ ]{\href{mailto:public@danielburger.online}{public@danielburger.online}}
  \date{\textit{8. August 2023}}

\begin{document}
\pagenumbering{roman}
\counterwithin{lstlisting}{section}
\counterwithin{figure}{section}
\counterwithin{table}{section}

% \maketitle
% \thispagestyle{empty}

\begin{sloppypar} % For better line breaks
  % \begin{abstract}
  %   Abstract coming soon.
  % \end{abstract}
  % \pagebreak

  % \pagenumbering{Roman}
  % \tableofcontents
  % \pagebreak

  % \listoffigures
  % \pagebreak

  % \listoftables
  % \pagebreak


  % Double spacing for feedback
  \doublespacing

  \pagenumbering{arabic}
  \section{Introduction} \label{sec:introduction}
  Parkinson’s disease, a neurodegenerative disorder, is primarily characterised by motor symptoms such as tremors, rigidity, and bradykinesia. These symptoms are primarily attributed to the degeneration of dopaminergic neurons in the basal ganglia (BG), a subcortical nuclei crucial for motor control and decision-making \citep {kravitz_regulation_2010}. Traditional models of the basal ganglia propose two opposing pathways: the direct pathway, which facilitates movement, and the indirect pathway, which inhibits movement. However, recent anatomical and functional evidence has challenged this classical model, suggesting a more complex interplay between these pathways \citep {dunovan_believer-skeptic_2016}.

  \cite {kravitz_regulation_2010} used optogenetic techniques to explore the roles of these pathways in animal models. Their findings support the classical model, demonstrating that activation of the direct pathway facilitates movement, while activation of the indirect pathway inhibits it. However, they also highlight the complexity of these pathways, suggesting that they function bidirectionally to regulate motor behaviour.

  On the other hand, \cite {dunovan_believer-skeptic_2016} propose a “Believer-Skeptic” model, arguing that the direct and indirect pathways dynamically compete during action selection, integrating contextual uncertainty with accumulating evidence. This model suggests that the basal ganglia play a crucial role in motor function, decision-making, and reinforcement learning.

  Building upon these findings, this essay will synthesise the current knowledge and debates surrounding the role of basal ganglia pathways in motor control and decision-making. We will explore how dynamic competition between the direct and indirect pathways modulates motor behaviour and decision-making. Furthermore, we will explore how alterations in these pathways contribute to the motor symptoms of Parkinson’s disease and discuss potential therapeutic strategies for ameliorating these symptoms.

  \pagebreak
  \singlespacing % No need for double spacing in the references
  \bibliographystyle{references/custom-apa}
  \bibliography{references/bibliography}

\end{sloppypar}
\end{document}