% Global document settings
\documentclass[10pt]{article}

% Packages
\usepackage{tgtermes}
\usepackage{graphicx}
\usepackage{natbib}
\usepackage{authblk}
\usepackage{array}
\usepackage{colortbl}
\usepackage{tocloft}
\usepackage{xcolor}
\usepackage{siunitx}
\usepackage{setspace}
\usepackage{listings}
\usepackage{caption}
\usepackage[T1]{fontenc}
\usepackage[nottoc]{tocbibind}
\usepackage[breaklinks]{hyperref}
\usepackage[font=small,skip=7pt]{caption}

% Custom colours
\definecolor{codegreen}{rgb}{0,0.6,0}
\definecolor{codegray}{rgb}{0.5,0.5,0.5}
\definecolor{codepurple}{rgb}{0.58,0,0.82}
\definecolor{backcolour}{rgb}{0.95,0.95,0.92}

% Listing styles
\lstdefinestyle{mystyle}{
  backgroundcolor=\color{backcolour},
  commentstyle=\color{codegreen},
  keywordstyle=\color{purple},
  numberstyle=\tiny\color{codegray},
  stringstyle=\color{codepurple},
  basicstyle=\ttfamily\footnotesize,
  breakatwhitespace=false,
  breaklines=true,
  captionpos=b,
  keepspaces=true,
  numbers=left,
  numbersep=5pt,
  showspaces=false,
  showstringspaces=true,
  showtabs=false,
  tabsize=2
  }
  \lstset{style=mystyle}

  % Custom commands
  \renewcommand\cftsecafterpnum{\vskip8pt}
  \renewcommand{\lstlistlistingname}{List of \lstlistingname s}
  \renewcommand{\bibsection}{\section*{Bibliography}}
  \renewcommand{\contentsname}{Table of Contents}
  \renewcommand{\bibsection}{\section{\bibname}}
  \renewcommand{\cftsecleader}{\cftdotfill{\cftdotsep}}

  % Custom settings
  \captionsetup{justification=centering}
  \PassOptionsToPackage{hyphens}{url}
  \urlstyle{same}
  \def\Urlmuskip{0mu}
  \def\UrlBreaks{\do\/\do-}
  \hypersetup{
    colorlinks = true,
    urlcolor = blue,
    linkcolor = black,
    citecolor = black,
  breaklinks=true,
  pdfpagemode=UseOutlines,
  bookmarksopen=true,
  bookmarksopenlevel=2,
  bookmarksnumbered=true
  }

  \title{\textbf{Evaluating the Quality of Life Concept:} \\ A Critical Examination of its Application and Value in Understanding Mental Health Burdens}
  \author[ ]{King’s College London}
  % \affil[ ]{\textbf{King’s College London}}
  % \affil[ ]{\href{mailto:public@danielburger.online}{public@danielburger.online}}
  \date{\textit{13. June 2023}}

\begin{document}
\pagenumbering{roman}
\counterwithin{lstlisting}{section}
\counterwithin{figure}{section}
\counterwithin{table}{section}

\maketitle
\thispagestyle{empty}

\begin{sloppypar} % For better line breaks
  \begin{abstract}
    Exploring the link between attention and conscious awareness in cognitive neuroscience has sparked numerous debates. This essay seeks to weigh the evidence supporting the idea that attention is a necessary component of conscious awareness. Drawing on empirical studies and additional philosophical perspectives, it delves into the entwined nature of these cognitive processes and considers opposing viewpoints. Additionally, the essay incorporates related concepts, such as Libet’s delay and the Global Workspace Theory, to provide a more comprehensive understanding of this complex relationship.

    By scrutinising these subjects, this essay aspires to enrich the reader’s comprehension of the interplay between attention and conscious awareness. It synthesises key insights in research, delivering a cohesive and up-to-date overview of prevailing findings.
  \end{abstract}
  \pagebreak

  \pagenumbering{Roman}
  \tableofcontents
  \pagebreak

  \listoffigures
  \pagebreak

  \listoftables
  \pagebreak


  % Double spacing for feedback
  \doublespacing

  \pagenumbering{arabic}
  \section{Introduction}
  \label{sec:introduction}

  The understanding of mental health burdens is profoundly nuanced, demanding a multidimensional lens of analysis that extends beyond mere clinical symptoms. Central to this expansive view is the concept of Quality of Life (QoL), a construct that has become pivotal in health research and clinical practice.

  While broadly used, the concept of Quality of Life is complex due to its inherently subjective nature. According to the World Health Organization, QoL is one's understanding of their place in society, shaped by the cultural and moral beliefs of their environment, as well as their personal aspirations, anticipations, principles, and worries \citep{harper_development_1998}. It is a broad-ranging concept, incorporating in a complex way the individual's physical health, psychological state, level of independence, social relationships, personal beliefs and their relationship to salient features of the environment. The domains of QoL, thus, encapsulate not just health but also personal, social, and environmental aspects.

  Applying QoL as a construct extends to various disciplines, including public health, medical practice, psychology, social work, and psychiatry. For instance, in public health, QoL measures can assist in evaluating the health of a population, prioritising health needs, or assessing the effectiveness of health promotion interventions. The QoL concept has been gaining momentum in psychiatry and mental health as an essential outcome measure. Instead of focusing solely on symptomatic relief, mental health professionals increasingly recognise the importance of a more holistic understanding of patients' well-being, including subjective perceptions of satisfaction and functioning.

  The application of QoL in psychiatry can be seen in several areas. In clinical trials, for instance, QoL metrics offer a patient-centred measure that complements traditional outcome markers such as symptom severity or remission rates \citep{malla_first-episode_2005}. Furthermore, QoL assessments can help tailor treatment plans to the individual patient's needs and goals in clinical practice, contributing to shared decision-making processes. Beyond the individual level, QoL measures provide valuable insights into population-based mental health research and policy-making, identifying unmet needs, evaluating intervention effectiveness, and guiding resource allocation.

  In the field of applied neuroscience, QoL is a critical construct to bridge our understanding of the biological underpinnings of mental disorders and the subjective experiences of those affected. While ongoing research investigates how specific neural mechanisms may directly impact QoL, studies have demonstrated that neurological changes associated with mental disorders can significantly affect an individual's perceived life satisfaction. For instance, research has found grey matter reductions in several brain regions in patients experiencing their first episode of depression, highlighting the potential neurological implications on the patient's overall well-being and life satisfaction \citep{zhang_brain_2016}.

  While it is evident that the QoL concept has vast applicability in understanding mental health burdens, it is not without criticisms and limitations. The following chapters will delve into the advantages and drawbacks of the QoL concept, critically examining its role in shaping our understanding of mental health.

  \section{Benefits of the Quality of Life Concept}
  \label{sec:benefits}

  The concept of Quality of Life (QoL) serves as a potent tool in mental health research and practice, offering benefits that address some of the limitations inherent in traditional clinical measures. This chapter explores these advantages, critically reviewing the empirical literature to provide a nuanced perspective.

  A. More Holistic Understanding of Patient Well-being
  One of the most significant benefits of the QoL concept is its broader, more comprehensive perspective on patient well-being. Traditionally, the assessment of mental health disorders relied heavily on symptomatic evaluations. However, the subjective experience of mental health is far more complex and extends beyond the presence or absence of symptoms. QoL assessments capture a range of factors that influence a person's well-being, including physical health, psychological status, social relationships, and environment.

  Studies suggest that these factors can often deviate from clinical symptoms, revealing unique aspects of a patient's experience that may otherwise remain unnoticed. For example, a study on patients with schizophrenia found that while clinical symptom severity decreased over time, many patients reported no improvement in their QoL \citep{eack_quality_2007}. These findings underscore the value of QoL measures in painting a more comprehensive picture of patient well-being.

  B. Enhanced Patient-Centered Care
  The QoL concept also supports a patient-centred approach in mental health care, aligning treatment goals with patients' unique needs and life contexts. Patient-centred care emphasises active patient participation in the treatment process and decision-making, an approach associated with improved treatment adherence and patient satisfaction \citep{dwamena_interventions_2012}.

  QoL measures can guide the delivery of personalised interventions by highlighting areas that matter most to the patient. For instance, the Quality of Life Enjoyment and Satisfaction Questionnaire (Q-LES-Q) has been used to obtain sensitive measures of enjoyment and satisfaction in various areas of daily functioning. This tool captures the patient's experience beyond the severity of illness or depression, potentially informing more personalised therapeutic approaches \citep{endicott_quality_1993}. This underscores the utility of QoL measures in tailoring interventions that resonate with patients' life contexts and goals.

  C. Insight into Population Health and Informing Health Policy
  On a broader scale, QoL metrics provide a robust understanding of population health and guiding health policies. By incorporating QoL indicators, mental health research can reveal disparities, track changes over time, and assess the impact of interventions or policy changes at a population level.

  For example, the Global Burden of Disease Study 2017 provides a systematic analysis of the incidence, prevalence, and years lived with disability for numerous diseases and injuries worldwide, including mental disorders \citep{gbd_2017_disease_and_injury_incidence_and_prevalence_collaborators_global_2018}. This comprehensive report contributes valuable data to inform global mental health policies by accounting for these factors. This example underscores the potential of QoL metrics to influence policy-making, highlighting the broader societal value of the QoL concept.

  \pagebreak
  \singlespacing % No need for double spacing in the references
  \bibliographystyle{references/custom-apa}
  \bibliography{references/bibliography}

\end{sloppypar}
\end{document}