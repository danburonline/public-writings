% Global document settings
\documentclass[10pt]{article}

% Packages
\usepackage{tgtermes}
\usepackage{graphicx}
\usepackage{natbib}
\usepackage{authblk}
\usepackage{array}
\usepackage{colortbl}
\usepackage{tocloft}
\usepackage{xcolor}
\usepackage{siunitx}
\usepackage{setspace}
\usepackage{listings}
\usepackage{caption}
\usepackage[T1]{fontenc}
\usepackage[nottoc]{tocbibind}
\usepackage[breaklinks]{hyperref}
\usepackage[font=small,skip=7pt]{caption}

% Custom colours
\definecolor{codegreen}{rgb}{0,0.6,0}
\definecolor{codegray}{rgb}{0.5,0.5,0.5}
\definecolor{codepurple}{rgb}{0.58,0,0.82}
\definecolor{backcolour}{rgb}{0.95,0.95,0.92}

% Listing styles
\lstdefinestyle{mystyle}{
  backgroundcolor=\color{backcolour},
  commentstyle=\color{codegreen},
  keywordstyle=\color{purple},
  numberstyle=\tiny\color{codegray},
  stringstyle=\color{codepurple},
  basicstyle=\ttfamily\footnotesize,
  breakatwhitespace=false,
  breaklines=true,
  captionpos=b,
  keepspaces=true,
  numbers=left,
  numbersep=5pt,
  showspaces=false,
  showstringspaces=true,
  showtabs=false,
  tabsize=2
  }
  \lstset{style=mystyle}

  % Custom commands
  \renewcommand\cftsecafterpnum{\vskip8pt}
  \renewcommand{\lstlistlistingname}{List of \lstlistingname s}
  \renewcommand{\bibsection}{\section*{Bibliography}}
  \renewcommand{\contentsname}{Table of Contents}
  \renewcommand{\bibsection}{\section{\bibname}}
  \renewcommand{\cftsecleader}{\cftdotfill{\cftdotsep}}

  % Custom settings
  \captionsetup{justification=centering}
  \PassOptionsToPackage{hyphens}{url}
  \urlstyle{same}
  \def\Urlmuskip{0mu}
  \def\UrlBreaks{\do\/\do-}
  \hypersetup{
    colorlinks = true,
    urlcolor = blue,
    linkcolor = black,
    citecolor = black,
  breaklinks=true,
  pdfpagemode=UseOutlines,
  bookmarksopen=true,
  bookmarksopenlevel=2,
  bookmarksnumbered=true
  }

  \title{\textbf{Attention as a Gateway to Consciousness:} \\ Evaluating the Evidence}
  \author[ ]{Daniel Burger}
  \affil[ ]{\textbf{King’s College London}}
  \affil[ ]{\href{mailto:public@danielburger.online}{public@danielburger.online}}
  \date{\textit{11. April 2023}}

\begin{document}
\pagenumbering{roman}
\counterwithin{lstlisting}{section}
\counterwithin{figure}{section}
\counterwithin{table}{section}

\maketitle
\thispagestyle{empty}

\begin{sloppypar} % For better line breaks
  \begin{abstract}
    In the realm of cognitive neuroscience, the relationship between attention and conscious awareness has been a subject of intense debate. This essay critically evaluates the evidence supporting the notion that attention is necessary for conscious awareness. Drawing from a range of empirical studies and theoretical perspectives, the essay explores the interdependence between these cognitive processes and considers alternative viewpoints on the nature of consciousness. Philosophical ideas are integrated, such as the concept of multiple streams of consciousness and Libet’s delay, to enrich the discussion and provide a more comprehensive understanding of the attention-consciousness relationship. By examining these complex topics, the essay sheds light on the intricate dynamics of attention and conscious awareness in the context of applied neuroscience.

  \end{abstract}
  \pagebreak

  \pagenumbering{Roman}
  \tableofcontents
  \pagebreak

  \listoffigures
  \pagebreak

  \listoftables
  \pagebreak


  % Double spacing for feedback
  \doublespacing

  \pagenumbering{arabic}
  \section{Introduction}
  \label{sec:introduction}

  The intricate relationship between attention and conscious awareness has long been a subject of intense debate and inquiry in the field of cognitive neuroscience. Attention, the cognitive process of selectively focusing on specific information while ignoring other stimuli, plays a crucial role in our ability to navigate and make sense of the complex world around us. Conscious awareness, on the other hand, refers to the subjective experience of perceiving and reflecting upon our thoughts, feelings, and sensory experiences. The central question that arises in the exploration of these cognitive processes is whether attention is necessary for conscious awareness. In other words, can we experience conscious awareness without directing our attention towards specific stimuli or mental content?

  This essay aims to critically evaluate the evidence supporting the statement that attention is necessary for conscious awareness. Drawing from a range of empirical studies and theoretical perspectives, we will examine the interdependence between attention and conscious awareness, exploring the various ways in which these cognitive processes may be linked. We will also consider alternative viewpoints that challenge the necessity of attention for conscious awareness, as well as integrate philosophical ideas, such as the concept of multiple streams of consciousness and Libet’s delay, to provide a more comprehensive understanding of the attention-consciousness relationship.

  In order to maintain a clear focus and adhere to the word limit, we will not exhaustively review every piece of literature on the topic, but rather carefully select key studies and theories that illuminate the complex dynamics between attention and conscious awareness. Through this critical examination, we aspire to contribute to the ongoing debate in the field of applied neuroscience and provide valuable insights for those seeking a deeper understanding of the attention-consciousness relationship.

  As we embark on this journey to dissect the intricate interplay between attention and conscious awareness, we invite the reader to reflect upon their own experiences, engage with the evidence, and contemplate the fascinating nature of human cognition.

  \section{Background and Definitions}
  \label{sec:background}

  \subsection{Consciousness}
  \label{sec:consciousness}

  Consciousness is a multifaceted phenomenon that encompasses our subjective experiences, thoughts, emotions, and perceptions. Various forms of consciousness include phenomenal consciousness, which relates to the qualitative aspects of experience, and access consciousness, which refers to the cognitive accessibility of information \citep{baars_essential_1997,montemayor_types_2021}. Understanding the diverse forms of consciousness is crucial for delving into the relationship between attention and conscious awareness.

  The role of consciousness in cognitive processes is complex and multifaceted. It serves as the foundation for our understanding of the world and ourselves, allowing us to process information, make decisions, and engage in goal-directed behavior \citep{dijksterhuis_goals_2010}. Furthermore, the concept of multiple streams of consciousness posits that consciousness is not a single, unified experience but rather comprises various parallel processes that can be influenced by attention \citep{montemayor_types_2021}. This perspective challenges traditional views of consciousness and encourages a more nuanced understanding of the attention-consciousness relationship.

  \subsection{Attention}
  \label{sec:attention}

  Attention, a core cognitive process, enables us to selectively focus on specific information while filtering out irrelevant stimuli. It plays a vital role in our ability to navigate the complexities of our environment, guiding our thoughts, perceptions, and actions \citep{cohen_attentional_2012}. Attention can be classified into different types, such as selective attention, which involves focusing on a single stimulus, and divided attention, which entails simultaneously attending to multiple stimuli \citep{koivisto_relationship_2009}.

  Understanding the various types of attention is essential for exploring their potential impact on conscious awareness. For instance, selective attention might play a different role in conscious awareness than divided attention, leading to distinct cognitive experiences \citep{kentridge_attended_2008}. By differentiating between these types of attention, we can better comprehend the nuances of the attention-consciousness relationship.

  \subsection{Libet’s delay}
  \label{sec:libet}

  Libet’s delay is a concept that refers to the time lag between the neural events underlying a conscious decision and the subjective experience of making that decision \citep{libet_time_1983}. This delay, typically on the order of several hundred milliseconds, has significant implications for understanding the nature of consciousness and its relationship to attention.

  The existence of Libet’s delay suggests that our subjective experience of consciousness might not always align with the actual neural processes occurring in our brains. It raises questions about the role of attention in shaping our conscious experiences and introduces an element of temporal complexity to the attention-consciousness relationship \citep{noah_recent_2020}. By considering the implications of Libet’s delay, we can gain a more comprehensive understanding of the dynamic interplay between attention and conscious awareness.

  \pagebreak
  \singlespacing % No need for double spacing in the references
  \bibliographystyle{references/custom-apa}
  \bibliography{references/bibliography}

\end{sloppypar}
\end{document}