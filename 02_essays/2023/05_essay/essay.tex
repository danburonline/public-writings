% Global document settings
\documentclass[10pt]{article}

% Packages
\usepackage{tgtermes}
\usepackage{graphicx}
\usepackage{natbib}
\usepackage{authblk}
\usepackage{array}
\usepackage{colortbl}
\usepackage{tocloft}
\usepackage{xcolor}
\usepackage{siunitx}
\usepackage{setspace}
\usepackage{listings}
\usepackage{caption}
\usepackage[T1]{fontenc}
\usepackage[nottoc]{tocbibind}
\usepackage[breaklinks]{hyperref}
\usepackage[font=small,skip=7pt]{caption}

% Custom colours
\definecolor{codegreen}{rgb}{0,0.6,0}
\definecolor{codegray}{rgb}{0.5,0.5,0.5}
\definecolor{codepurple}{rgb}{0.58,0,0.82}
\definecolor{backcolour}{rgb}{0.95,0.95,0.92}

% Listing styles
\lstdefinestyle{mystyle}{
  backgroundcolor=\color{backcolour},
  commentstyle=\color{codegreen},
  keywordstyle=\color{purple},
  numberstyle=\tiny\color{codegray},
  stringstyle=\color{codepurple},
  basicstyle=\ttfamily\footnotesize,
  breakatwhitespace=false,
  breaklines=true,
  captionpos=b,
  keepspaces=true,
  numbers=left,
  numbersep=5pt,
  showspaces=false,
  showstringspaces=true,
  showtabs=false,
  tabsize=2
  }
  \lstset{style=mystyle}

  % Custom commands
  \renewcommand\cftsecafterpnum{\vskip8pt}
  \renewcommand{\lstlistlistingname}{List of \lstlistingname s}
  \renewcommand{\bibsection}{\section*{Bibliography}}
  \renewcommand{\contentsname}{Table of Contents}
  \renewcommand{\bibsection}{\section{\bibname}}
  \renewcommand{\cftsecleader}{\cftdotfill{\cftdotsep}}

  % Custom settings
  \captionsetup{justification=centering}
  \PassOptionsToPackage{hyphens}{url}
  \urlstyle{same}
  \def\Urlmuskip{0mu}
  \def\UrlBreaks{\do\/\do-}
  \hypersetup{
    colorlinks = true,
    urlcolor = blue,
    linkcolor = black,
    citecolor = black,
  breaklinks=true,
  pdfpagemode=UseOutlines,
  bookmarksopen=true,
  bookmarksopenlevel=2,
  bookmarksnumbered=true
  }

  \title{\textbf{Attention as a Gateway to Consciousness:} \\ Evaluating the Evidence}
  \author[ ]{Daniel Burger}
  \affil[ ]{\textbf{King’s College London}}
  \affil[ ]{\href{mailto:public@danielburger.online}{public@danielburger.online}}
  \date{\textit{11. April 2023}}

\begin{document}
\pagenumbering{roman}
\counterwithin{lstlisting}{section}
\counterwithin{figure}{section}
\counterwithin{table}{section}

\maketitle
\thispagestyle{empty}

\begin{sloppypar} % For better line breaks
  \begin{abstract}
    In the realm of cognitive neuroscience, the relationship between attention and conscious awareness has been a subject of intense debate. This essay critically evaluates the evidence supporting the notion that attention is necessary for conscious awareness. Drawing from a range of empirical studies and theoretical perspectives, the essay explores the interdependence between these cognitive processes and considers alternative viewpoints on the nature of consciousness. Philosophical ideas are integrated, such as the concept of multiple streams of consciousness and Libet’s delay, to enrich the discussion and provide a more comprehensive understanding of the attention-consciousness relationship. By examining these complex topics, the essay sheds light on the intricate dynamics of attention and conscious awareness in the context of applied neuroscience.

  \end{abstract}
  \pagebreak

  \pagenumbering{Roman}
  \tableofcontents
  \pagebreak

  \listoffigures
  \pagebreak

  \listoftables
  \pagebreak


  % Double spacing for feedback
  \doublespacing

  \pagenumbering{arabic}
  \section{Introduction}
  \label{sec:introduction}

  The intricate relationship between attention and conscious awareness has long been a subject of intense debate and inquiry in the field of cognitive neuroscience. Attention, the cognitive process of selectively focusing on specific information while ignoring other stimuli, plays a crucial role in our ability to navigate and make sense of the complex world around us. Conscious awareness, on the other hand, refers to the subjective experience of perceiving and reflecting upon our thoughts, feelings, and sensory experiences. The central question that arises in the exploration of these cognitive processes is whether attention is necessary for conscious awareness. In other words, can we experience conscious awareness without directing our attention towards specific stimuli or mental content?

  This essay aims to critically evaluate the evidence supporting the statement that attention is necessary for conscious awareness. Drawing from a range of empirical studies and theoretical perspectives, we will examine the interdependence between attention and conscious awareness, exploring the various ways in which these cognitive processes may be linked. We will also consider alternative viewpoints that challenge the necessity of attention for conscious awareness, as well as integrate philosophical ideas, such as the concept of multiple streams of consciousness and Libet’s delay, to provide a more comprehensive understanding of the attention-consciousness relationship.

  In order to maintain a clear focus and adhere to the word limit, we will not exhaustively review every piece of literature on the topic, but rather carefully select key studies and theories that illuminate the complex dynamics between attention and conscious awareness. Through this critical examination, we aspire to contribute to the ongoing debate in the field of applied neuroscience and provide valuable insights for those seeking a deeper understanding of the attention-consciousness relationship.

  As we embark on this journey to dissect the intricate interplay between attention and conscious awareness, we invite the reader to reflect upon their own experiences, engage with the evidence, and contemplate the fascinating nature of human cognition.

  \pagebreak
  \singlespacing % No need for double spacing in the references
  \bibliographystyle{references/custom-apa}
  \bibliography{references/bibliography}

\end{sloppypar}
\end{document}