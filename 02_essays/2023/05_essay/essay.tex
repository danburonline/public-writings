% Global document settings
\documentclass[10pt]{article}

% Packages
\usepackage{tgtermes}
\usepackage{graphicx}
\usepackage{natbib}
\usepackage{authblk}
\usepackage{array}
\usepackage{colortbl}
\usepackage{tocloft}
\usepackage{xcolor}
\usepackage{siunitx}
\usepackage{setspace}
\usepackage{listings}
\usepackage{caption}
\usepackage[T1]{fontenc}
\usepackage[nottoc]{tocbibind}
\usepackage[breaklinks]{hyperref}
\usepackage[font=small,skip=7pt]{caption}

% Custom colours
\definecolor{codegreen}{rgb}{0,0.6,0}
\definecolor{codegray}{rgb}{0.5,0.5,0.5}
\definecolor{codepurple}{rgb}{0.58,0,0.82}
\definecolor{backcolour}{rgb}{0.95,0.95,0.92}

% Listing styles
\lstdefinestyle{mystyle}{
  backgroundcolor=\color{backcolour},
  commentstyle=\color{codegreen},
  keywordstyle=\color{purple},
  numberstyle=\tiny\color{codegray},
  stringstyle=\color{codepurple},
  basicstyle=\ttfamily\footnotesize,
  breakatwhitespace=false,
  breaklines=true,
  captionpos=b,
  keepspaces=true,
  numbers=left,
  numbersep=5pt,
  showspaces=false,
  showstringspaces=true,
  showtabs=false,
  tabsize=2
  }
  \lstset{style=mystyle}

  % Custom commands
  \renewcommand\cftsecafterpnum{\vskip8pt}
  \renewcommand{\lstlistlistingname}{List of \lstlistingname s}
  \renewcommand{\bibsection}{\section*{Bibliography}}
  \renewcommand{\contentsname}{Table of Contents}
  \renewcommand{\bibsection}{\section{\bibname}}
  \renewcommand{\cftsecleader}{\cftdotfill{\cftdotsep}}

  % Custom settings
  \captionsetup{justification=centering}
  \PassOptionsToPackage{hyphens}{url}
  \urlstyle{same}
  \def\Urlmuskip{0mu}
  \def\UrlBreaks{\do\/\do-}
  \hypersetup{
    colorlinks = true,
    urlcolor = blue,
    linkcolor = black,
    citecolor = black,
  breaklinks=true,
  pdfpagemode=UseOutlines,
  bookmarksopen=true,
  bookmarksopenlevel=2,
  bookmarksnumbered=true
  }

  \title{\textbf{Attention as a Gateway to Consciousness:} \\ Evaluating the Evidence}
  \author[ ]{Daniel Burger}
  \affil[ ]{\textbf{King’s College London}}
  \affil[ ]{\href{mailto:public@danielburger.online}{public@danielburger.online}}
  \date{\textit{11. April 2023}}

\begin{document}
\pagenumbering{roman}
\counterwithin{lstlisting}{section}
\counterwithin{figure}{section}
\counterwithin{table}{section}

\maketitle
\thispagestyle{empty}

\begin{sloppypar} % For better line breaks
  \begin{abstract}
    In the realm of cognitive neuroscience, the relationship between attention and conscious awareness has been a subject of intense debate. This essay critically evaluates the evidence supporting the notion that attention is necessary for conscious awareness. Drawing from a range of empirical studies and theoretical perspectives, the essay explores the interdependence between these cognitive processes and considers alternative viewpoints on the nature of consciousness. Philosophical ideas are integrated, such as the concept of multiple streams of consciousness and Libet’s delay, to enrich the discussion and provide a more comprehensive understanding of the attention-consciousness relationship. By examining these complex topics, the essay sheds light on the intricate dynamics of attention and conscious awareness in the context of applied neuroscience.

  \end{abstract}
  \pagebreak

  \pagenumbering{Roman}
  \tableofcontents
  \pagebreak

  \listoffigures
  \pagebreak

  \listoftables
  \pagebreak


  % Double spacing for feedback
  \doublespacing

  \pagenumbering{arabic}
  \section{Introduction}
  \label{sec:introduction}

  The intricate relationship between attention and conscious awareness has long been a subject of intense debate and inquiry in the field of cognitive neuroscience. Attention, the cognitive process of selectively focusing on specific information while ignoring other stimuli, plays a crucial role in our ability to navigate and make sense of the complex world around us. Conscious awareness, on the other hand, refers to the subjective experience of perceiving and reflecting upon our thoughts, feelings, and sensory experiences. The central question that arises in the exploration of these cognitive processes is whether attention is necessary for conscious awareness. In other words, can we experience conscious awareness without directing our attention towards specific stimuli or mental content?

  This essay aims to critically evaluate the evidence supporting the statement that attention is necessary for conscious awareness. Drawing from a range of empirical studies and theoretical perspectives, we will examine the interdependence between attention and conscious awareness, exploring the various ways in which these cognitive processes may be linked. We will also consider alternative viewpoints that challenge the necessity of attention for conscious awareness, as well as integrate philosophical ideas, such as the concept of multiple streams of consciousness and Libet’s delay, to provide a more comprehensive understanding of the attention-consciousness relationship.

  In order to maintain a clear focus and adhere to the word limit, we will not exhaustively review every piece of literature on the topic, but rather carefully select key studies and theories that illuminate the complex dynamics between attention and conscious awareness. Through this critical examination, we aspire to contribute to the ongoing debate in the field of applied neuroscience and provide valuable insights for those seeking a deeper understanding of the attention-consciousness relationship.

  As we embark on this journey to dissect the intricate interplay between attention and conscious awareness, we invite the reader to reflect upon their own experiences, engage with the evidence, and contemplate the fascinating nature of human cognition.

  \section{Background and Definitions}
  \label{sec:background}

  \subsection{Consciousness}
  \label{sec:consciousness}

  Consciousness is a multifaceted phenomenon that encompasses our subjective experiences, thoughts, emotions, and perceptions. Various forms of consciousness include phenomenal consciousness, which relates to the qualitative aspects of experience, and access consciousness, which refers to the cognitive accessibility of information \citep{baars_essential_1997,montemayor_types_2021}. Understanding the diverse forms of consciousness is crucial for delving into the relationship between attention and conscious awareness.

  The role of consciousness in cognitive processes is complex and multifaceted. It serves as the foundation for our understanding of the world and ourselves, allowing us to process information, make decisions, and engage in goal-directed behavior \citep{dijksterhuis_goals_2010}. Furthermore, the concept of multiple streams of consciousness posits that consciousness is not a single, unified experience but rather comprises various parallel processes that can be influenced by attention \citep{montemayor_types_2021}. This perspective challenges traditional views of consciousness and encourages a more nuanced understanding of the attention-consciousness relationship.

  \subsection{Attention}
  \label{sec:attention}

  Attention, a core cognitive process, enables us to selectively focus on specific information while filtering out irrelevant stimuli. It plays a vital role in our ability to navigate the complexities of our environment, guiding our thoughts, perceptions, and actions \citep{cohen_attentional_2012}. Attention can be classified into different types, such as selective attention, which involves focusing on a single stimulus, and divided attention, which entails simultaneously attending to multiple stimuli \citep{koivisto_relationship_2009}.

  Understanding the various types of attention is essential for exploring their potential impact on conscious awareness. For instance, selective attention might play a different role in conscious awareness than divided attention, leading to distinct cognitive experiences \citep{kentridge_attended_2008}. By differentiating between these types of attention, we can better comprehend the nuances of the attention-consciousness relationship.

  \subsection{Libet’s delay}
  \label{sec:libet}

  Libet’s delay is a concept that refers to the time lag between the neural events underlying a conscious decision and the subjective experience of making that decision \citep{libet_time_1983}. This delay, typically on the order of several hundred milliseconds, has significant implications for understanding the nature of consciousness and its relationship to attention.

  The existence of Libet’s delay suggests that our subjective experience of consciousness might not always align with the actual neural processes occurring in our brains. It raises questions about the role of attention in shaping our conscious experiences and introduces an element of temporal complexity to the attention-consciousness relationship \citep{noah_recent_2020}. By considering the implications of Libet’s delay, we can gain a more comprehensive understanding of the dynamic interplay between attention and conscious awareness.

  \section{Evidence supporting the necessity of attention for conscious awareness}
  \label{sec:evidence}

  \subsection{Empirical studies}
  \label{sec:empirical}

  Several empirical studies provide evidence for the link between attention and conscious awareness. One such study, conducted by \cite{cohen_attentional_2012}, investigated the attentional requirements of consciousness by manipulating the allocation of attention in a visual search task. The authors found that when attention was directed away from a target stimulus, participants were less likely to report conscious awareness of the stimulus, suggesting that attention plays a critical role in conscious perception.

  Similarly, \cite{kentridge_spatial_2004} explored the role of attention in blindsight, a neurological condition in which individuals with damage to the primary visual cortex can respond to visual stimuli without conscious awareness. In their study, the authors demonstrated that when spatial attention was directed towards a stimulus, participants with blindsight exhibited faster response times, despite a lack of conscious awareness. This finding supports the idea that attention can influence unconscious processing and potentially modulate conscious awareness.

  Another study by \cite{sumner_attentional_2006} investigated the role of attention in sensorimotor processes in the absence of perceptual awareness. The authors employed a visual masking paradigm to render stimuli imperceptible and found that attention could still modulate participants' motor responses to the masked stimuli. This result implies that attention can modulate cognitive processes even when conscious awareness is absent, further highlighting the intricate relationship between attention and conscious awareness.

  \subsection{Theoretical perspectives}
  \label{sec:theoretical}

  Various theoretical perspectives also support the notion that attention is necessary for conscious awareness. \cite{baars_essential_1997}'s Global Workspace Theory posits that consciousness arises when information becomes globally available within the brain, and attention plays a crucial role in selecting and broadcasting this information. According to this theory, attention acts as a gatekeeper that determines which information enters the global workspace and subsequently becomes part of our conscious experience.

  \cite{de_brigard_role_2012} proposed the Attentional Relevance Theory, which suggests that attention is necessary for the conscious recollection of past events. According to this theory, attention serves to enhance the encoding and retrieval of memories by prioritizing information that is relevant to our goals and interests. This perspective emphasizes the role of attention in shaping the content of our conscious experiences, particularly in the domain of memory.

  Finally, \cite{dijksterhuis_goals_2010} put forth the idea that attention plays a key role in goal-directed behavior, which is intimately linked to conscious awareness. They argue that attention serves to activate and maintain cognitive representations of goals, enabling us to consciously pursue and achieve desired outcomes. This perspective highlights the importance of attention in bridging the gap between our conscious intentions and actions, further reinforcing the necessity of attention for conscious awareness.

  \section{Alternative viewpoints and evidence}
  \label{sec:alternative}

  \subsection{Empirical studies}
  \label{sec:empirical_alt}
  While several studies support the necessity of attention for conscious awareness, others challenge this notion. \cite{aru_phenomenal_2013} investigated whether phenomenal awareness could emerge without attention, using a visual paradigm in which participants reported their conscious experience of stimuli under various attentional manipulations. The authors found evidence for conscious perception even when attention was directed away from the target stimulus, suggesting that attention may not be strictly necessary for conscious awareness.

  \cite{kentridge_attended_2008} also questioned the sufficiency of attention for visual awareness, examining the interplay between attention and awareness in a patient with visual form agnosia, a condition characterized by the inability to recognize objects despite preserved low-level vision. The authors found that the patient could allocate attention to a stimulus without reporting conscious awareness of its shape or orientation, indicating that attention may be necessary but not sufficient for visual awareness.

  Furthermore, \cite{kozuch_gorillas_2019} conducted a critical reevaluation of the evidence for attention being necessary for consciousness, challenging the conclusions of several well-known studies, including the influential work by \cite{cohen_attentional_2012}. Kozuch argued that many studies supporting the necessity of attention for consciousness were methodologically flawed or misinterpreted, suggesting that the attention-consciousness relationship is still open to debate.

  \subsection{Theoretical perspectives}
  \label{sec:theoretical_alt}

  Several theoretical perspectives and philosophical ideas propose alternative viewpoints on the attention-consciousness relationship. \cite{montemayor_types_2021} posited that consciousness encompasses multiple types, each with distinct neural correlates and functional roles. This perspective challenges the idea of a unified attention-consciousness relationship, suggesting that attention may differentially influence various types of consciousness.

  \cite{noah_recent_2020} presented a comprehensive review of recent evidence concerning the attention-consciousness relationship, concluding that while attention is necessary for conscious perception, it is not sufficient. They argued that additional factors, such as the interaction between top-down and bottom-up processes, contribute to conscious awareness. This viewpoint highlights the complexity of the attention-consciousness relationship and encourages further exploration of the underlying cognitive and neural mechanisms.

  These alternative viewpoints and empirical findings demonstrate that the relationship between attention and conscious awareness is far from settled, inviting further investigation and debate. By considering these alternative perspectives, we can better understand the nuanced and multifaceted nature of attention and consciousness, and appreciate the complexity of the human cognitive experience.

  \section{The role of attention in multiple streams of consciousness}
  \label{sec:streams}

  The concept of multiple streams of consciousness challenges the idea of a single, unified conscious experience. Instead, it proposes that consciousness consists of numerous, parallel streams that contribute to our subjective experience. This perspective has significant implications for understanding the attention-consciousness relationship and offers novel insights into the role of attention in shaping the experience of various streams of consciousness.

  One possibility is that attention acts as a selective mechanism that determines which streams of consciousness gain prominence at any given moment. By directing attention to specific aspects of our experience, we may be able to modulate the relative importance of different streams of consciousness, allowing us to prioritize and navigate the complex landscape of our subjective experience. In this view, attention serves as a flexible and adaptive tool for orchestrating the various streams of consciousness, enabling us to focus on what is most relevant or engaging.

  The role of attention in shaping the experience of multiple streams of consciousness can also be understood through the lens of cognitive control. Cognitive control refers to our ability to guide our thoughts and actions in line with our goals and intentions. By allocating attention to specific streams of consciousness, we can exercise cognitive control over our subjective experience, selectively enhancing or suppressing particular aspects of our conscious awareness. This view emphasizes the active and dynamic nature of attention in managing multiple streams of consciousness.

  Libet’s delay, the time lag between the onset of neural activity associated with a conscious decision and the subjective awareness of that decision, also bears relevance to the perception of multiple streams of consciousness. This delay suggests that our conscious experience may not be an instantaneous reflection of neural activity but rather a temporally extended and integrated representation of various streams of information. The role of attention in this context may involve the integration and consolidation of these temporally separated streams of consciousness, creating a coherent and unified subjective experience.

  In summary, the concept of multiple streams of consciousness offers a novel perspective on the attention-consciousness relationship, emphasizing the importance of attention in shaping, prioritizing, and integrating our subjective experience. By considering the role of attention in managing multiple streams of consciousness, we can gain a deeper understanding of the complex interplay between attention and conscious awareness, and ultimately, the nature of the human mind.

  \pagebreak
  \singlespacing % No need for double spacing in the references
  \bibliographystyle{references/custom-apa}
  \bibliography{references/bibliography}

\end{sloppypar}
\end{document}