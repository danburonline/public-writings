% Global document settings
\documentclass[10pt]{article}

% Packages
\usepackage{tgtermes}
\usepackage{graphicx}
\usepackage{natbib}
\usepackage{authblk}
\usepackage{array}
\usepackage{colortbl}
\usepackage{tocloft}
\usepackage{xcolor}
\usepackage{siunitx}
\usepackage{tocloft}
\usepackage{setspace}
\usepackage{listings}
\usepackage{caption}
\usepackage[T1]{fontenc}
\usepackage[nottoc]{tocbibind}
\usepackage[breaklinks]{hyperref}
\usepackage[font=small,skip=7pt]{caption}

% Custom colours
\definecolor{codegreen}{rgb}{0,0.6,0}
\definecolor{codegray}{rgb}{0.5,0.5,0.5}
\definecolor{codepurple}{rgb}{0.58,0,0.82}
\definecolor{backcolour}{rgb}{0.95,0.95,0.92}

% Listing styles
\lstdefinestyle{mystyle}{
  backgroundcolor=\color{backcolour},
  commentstyle=\color{codegreen},
  keywordstyle=\color{purple},
  numberstyle=\tiny\color{codegray},
  stringstyle=\color{codepurple},
  basicstyle=\ttfamily\footnotesize,
  breakatwhitespace=false,
  breaklines=true,
  captionpos=b,
  keepspaces=true,
  numbers=left,
  numbersep=5pt,
  showspaces=false,
  showstringspaces=true,
  showtabs=false,
  tabsize=2
  }
  \lstset{style=mystyle}

  % Custom commands
  \renewcommand\cftsecafterpnum{\vskip8pt}
  \renewcommand{\lstlistlistingname}{List of \lstlistingname s}
  \renewcommand{\bibsection}{\section*{Bibliography}}
  \renewcommand{\contentsname}{Table of Contents}
  \renewcommand{\bibsection}{\section{\bibname}}
  \renewcommand{\cftsecleader}{\cftdotfill{\cftdotsep}}

  % Custom settings
  \captionsetup{justification=centering}
  \PassOptionsToPackage{hyphens}{url}
  \urlstyle{same}
  \def\Urlmuskip{0mu}
  \def\UrlBreaks{\do\/\do-}
  \hypersetup{
    colorlinks = true,
    urlcolor = blue,
    linkcolor = black,
    citecolor = black,
  breaklinks=true,
  pdfpagemode=UseOutlines,
  bookmarksopen=true,
  bookmarksopenlevel=2,
  bookmarksnumbered=true
  }

  \title{\textbf{From In Vivo to In Silico:} \\ The Role of Animal Models in Advancing Our Understanding of Brain Diseases}
  \author[1]{Daniel Burger}
  \affil[1]{\textbf{King's College London}}
  \affil[ ]{\href{mailto:daniel.burger@kcl.ac.uk}{daniel.burger@kcl.ac.uk}}
  \date{\textit{14. February 2023}}

\begin{document}
\pagenumbering{roman}
\counterwithin{lstlisting}{section}
\counterwithin{figure}{section}
\counterwithin{table}{section}

\maketitle
\thispagestyle{empty}

\begin{sloppypar} % For better line breaks
  \begin{abstract}
    Animal models have played a critical role in advancing our understanding of brain diseases. In this essay, we discuss their advantages and limitations and examine two examples of successful advances in our understanding of brain diseases, including one case where they did not deliver the desired outcomes.

    We then look to the future of neuroscience research, including the potential of using cell cultures and computational models in conjunction with animal models. We conclude by emphasising the ongoing importance of animal models in advancing our understanding of brain diseases.

  \end{abstract}
  \pagebreak

  \pagenumbering{Roman}
  \tableofcontents
  \pagebreak

  \listoffigures
  \pagebreak

  \listoftables
  \pagebreak


  % Double spacing for feedback
  \doublespacing

  \pagenumbering{arabic}
  \section{Introduction}
  \label{sec:introduction}

  Animal models have been invaluable tools in biomedical research, particularly in the field of neuroscience. Through the use of animal models, it is possible to delve into the complexity of our brains and uncover ways that diseases manifest themselves. In vitro, in vivo, and in silico research are some of the methods utilised in neuroscience research, as shown in \autoref{tab:overview-research-methods}. However, in vivo animal models provide a unique advantage in allowing researchers to study the entire organism in a controlled environment, where many variables can be controlled.

  \vspace{10pt} % Increase vertical spacing before table
  \begin{table}[ht]
    \centering
    \renewcommand{\arraystretch}{1.5} % Increase vertical spacing
    \setlength{\tabcolsep}{12pt} % Increase horizontal spacing
    \resizebox{\linewidth}{!}{%
      \begin{tabular}{|>{\hspace{0pt}}m{0.18\linewidth}|>{\hspace{0pt}}m{0.78\linewidth}|} % adjust column widths
        \hline
        \rowcolor[rgb]{0.961,0.961,0.961} \textbf{Method} & \textbf{Definition}                                                                                                         \\
        \hline
        In vitro                                          & Research conducted using isolated biological components in a controlled environment, such as cell cultures or organoids.    \\
        \hline
        In vivo                                           & Research conducted using living organisms, often using animal models, to study the effects of a treatment or intervention.  \\
        \hline
        In silico                                         & Research conducted using computational models or simulations to predict the effects of interventions on biological systems. \\
        \hline
      \end{tabular}
    }
    \caption{Comparison of In Vitro, In Vivo, and In Silico Research Methods.}
    \label{tab:overview-research-methods}
  \end{table}

  The use of animal models is crucial in the study of diseases affecting the human brain. While we cannot use humans in every case due to ethical concerns, animal models allow us to study the progression of diseases in a way that is impossible with human subjects. In addition, researchers can use animal models to investigate disease mechanisms at the molecular, cellular, and systemic levels, which is difficult or impossible to do in humans since, for example, the modification of human genomes is not possible due to ethical and safety concerns.

  Animal models have significantly advanced our understanding of many neurological disorders, such as Alzheimer’s, Parkinson’s, and stroke. They allow us to gain insight into the underlying mechanisms of these diseases, which can then lead to new therapies or treatments. For example, studies utilising mouse models have helped researchers better understand the role of genetics in Alzheimer’s disease and Parkinson’s (CITATION). Additionally, animal models have been instrumental in providing insight into the effects of substances such as drugs or alcohol on the brain (CITATION).

  Research with animal models is commonly utilised to assess the safety and effectiveness of potential treatments and drug therapies for neurological diseases, thus reducing the chances of unfavourable outcomes in human trials. This process involves obtaining various types of value from animal models, such as face value, predictive value, and construct value, to assist with translating research findings to humans. Additionally, researchers can now manipulate neuronal networks in model animals using new technologies such as optogenetics and chemogenetics, providing a better understanding of disease pathology.

  However, it is important to note that animal models do have limitations. While they can provide valuable insight into disease mechanisms, they do not always translate perfectly to human disease. One example is the development of Alzheimer’s disease treatments, where many drugs have shown efficacy in animal models but have failed in human trials (CITATION). Moreover, there are ethical concerns surrounding the use of animals in research, which has led to the development of alternatives such as cell cultures and dish brains (CITATION).

  In conclusion, animal models have been instrumental in advancing our understanding and treatment of diseases, especially in the field of neuroscience. While they have limitations and ethical considerations, the ability to study the entire organism, generate genetically modified strains, and modify neuronal networks in controlled environments provides researchers with invaluable insight into disease mechanisms. Moreover, with the advent of new technologies and complementary methods, such as computational neuroscience and in vitro approaches, animal models can be combined with other techniques to help solve the mysteries of the human brain.

  \section{Benefits of Animal Models}
  \label{sec:benefits}

  \section{Limitations of Animal Models}
  \label{sec:limitations}

  \section{Combination of Different Research Methods}
  \label{sec:combination}

  \section{Conclusion}
  \label{sec:conclusion}

  \pagebreak
  \bibliographystyle{references/custom-apa}
  \bibliography{references/bibliography}

\end{sloppypar}
\end{document}