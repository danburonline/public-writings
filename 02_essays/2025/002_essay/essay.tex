\documentclass[10pt]{article}
% ! ========================
% ! # MARK: Template Control
% ! ========================

% Defines the template variant. Can be 'essay' (default) or 'paper'.
% This command must come before the document class definition in the main file.
\providecommand{\templatevariant}{essay}


% ! =======================
% ! # MARK: Package Loading
% ! =======================

% Common Packages
\usepackage[T1]{fontenc} % Font encoding
\usepackage{tgtermes} % TeX Gyre Termes font (Times clone)
\usepackage{geometry} % Page layout control
\usepackage{setspace} % Line spacing control
\usepackage{graphicx} % Enhanced graphics
\usepackage{xcolor} % Color definitions
\usepackage{colortbl} % Color in tables
\usepackage{hhline} % Double horizontal lines in tables
\usepackage{makecell} % Thicker lines in tables
\usepackage{tabularx, booktabs} % Advanced table layouts
\usepackage{enumitem} % Custom list environments
\usepackage{amsmath} % Advanced math environments
\usepackage{amssymb} % Math symbols
\usepackage{siunitx} % SI units package
\usepackage{listings} % Code listings
\usepackage{natbib} % Bibliography and citations
\usepackage{authblk} % Author and affiliation blocks
\usepackage{tocloft} % ToC, LoF, LoT styling
\usepackage[nottoc]{tocbibind} % Add Bib, Index, etc., to ToC
\usepackage{microtype} % Improved typography (justification, spacing)
\usepackage{url} % URL formatting
\usepackage[breaklinks,linktocpage]{hyperref} % Hyperlinks (must be loaded late)

% Variant-Specific Packages
% Define variant names for comparison
\def\papervariant{paper}
\def\essayvariant{essay}

% Load packages only required for the 'paper' template
\ifx\templatevariant\papervariant
  \usepackage{multicol}     % For multi-column layout
  \usepackage{dblfloatfix}  % Fixes for double-column floats
\fi


% ! ==============================
% ! # MARK: Page Layout & Geometry
% ! ==============================

\ifx\templatevariant\papervariant
  % Paper Template Layout (Two-Column)
  \geometry{
    left=3cm,
    right=3cm,
    top=2.5cm,
    bottom=3cm
  }
  % Settings for two-column layout
  \setlength{\columnseprule}{0pt} % No vertical rule between columns
  \setlength{\columnsep}{0.7cm}   % Space between columns
\else
  % Essay Template Layout (Single-Column, Default)
  \geometry{
    left=4cm,
    right=4cm,
    top=3cm,
    bottom=3.5cm
  }
\fi


% ! ==========================
% ! # MARK: Typography & Fonts
% ! ==========================

% Page Numbering
\renewcommand{\thepage}{\footnotesize\arabic{page}}

% List Spacing
\setlist[itemize]{noitemsep, topsep=7pt, partopsep=0pt, leftmargin=*, label=\textendash}


% ! ================================
% ! # MARK: Floats, Figures & Tables
% ! ================================

% Float Placement Parameters
\renewcommand{\topfraction}{0.9}
\renewcommand{\bottomfraction}{0.8}
\renewcommand{\textfraction}{0.1}
\renewcommand{\floatpagefraction}{0.8}
% Additional float settings for two-column paper template
\ifx\templatevariant\papervariant
  \renewcommand{\dbltopfraction}{0.9}
  \renewcommand{\dblfloatpagefraction}{0.8}
\fi

% Caption Styling
\usepackage[font=footnotesize,skip=7pt,labelfont=bf]{caption}
\captionsetup{justification=raggedright} % Left-align all captions
\newcommand{\floatcaption}[2]{\caption[#1.]{#1~#2.}} % Custom caption command


% ! ================================
% ! # MARK: Bibliography & Citations
% ! ================================

\renewcommand{\bibname}{References} % Change bibliography title
\setlength\bibindent{0pt}           % No indentation for bibliography entries

% Adjust layout and font size in the bibliography environment
\let\oldthebibliography=\thebibliography
\let\endoldthebibliography=\endthebibliography
\renewenvironment{thebibliography}[1]{%
  \begin{oldthebibliography}{#1}%
    \raggedright%
    \footnotesize%
    \setlength{\itemsep}{3pt}%
    \setlength{\parsep}{0pt}%
    \setlength{\parskip}{0pt}%
    }{%
  \end{oldthebibliography}%
}


% ! =====================
% ! # MARK: Code Listings
% ! =====================

% Custom Colors for Code
\definecolor{codegreen}{rgb}{0,0.5,0}
\definecolor{codegray}{rgb}{0.4,0.4,0.4}
\definecolor{codepurple}{rgb}{0.58,0,0.82}
\definecolor{backcolour}{rgb}{0.96,0.96,0.96}
\definecolor{lightgray}{gray}{0.8}

% Language Definition (Example: JavaScript)
\lstdefinelanguage{JavaScript}{
  keywords={break, case, catch, continue, debugger, default, delete, do, else, finally, for, function, if, in, instanceof, new, return, switch, this, throw, try, typeof, var, void, while, with},
  morecomment=[l]{//},
  morecomment=[s]{/*}{*/},
  morestring=[b]',
  morestring=[b]",
  sensitive=true
}

% Listing Style Definition
\lstdefinestyle{mystyle}{
  backgroundcolor=\color{backcolour},
  commentstyle=\color{codegreen},
  keywordstyle=\color{purple},
  numberstyle=\tiny\color{codegray},
  stringstyle=\color{codepurple},
  basicstyle=\ttfamily\footnotesize,
  breakatwhitespace=false,
  breaklines=true,
  captionpos=b,
  frame=tb,
  framerule=0pt,
  framextopmargin=6pt,
  framexbottommargin=6pt,
  keepspaces=true,
  numbers=left,
  numbersep=5pt,
  showspaces=false,
  showstringspaces=false,
  showtabs=false,
  tabsize=2
}
\lstset{style=mystyle} % Apply the defined style globally


% ! ===================================
% ! # MARK: Document Structure & Titles
% ! ===================================

% Table of Contents (ToC)
\renewcommand{\contentsname}{Table of Contents}
\renewcommand\cftsecafterpnum{\vskip8pt}              % Vertical space after section numbers
\renewcommand{\cftsecleader}{\cftdotfill{\cftdotsep}} % Dotted leaders for sections

% List of Listings (LoL)
\renewcommand{\lstlistlistingname}{List of \lstlistingname s}


% ! ==========================
% ! # MARK: Hyperlinks & URLs
% ! ==========================

\hypersetup{
  colorlinks = true,
  urlcolor   = blue,
  linkcolor  = blue,
  citecolor  = blue,
  breaklinks = true
}
% URL Line Breaking
\PassOptionsToPackage{hyphens}{url}
\urlstyle{same}
\def\Urlmuskip{0mu plus 1mu}
\def\UrlBreaks{\do\/\do-}
\def\UrlBigBreaks{\do\/\do-\do:\do.}


% ! =======================
% ! # MARK: Draft Watermark
% ! =======================

% Uncomment the following lines to add a "DRAFT" watermark on every page.
% \usepackage{background}
% \backgroundsetup{contents=DRAFT, opacity=0.25, color=gray}
% Line Spacing
% \doublespacing % Uncomment for review drafts


\begin{document}
\pagenumbering{roman}
\counterwithin{lstlisting}{section}
\counterwithin{figure}{section}
\counterwithin{table}{section}
\setlength{\footskip}{65pt}

% ! ===========================
% ! # MARK: Title, author, etc.
% ! ===========================

\title{\textbf{Death is an \\ Engineering Challenge}}
\author[1]{Daniel Burger}
\affil[1]{\textbf{Eightsix Science}}
\affil[ ]{\href{mailto:daniel@eightsix.science}{daniel@eightsix.science}}
\author[2]{Masataka Watanabe}
\affil[2]{\textbf{University of Tokyo}}
\affil[ ]{\href{mailto:watanabe@sys.t.u-tokyo.ac.jp}{watanabe@sys.t.u-tokyo.ac.jp}}
\author[3]{Gabriel Cunha}
\affil[3]{\textbf{Neurosyncs}}
\affil[ ]{\href{mailto:gabriel@neurosyncs.com}{gabriel@neurosyncs.com}}
\date{\textit{\today}}
\maketitle
\thispagestyle{empty}

\begin{sloppypar}

  \begin{figure}[ht]
    \centering
    \includegraphics[width=\textwidth]{figures/cover.png}
    \label{fig:cover}
  \end{figure}
  \newpage

  % ! ================
  % ! # MARK: Abstract
  % ! ================

  \begin{abstract}
    We introduce Synconetics, a new scientific discipline dedicated to solving death through synthetic consciousness mechanics—a set of practical, engineering-focused, transdisciplinary approaches grounded in solutions achievable today. Synconetics prioritises evidence-based, buildable technologies over philosophical speculation, aiming to preserve the continuity of human consciousness across different substrates.
  \end{abstract}

  \pagebreak
  \pagenumbering{Roman}
  \tableofcontents
  \pagebreak
  \listoffigures
  \pagebreak
  \listoftables
  \pagebreak
  \addcontentsline{toc}{section}{\lstlistlistingname}
  \lstlistoflistings
  \pagebreak
  \pagenumbering{arabic}

  % ! ========================
  % ! # MARK: Document content
  % ! ========================

  \section{Introduction}
  \label{sec:introduction}

  \subsection{First Principles of Death as an Engineering Problem}
  \label{sec:first-principles}

  Frame biological death not as an intrinsic necessity, but as a technical failure mode of the supporting substrate; consequently, it is potentially tractable through engineering intervention. <Critique: While framing death as a technical failure is a powerful perspective shift, is "intrinsic necessity" the right counterpoint? Death is intrinsically necessary for biological organisms as currently constituted due to thermodynamics, accumulated damage, etc. Perhaps better to frame it as "not a fundamental physical necessity" or "not an unsolvable problem imposed by fundamental laws"? Also, "potentially tractable" is appropriately cautious, but could we hint at why it's potentially tractable beyond just reframing it (e.g., because physical processes are, in principle, manipulable, and no known physical law forbids the indefinite persistence of complex, self-maintaining systems)? >

  Define death, from this engineering standpoint, as the irreversible cessation of the specific, complex physical processes that currently underpin an individual's continuous conscious experience. <Question: What level of specificity is implied by "specific, complex physical processes"? Molecular? Cellular? Network dynamics? Information-theoretic patterns instantiated physically? Is "irreversible" defined practically (cannot be reversed with current/foreseeable tech) or theoretically (e.g., due to information loss preventing reversal even in principle)? This definition is crucial and needs to be operationally tight, even if initially abstract.> <Critique: Does focusing solely on conscious experience risk neglecting potentially vital unconscious processes that support consciousness and identity? Should the definition encompass the cessation of processes necessary for consciousness, even if not directly constitutive of it?>

  Postulate the primary engineering objective: ensuring the uninterrupted continuation of an individual's unique stream of consciousness, necessitating methods that preserve its continuity across time and potentially across different supporting physical substrates. <Critique: "Stream of consciousness" is phenomenological. How does this translate into measurable engineering parameters or physical correlates that can be targeted? Is "uninterrupted continuation" absolute (no gaps, however brief, at any level of analysis) or functional (subjective sense of continuity maintained, potentially allowing for micro-interruptions below the threshold of subjective awareness or functional disruption)? The link between the phenomenological goal and the required physical continuity needs constant, explicit reinforcement and operationalisation.>

  Conceptualise individual conscious existence as a continuous physical process, intrinsically linked to its material instantiation. This process traces a unique four-dimensional world-line in spacetime. The engineering imperative is the preservation of this specific world-line's continuity, rigorously precluding destructive copying, pausing-and-restarting, or substitution with a functionally identical but distinct entity. <Feedback: This is a core concept. The "world-line" metaphor is potent but needs careful handling. Does it refer primarily to the matter constituting the substrate, the pattern of activity, or the process itself (which implies both specific matter undergoing specific activity)? Clarifying the answer to my earlier question (a, b, or c – likely (c)) is vital here. How rigorously can we define "destructive copying" vs. "gradual replacement/augmentation" in physical terms (e.g., based on percentage replaced per unit time, maintaining functional isomorphism throughout)? What constitutes an unacceptable "interruption" vs. acceptable "maintenance" or "stasis"? Is the preclusion of pausing-and-restarting an absolute axiom based on physical principles (e.g., information loss during pause), or a precautionary principle due to uncertainty about preserving identity through such a process? This strong stance requires robust justification, perhaps linking it to the physics of complex systems or information.>

  Shift the analytical focus from the biological organism per se to the phenomenon of death – the point of process cessation – to maintain a first-principles, problem-oriented engineering perspective. <Feedback: Clear and useful framing. Reinforces the engineering approach.>

  Distinguish the ultimate objective from mere lifespan extension of the current biological form. The aim is radical resilience against the cessation of conscious continuity – making dying substantially more difficult by mitigating substrate vulnerability and failure modes. <Critique: "Radical resilience" and "substantially more difficult" are qualitative. Can we hint at quantifiable metrics or target states, even if abstractly (e.g., resilience against specific classes of physical insults, achieving a target mean time between failure modes orders of magnitude greater than biological limits)? This would strengthen the engineering focus and differentiate it more sharply from standard longevity.>

  Identify the inherent fragility and limited repairability of the current biological substrate (specifically the brain) as the principal vulnerability point. Decoupling the essential processes of consciousness from exclusive reliance on this singular, fragile biological form emerges as a logical engineering strategy. <Feedback: Logical connection. "Decoupling" is a key term – does it imply full separation eventually, or primarily augmentation, repair, and partial replacement initially? Clarifying the scope of "decoupling" intended (partial vs. total) might be useful.>

  Emphasise that while component replacement external to the core processes of consciousness (e.g., organs, limbs) is compatible with identity continuity, any intervention involving the core physical substrate must meticulously maintain the unbroken continuity of the specific process-world-line. <Question: Where is the boundary drawn for the "core physical substrate"? Is it the entire brain? Specific critical regions (e.g., brainstem, thalamocortical loops)? Does this boundary depend on the current understanding of neural correlates of consciousness? Defining this boundary, even if provisionally and subject to revision, is critical for practical application and distinguishing permissible from impermissible interventions.>

  Introduce substrate independence not as an abstract philosophical notion, but as a potential engineering outcome: the possibility of consciousness-supporting processes persisting through gradual transition or augmentation involving non-biological components, contingent on maintaining process continuity. Note its distinction from conventional 'mind uploading'. <Feedback: Good distinction. "Gradual transition" is key. What defines "gradual" enough? Is it about the rate of change relative to the system's dynamics, the size/function of replaced components, or maintaining functional stability and subjective continuity throughout? This needs operational definition later, perhaps linking to timescales of neural adaptation or information integration.>

  \subsection{Critiquing Conventional Paradigms: The Need for a New Approach}
  \label{sec:new-paradigm}

  Characterise prevailing approaches often labelled 'mind uploading' (MU) or Whole Brain Emulation (WBE). Note their frequent foundation in computational neuroscience, assuming consciousness is fundamentally an abstract information pattern separable from its initial physical medium and perfectly replicable in silico. <Critique: Is it fair to characterise all WBE approaches this way? Some proponents (e.g., Sandberg and Bostrom's roadmap) argue for very high-fidelity physical simulation, potentially capturing more than just abstract patterns. Perhaps qualify with "Many prominent interpretations" or "Approaches focused solely on functional replication"? Also, "perfectly replicable" is a strong claim – perhaps "sufficiently replicable for functional equivalence" is what proponents often aim for, even if Synconetics disputes the sufficiency for identity continuity.>

  Critique the core assumption underpinning many MU/WBE proposals: that consciousness, along with personal identity, can be fully captured by abstracting and replicating functional or informational patterns, potentially neglecting the indispensable role of the specific, continuous dynamics of the physical substrate. This implicitly assumes a strong computational theory of mind is sufficient for identity preservation, which remains an unproven hypothesis. <Feedback: Clear critique. "Indispensable role" is the core counter-argument. What evidence, beyond intuition or philosophical preference for physicalism, supports this indispensability? Can we point to specific physical phenomena in the brain (e.g., quantum effects, unique material properties, analogue dynamics, chaotic sensitivity, specific thermodynamic properties) that are plausibly consciousness-relevant and difficult to abstract/replicate purely functionally or digitally without loss? Acknowledging the lack of definitive proof either way, but arguing for caution based on physicalism and the high stakes (existential risk), might be stronger.>

  Highlight the profound epistemological and practical challenge: achieving a sufficiently high-fidelity emulation or simulation that guarantees the preservation of the original consciousness (not merely creating a functional replica) may require modelling physical details and dynamics to a degree that approaches the complexity of the original system, potentially rendering it practically intractable or theoretically uncertain. <Feedback: Strong point. Connects to the limits of modelling complex systems. Could implicitly reference chaos theory, sensitivity to initial conditions, or the Landauer limit's implications for information erasure during copying.>

  Advocate for a crucial methodological shift: focus research and engineering efforts on the tangible, physical substrate (the brain) and its continuous processes. Ground the approach in established physics, neuroscience, and materials science, rather than relying primarily on abstract computational metaphors or philosophically contested concepts of 'mind' divorced from physical instantiation. <Feedback: Clear statement of the alternative methodology. Emphasises working with the existing physical system.>

  Argue that the reliance within some MU/WBE frameworks on ill-defined terms (e.g., 'information pattern' as sufficient for identity) and strong, unproven assumptions positions them closer to speculative philosophy or theoretical computer science than to pragmatic, buildable engineering solutions for preserving existing individuals. <Critique: While the critique is valid, calling it "closer to speculative philosophy" might sound dismissive rather than rigorously analytical. Perhaps phrase as "relies heavily on philosophical assumptions about identity that lack empirical validation and may not be testable" or "prioritises theoretical computational models over verifiable physical intervention strategies." Ensure the tone remains analytical and focused on methodological differences.>

  Acknowledge, without extensive detour, the widely discussed philosophical quandaries associated with destructive MU/WBE (e.g., the Ship of Theseus/teleportation paradox concerning identity and continuity; the verification problem regarding the internal state of the emulation – the 'philosophical zombie' possibility). These highlight the risks of approaches that do not guarantee continuity. <Feedback: Necessary acknowledgement. Frames these not just as philosophical puzzles but as indicators of practical risks related to the chosen methodology.>

  Conclude that many current MU/WBE paradigms fail to directly address the core engineering requirement defined here: the guaranteed, continuous preservation of the specific, individual 4D process-world-line. This necessitates a fundamental reorientation in methodology and goals. <Feedback: Logical conclusion based on the preceding points. Clearly states the divergence.>

  \subsection{Synconetics: Establishing a New Discipline}
  \label{sec:new-discipline}

  Propose the formal establishment of a distinct scientific and engineering discipline – Synconetics – dedicated to the challenge of ensuring conscious continuity through engineered means. This distinction is necessitated by the specific focus on continuity, the critique of alternative paradigms, and the required transdisciplinary approach. <Feedback: Clear statement of intent. Justifies the need for a new label.>

  Define Synconetics as the field focused on developing Synthetic Consciousness Mechanics: the practical, engineering-driven methodologies for interfacing with, augmenting, repairing, or gradually transitioning the physical substrate of consciousness to ensure its uninterrupted continuation. <Question: "Synthetic Consciousness Mechanics" – does "synthetic" refer primarily to the methods being engineered, or the potential outcome (a partially or wholly synthetic substrate)? Does "mechanics" imply a focus on deterministic, predictable interactions, potentially downplaying stochastic or emergent aspects of brain function critical to consciousness? Consider alternatives like "Continuity Engineering", "Substrate Mechanics for Consciousness", or "Applied Neurocontinuity". The chosen name should precisely reflect the focus.> <Critique: "Interfacing with, augmenting, repairing, or gradually transitioning" – this list is good, but is it exhaustive of the potential approaches within Synconetics? Does it cover, for instance, protective measures that don't involve direct substrate alteration?>

  Advocate retiring ambiguous and potentially misleading terms like 'mind uploading', favouring precise, operationally defined terminology grounded in the engineering objectives of continuity and substrate interaction/replacement. <Feedback: Strong justification for new terminology. Emphasises operational definitions.>

  Characterise Synconetics as inherently transdisciplinary, demanding synergistic integration of expertise from neuroscience (systems, cellular, molecular), neuroengineering, materials science, physics (especially condensed matter and non-equilibrium thermodynamics), bioengineering, robotics, phenomenology, and rigorous philosophy of mind (focused on identity and continuity), avoiding confinement within any single existing disciplinary framework. <Feedback: Good list, emphasises breadth. The inclusion of phenomenology and philosophy is important but needs careful integration to maintain the engineering focus – perhaps framed as providing essential constraints, ethical boundaries, or requirements for what constitutes successful continuity preservation from the first-person perspective.>

  Express concern regarding the potential misallocation of research effort and funding towards paradigms based on questionable assumptions about information patterns and destructive replication. Emphasise the need to explore diverse, continuity-preserving strategies. <Feedback: Justifies the need for a distinct field/funding stream, framed around methodological soundness.>

  Highlight the ethical imperative driving Synconetics: pursuing approaches that minimise existential risk to the individual, avoiding scenarios that could result in mere replication or the creation of entities lacking genuine consciousness (framing the 'philosophical zombie' concern as a critical failure mode to be rigorously avoided by focusing on continuity). <Feedback: Strong ethical grounding. Links continuity directly to mitigating existential risk and the zombie problem, positioning it as a risk-averse strategy.>

  State the primary purpose of Synconetics: to consolidate research efforts around verifiable, engineering-driven strategies for preserving conscious continuity, prioritising methodologies that are theoretically sound, ethically defensible, and potentially realisable with current or foreseeable technological advancements, rather than relying on distant, speculative breakthroughs in fundamental science (e.g., a complete theory of consciousness). <Critique: "Verifiable" – how can the preservation of subjective continuity be verified externally? This is a deep problem. Does "verifiable" here refer to the verification of the physical process continuity itself (e.g., continuous function, structure, energy flow at relevant scales), which is taken as the necessary condition or best possible proxy for subjective continuity? This crucial link and its assumptions need clarification. Also, "potentially realisable with current or foreseeable technological advancements" sets a pragmatic tone, but how is "foreseeable" defined (e.g., 10 years, 50 years)? This could be challenged as potentially limiting ambition, though it aligns with the engineering focus.>

  \subsection{Foundational Principles of Synconetics}
  \label{sec:foundational-principles}

  Principle 1 (Contingency of Death): Biological death is understood as a contingent failure of a specific type of complex system, not a metaphysical or logical necessity. Its prevention, circumvention, or reversal is therefore a valid and potentially achievable engineering objective. <Critique: Reiteration of the point in 1.1. Is "reversal" a realistic goal within Synconetics' focus on continuity? Reversal implies restarting after cessation, which seems counter to the core premise unless defined very carefully (e.g., reversal of the damage leading to cessation before cessation occurs). Perhaps focus on "prevention," "circumvention," and "mitigation"? Also, re-address the "intrinsic necessity" point from 1.1 – perhaps "not a fundamental physical necessity imposed by laws preventing indefinite complex organisation under suitable conditions".>

  Principle 2 (Engineering Methodology): The problem of ensuring conscious continuity must be addressed through rigorous, evidence-based engineering methodology. This involves focusing on physical processes, measurable parameters, and buildable systems, grounded in established scientific principles, while minimising reliance on untestable philosophical assumptions or specific, unproven theories of mind (e.g., strong computationalism). <Question: What constitutes "evidence" in this context, especially regarding the link between specific physical processes and subjective continuity? How are "measurable parameters" defined for the core process of consciousness itself, beyond neural correlates (e.g., measures of complexity, integration, specific dynamic patterns)? While avoiding unproven theories is good, doesn't any approach rely on some foundational assumptions (e.g., physicalism, the relevance of certain physical scales)? The key is making these assumptions explicit, testable where possible, and justifying them based on current science.>

  Principle 3 (Primacy of Continuity): The non-negotiable engineering target is the continuous, uninterrupted preservation of the individual's unique process-world-line, intrinsically linked to its physical instantiation. Methodologies must demonstrably preserve this continuity, sidestepping approaches based on destruction and replication. This focus on the how of continuity may precede a complete understanding of the what of consciousness. <Critique: "Non-negotiable" is strong; is it an axiom or a core hypothesis guiding the methodology? "Demonstrably preserve" – again, the verification challenge. How is this demonstrated? Through continuous monitoring of physical parameters assumed critical (structural, functional, dynamic)? Through maintaining causal links? Needs operationalisation, even if abstractly. The idea of focusing on 'how' before 'what' is pragmatic but needs defence against the objection that without knowing 'what' features are essential for consciousness/identity, we might preserve the wrong 'how' (e.g., preserving structure but losing crucial dynamics). How does Synconetics propose to identify the relevant aspects of the process to preserve?>

  Principle 4 (Hypothesis of Physical Realisability): Grounded in physicalism, Synconetics operates on the working hypothesis that the essential properties supporting consciousness arise from the dynamic organisation of matter and energy. If these dynamics can be understood and replicated or sustained through alternative physical means, then maintaining consciousness across modified or synthetic substrates is physically plausible, provided continuity is maintained during any transition. (Analogy: Understanding aerodynamic principles enabled engineered flight, distinct from biological flight; similarly, understanding the principles of neural function supporting consciousness could enable engineered solutions for its persistence). <Feedback: Clear statement of the physicalist grounding. "Understood and replicated or sustained" – the level of understanding required is key. Does it require a full reductive explanation, or a sufficient functional/dynamic understanding for engineering purposes (like the Wright brothers)? The analogy helps clarify this pragmatic stance.> <Question/Critique: Does "replicated" here clash with the anti-copying stance? Perhaps "sustained," "instantiated," or "perpetuated" in alternative substrates is better phrasing? The key is that the process continues and evolves, not that a static state is copied. The focus should be on maintaining the ongoing dynamics through substrate change.>

  Principle 5 (Focus on Core Goal): While potential secondary advantages might arise from non-biological substrates (e.g., enhanced durability, speed, environmental tolerance, modifiability), these are subordinate to the primary, non-negotiable goal of ensuring continuity and survival. The pursuit of enhancements must not compromise the core objective. Synconetics initially prioritises survival and resilience over augmentation. <Feedback: Important prioritisation, adds focus and manages scope. Reinforces the seriousness of the primary objective.>


  \section{Nomenclature and Definitions}
  \label{sec:nomenclature}

  % ? here we explain the nomenclature and definitions of the new discipline. sub-terms, etc. and what they all mean based on the first principles perspective, the axioms, etc.


  \section{Methodologies and Approaches}
  \label{sec:methodologies}

  % ? in order to not make this theoretical only, and in order to show that we can start in the field of synconetics today, we the co-authors will present two approaches aligned with everything we've described so far and being the first two approaches, to probably many more to come that achieve our goal of synthetic consciousness mechanics.

  \subsection{Daniel's Approach}
  \label{sec:daniel-approach}

  \subsection{Masataka's Approach}
  \label{sec:masataka-approach}



  \section{Roadmap and Funding}
  \label{sec:next-years}

  \section{Economics and Impact}
  \label{sec:economics}

  \section{Conclusion and Call to Action}
  \label{sec:conclusion}

  % ! ========================
  % ! # MARK: References, etc.
  % ! ========================

  \pagebreak
  \bibliographystyle{../../templates/custom-apa}
  \bibliography{references/bibliography}
  \nocite{*}

\end{sloppypar}
\end{document}
