\documentclass[10pt]{article}
% ! ========================
% ! # MARK: Template Control
% ! ========================

% Defines the template variant. Can be 'essay' (default) or 'paper'.
% This command must come before the document class definition in the main file.
\providecommand{\templatevariant}{essay}


% ! =======================
% ! # MARK: Package Loading
% ! =======================

% Common Packages
\usepackage[T1]{fontenc} % Font encoding
\usepackage{tgtermes} % TeX Gyre Termes font (Times clone)
\usepackage{geometry} % Page layout control
\usepackage{setspace} % Line spacing control
\usepackage{graphicx} % Enhanced graphics
\usepackage{xcolor} % Color definitions
\usepackage{colortbl} % Color in tables
\usepackage{hhline} % Double horizontal lines in tables
\usepackage{makecell} % Thicker lines in tables
\usepackage{tabularx, booktabs} % Advanced table layouts
\usepackage{enumitem} % Custom list environments
\usepackage{amsmath} % Advanced math environments
\usepackage{amssymb} % Math symbols
\usepackage{siunitx} % SI units package
\usepackage{listings} % Code listings
\usepackage{natbib} % Bibliography and citations
\usepackage{authblk} % Author and affiliation blocks
\usepackage{tocloft} % ToC, LoF, LoT styling
\usepackage[nottoc]{tocbibind} % Add Bib, Index, etc., to ToC
\usepackage{microtype} % Improved typography (justification, spacing)
\usepackage{url} % URL formatting
\usepackage[breaklinks,linktocpage]{hyperref} % Hyperlinks (must be loaded late)

% Variant-Specific Packages
% Define variant names for comparison
\def\papervariant{paper}
\def\essayvariant{essay}

% Load packages only required for the 'paper' template
\ifx\templatevariant\papervariant
  \usepackage{multicol}     % For multi-column layout
  \usepackage{dblfloatfix}  % Fixes for double-column floats
\fi


% ! ==============================
% ! # MARK: Page Layout & Geometry
% ! ==============================

\ifx\templatevariant\papervariant
  % Paper Template Layout (Two-Column)
  \geometry{
    left=3cm,
    right=3cm,
    top=2.5cm,
    bottom=3cm
  }
  % Settings for two-column layout
  \setlength{\columnseprule}{0pt} % No vertical rule between columns
  \setlength{\columnsep}{0.7cm}   % Space between columns
\else
  % Essay Template Layout (Single-Column, Default)
  \geometry{
    left=4cm,
    right=4cm,
    top=3cm,
    bottom=3.5cm
  }
\fi


% ! ==========================
% ! # MARK: Typography & Fonts
% ! ==========================

% Page Numbering
\renewcommand{\thepage}{\footnotesize\arabic{page}}

% List Spacing
\setlist[itemize]{noitemsep, topsep=7pt, partopsep=0pt, leftmargin=*, label=\textendash}


% ! ================================
% ! # MARK: Floats, Figures & Tables
% ! ================================

% Float Placement Parameters
\renewcommand{\topfraction}{0.9}
\renewcommand{\bottomfraction}{0.8}
\renewcommand{\textfraction}{0.1}
\renewcommand{\floatpagefraction}{0.8}
% Additional float settings for two-column paper template
\ifx\templatevariant\papervariant
  \renewcommand{\dbltopfraction}{0.9}
  \renewcommand{\dblfloatpagefraction}{0.8}
\fi

% Caption Styling
\usepackage[font=footnotesize,skip=7pt,labelfont=bf]{caption}
\captionsetup{justification=raggedright} % Left-align all captions
\newcommand{\floatcaption}[2]{\caption[#1.]{#1~#2.}} % Custom caption command


% ! ================================
% ! # MARK: Bibliography & Citations
% ! ================================

\renewcommand{\bibname}{References} % Change bibliography title
\setlength\bibindent{0pt}           % No indentation for bibliography entries

% Adjust layout and font size in the bibliography environment
\let\oldthebibliography=\thebibliography
\let\endoldthebibliography=\endthebibliography
\renewenvironment{thebibliography}[1]{%
  \begin{oldthebibliography}{#1}%
    \raggedright%
    \footnotesize%
    \setlength{\itemsep}{3pt}%
    \setlength{\parsep}{0pt}%
    \setlength{\parskip}{0pt}%
    }{%
  \end{oldthebibliography}%
}


% ! =====================
% ! # MARK: Code Listings
% ! =====================

% Custom Colors for Code
\definecolor{codegreen}{rgb}{0,0.5,0}
\definecolor{codegray}{rgb}{0.4,0.4,0.4}
\definecolor{codepurple}{rgb}{0.58,0,0.82}
\definecolor{backcolour}{rgb}{0.96,0.96,0.96}
\definecolor{lightgray}{gray}{0.8}

% Language Definition (Example: JavaScript)
\lstdefinelanguage{JavaScript}{
  keywords={break, case, catch, continue, debugger, default, delete, do, else, finally, for, function, if, in, instanceof, new, return, switch, this, throw, try, typeof, var, void, while, with},
  morecomment=[l]{//},
  morecomment=[s]{/*}{*/},
  morestring=[b]',
  morestring=[b]",
  sensitive=true
}

% Listing Style Definition
\lstdefinestyle{mystyle}{
  backgroundcolor=\color{backcolour},
  commentstyle=\color{codegreen},
  keywordstyle=\color{purple},
  numberstyle=\tiny\color{codegray},
  stringstyle=\color{codepurple},
  basicstyle=\ttfamily\footnotesize,
  breakatwhitespace=false,
  breaklines=true,
  captionpos=b,
  frame=tb,
  framerule=0pt,
  framextopmargin=6pt,
  framexbottommargin=6pt,
  keepspaces=true,
  numbers=left,
  numbersep=5pt,
  showspaces=false,
  showstringspaces=false,
  showtabs=false,
  tabsize=2
}
\lstset{style=mystyle} % Apply the defined style globally


% ! ===================================
% ! # MARK: Document Structure & Titles
% ! ===================================

% Table of Contents (ToC)
\renewcommand{\contentsname}{Table of Contents}
\renewcommand\cftsecafterpnum{\vskip8pt}              % Vertical space after section numbers
\renewcommand{\cftsecleader}{\cftdotfill{\cftdotsep}} % Dotted leaders for sections

% List of Listings (LoL)
\renewcommand{\lstlistlistingname}{List of \lstlistingname s}


% ! ==========================
% ! # MARK: Hyperlinks & URLs
% ! ==========================

\hypersetup{
  colorlinks = true,
  urlcolor   = blue,
  linkcolor  = blue,
  citecolor  = blue,
  breaklinks = true
}
% URL Line Breaking
\PassOptionsToPackage{hyphens}{url}
\urlstyle{same}
\def\Urlmuskip{0mu plus 1mu}
\def\UrlBreaks{\do\/\do-}
\def\UrlBigBreaks{\do\/\do-\do:\do.}


% ! =======================
% ! # MARK: Draft Watermark
% ! =======================

% Uncomment the following lines to add a "DRAFT" watermark on every page.
% \usepackage{background}
% \backgroundsetup{contents=DRAFT, opacity=0.25, color=gray}
% Line Spacing
% \doublespacing % Uncomment for review drafts


\begin{document}
\pagenumbering{roman}
\counterwithin{lstlisting}{section}
\counterwithin{figure}{section}
\counterwithin{table}{section}
\setlength{\footskip}{65pt}

% ! ===========================
% ! # MARK: Title, author, etc.
% ! ===========================

\title{\textbf{Death is an \\ Engineering Challenge}}
\author[1]{Daniel Burger}
\affil[1]{\textbf{Eightsix Science}}
\affil[ ]{\href{mailto:daniel@eightsix.science}{daniel@eightsix.science}}
\author[2]{Masataka Watanabe}
\affil[2]{\textbf{University of Tokyo}}
\affil[ ]{\href{mailto:watanabe@sys.t.u-tokyo.ac.jp}{watanabe@sys.t.u-tokyo.ac.jp}}
\author[3]{Gabriel Cunha}
\affil[3]{\textbf{Neurosyncs}}
\affil[ ]{\href{mailto:gabriel@neurosyncs.com}{gabriel@neurosyncs.com}}
\date{\textit{\today}}
\maketitle
\thispagestyle{empty}

\begin{sloppypar}

  \begin{figure}[ht]
    \centering
    \includegraphics[width=\textwidth]{figures/cover.png}
    \label{fig:cover}
  \end{figure}
  \newpage

  % ! ================
  % ! # MARK: Abstract
  % ! ================

  \begin{abstract}
    We introduce Synconetics, a new scientific discipline dedicated to solving death through synthetic consciousness mechanics—a set of practical, engineering-focused, transdisciplinary approaches grounded in solutions achievable today. Synconetics prioritises evidence-based, buildable technologies over philosophical speculation, aiming to preserve the continuity of human consciousness across different substrates.
  \end{abstract}

  \pagebreak
  \pagenumbering{Roman}
  \tableofcontents
  \pagebreak
  \listoffigures
  \pagebreak
  \listoftables
  \pagebreak
  \addcontentsline{toc}{section}{\lstlistlistingname}
  \lstlistoflistings
  \pagebreak
  \pagenumbering{arabic}

  % ! ========================
  % ! # MARK: Document content
  % ! ========================

  \section{Introduction}
  \label{sec:introduction}

  \subsection{First Principles of Death}
  \label{sec:first-principles}

  % ? here the goal is to explain what we mean by treating death as an engineering challenge. that we look at it from a first principles perspective and that purely from this perspective, death seems to be solvable as we need to preserve the 4D world line, whatever that exactly means, without really understanding it. maybe draw examples from rocketry, as we don't need to understand the fundamental law of quantum mechanics to build a useful rocket engine to land a man on the moon.

  - we want to treat death as an engineering challenge. what we mean here, is that we see what death means <explain what death is from a very first principles perspective> and that it is purely from first principles perspective, an engineering challenge.
  - we want to preserve the continuity of human consciousness across different substrates (substrate independent minds)
  - we look at whatever we are (our conciousness) as a process of several actions in physical space, or receiving whatever is being received if this is e.g. also the case for our conciousness, and look at it as a 4D world line (as in the case of general relativity) and whatever happens, we try to preserve that. no interruption, no copy, no new line emerging next to it, even if the same properties, as it's an atomic process in the 4D world line.
  - we focus on death itself as the focus, not the object that defines death (the human body), as this shifts the focus from the problem to the solution and first principles.
  - the end goal is lifespan extension, but compared to existing longevity approaches, we are not only trying to defeat death, but make it really hard to die. we can land a man on the moon, but we it's still incredible hard to do so.
  - the reason for this is that we disconnect the mind from the body. as it's the body where we're embodied that makes it easy to die. even if we can extend the lifespan, it's super easy for somebody to stab you in the face and kill you. the most radical approach to defeating death from a first principles approach is to preserve this 4D line. we can replace organs, we can replace our entire body, but when we start what makes us, presumably in the brain, we need to preserve that.
  - we need to preserve the 4D world line. for that there are a few approaches, the most prominent to this is the idea of achieving substrate independence. and the most prominent approach to that is to achieve it through what most people call mind uploading. however there are some problems with the current way of thinking that surrounds the research of mind uploading.

  \subsection{A New Way of Thinking}
  \label{sec:new-thinking}

  % ? here we want to explain that we need a new way of thinking about death as we shift the focus from the problem to the solution due to focusing on death.

  - explain what mind uploading is and how it mostly relates to the field of whole brain emulation.
  - say that there are a few problems with it, as the main problem is that we focus on the mind itself to be preserved, and not the underlying substrate, which is key to preserve the 4D world line. we focus on this blurry concept of mind and conciousness and assume we can model it on a computer to preserve it. however, reality is more complex than our best models. the best model would include every parameter of reality in order to model reality to its best satisfaction, however, such a model would be reality itself (hence we assume that trying to emulate reality inside reality itself is asymptotically approaches infinity)
  - we need to focus on the underlying substrate, the brain, and work with what we know for sure, such as that consciousness exists, that brains exist, that reality is physical, and that the brain is physical and that such a concept of a 4D line exists.
  - the problem with mind uploading already assumes computability, that the mind is something of a process, and generally everything is very inspired by computer science. mind uploading has the fundamental flaw in thinking that we focus on the blurry concepts that are hard define, and not the physical obvious things we have at hand, hence mind uploading is never an engineering approach that can actually be built with strong confidence and without huge assumptions and beliefs. mind uploading seems more of a religion than a science just purely based on this way of thinking.
  - there are a bunch of other problems with mind uploading, which we don't want to deepen too much as it has been discussed a lot already, such as the teleportation paradox, scale separation, philosophical zombies, etc. what is evident, that next to the underlying wrong way of thinking of the actual problem in the room; preserving the 4D world line in order to cure death and make it really hard to die, we need to shift our way of thinking away from this and into something that is actually an engineering approach with first principles we know for sure without philosophical assumptions.

  \subsection{New Discipline Needed}
  \label{sec:new-discipline}

  % ? here we want to explain why we want to differentiate ourselves from the very prominent field of mind uploading and sister-field of whole brain emulation due to the previously explained reasons as well as some other reasons mentioned here. basically it;s also like we don't want that a huge scientific body is working on just one way based one one way of thinking that we accuse of being faulty and problematic and then make it impossible for other approaches to be taken seriously (funding etc). which actually all work towards the same goal. <maybe make an example of string theory in physics in the 80s, if that's applicable>

  - Due to these problems, we want to differentiate ourself from the current way of thinking in the works from people working on mind uploading, and create a new discipline that separates itself from the current way of thinking.
  - we need to create a new discipline that is focused on the engineering approach to death, and that is focused on the first principles perspective of death.
  - generally want to retire the term mind uploading, as it's a very vague term that doesn't really tell us anything about what we're actually trying to achieve. as well as it comes from science fiction, and not science.
  - <describe what needs to be done to create a new scientific discipline>
  - this new discipline shouldn't be coming from one singular discipline as a subdiscipline <give an example> or a mix of disciplines (such as neuropsychology), but it should be a new discipline on its own combing transdisciplinary approaches to solve the problem of death.
  - e.g. mind uploading, or whole brain emulation, which is <a sister of mind uploading>, are all subdisciplines of computer science with neuroscience (computational neuroscience), which limit the thinking of the current space in this field, which is clearly the case as we have totally disconnected discussions on the actual problem in the room.
  - we want to not let the majority of funding and efforts go into the field of mind uploading due to the explained issues of thinking, and philosophical issues as well, where we all might end up in a great filter, where we are all simulated minds on computers, which would be a graveyard of philosophical zombies, that would be the funniest tragedy of the great filter
  - hence bringing together people into a new discipline, with some aligned ideas and concepts to work on what actually matters; defeating death and making it really hard to die, not in the next 200 years with assumptions of technological advances that might be solved (such as what consciousness is, if machine consciousness is even possible, nano tech, etc.) all things that don't exist, but with an actual engineering approach to the problem that we can start working on today, and tomorrow in order to solve it the day after tomorrow.

  \subsection{Aligned Values, axioms?}
  \label{sec:aligned-values}

  % ? here we want to explain the basic axioms that we are working with and how we lead towards synthetic consciousness mechanics. this part needs to be extremely rigorous and based on first principles.

  - death is avoidable and desired
  - death needs to be treated as an engineering problem grounded in first principles, totally freed from philosophical assumptions and potentially faulty ways of thinking.
  - we want to make dying hard in order to preserve this 4D world line, whatever that means. (generally, it's about sidestepping assumptions, but from an external perspective, where we can work with it, but not ignore it like in e.g. dogmas in religions).
  - we say that substrate independence might be possible, as the brain itself is matter in reality composed in a specific way, and if we could replicate this in reality to 100\%, we would get a concious being as nature has given us, so there might be some law of consciousness. we don't wait for these pure scientific findings to be known, but we say that from a pure first principles point of view, it's possible. if whatever that is, needs to be biological or e.g. synthetic or as in the case of mind uploading artificial, we don't know, but maybe. <make comparison of wright brothers studying bird flight, building synthetic bird wings, finding the underlying laws of flying (not really understanding it perfectly, but making it an engineering-perspective okay enough to work for what goal they wanted to achieve: making humans flight for a very long time), and then how they built the artificial wing and motors and hence motorized flight for humans. after a few years, we built then better wings, and concepts such as the jet engine, which let us fly supersonic, which is not something evolution has given us, but something where we needed to escape the velocity of evolution constrained in biology and build things that weren't possible without. we say that we want to have this for the brain, whatever that will mean. but we need to start with what works today, which are our brains, and the constraints of understanding of the mind, consciousness, etc.>
  - there are certain benefits with computerized brains, such as the ability to store and process information in a way that is not constrained by the physical limits of the brain, co-location of the brain in physical space, higher processing speeds, etc. – however, these are assumptions and desires and go beyond the main goal for now to defeat death and make it really hard to die. there is also the concept of making death impossible, which might be true in the future after we "find the jet engine of the brain to fly supersonic", but we don't need to wait for that in order for us to start working on the problem.
  - we hence focus on the main goal, having something to work with to get a concious 4D line being preserved with continuity sidestepping all issues with other approaches, and making this synthetic consciousness mechanics.


  \section{Nomenclature and Definitions}
  \label{sec:nomenclature}

  % ? here we explain the nomenclature and definitions of the new discipline. sub-terms, etc. and what they all mean based on the first principles perspective, the axioms, etc.


  \section{Methodologies and Approaches}
  \label{sec:methodologies}

  % ? in order to not make this theoretical only, and in order to show that we can start in the field of synconetics today, we the co-authors will present two approaches aligned with everything we've described so far and being the first two approaches, to probably many more to come that achieve our goal of synthetic consciousness mechanics.

  \subsection{Daniel's Approach}
  \label{sec:daniel-approach}

  \subsection{Masataka's Approach}
  \label{sec:masataka-approach}



  \section{Roadmap and Funding}
  \label{sec:next-years}

  \section{Economics and Impact}
  \label{sec:economics}

  \section{Conclusion and Call to Action}
  \label{sec:conclusion}

  % ! ========================
  % ! # MARK: References, etc.
  % ! ========================

  \pagebreak
  \bibliographystyle{../../templates/custom-apa}
  \bibliography{references/bibliography}
  \nocite{*}

\end{sloppypar}
\end{document}
