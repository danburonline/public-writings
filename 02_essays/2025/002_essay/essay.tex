\documentclass[10pt]{article}
% ! ========================
% ! # MARK: Template Control
% ! ========================

% Defines the template variant. Can be 'essay' (default) or 'paper'.
% This command must come before the document class definition in the main file.
\providecommand{\templatevariant}{essay}


% ! =======================
% ! # MARK: Package Loading
% ! =======================

% Common Packages
\usepackage[T1]{fontenc} % Font encoding
\usepackage{tgtermes} % TeX Gyre Termes font (Times clone)
\usepackage{geometry} % Page layout control
\usepackage{setspace} % Line spacing control
\usepackage{graphicx} % Enhanced graphics
\usepackage{xcolor} % Color definitions
\usepackage{colortbl} % Color in tables
\usepackage{hhline} % Double horizontal lines in tables
\usepackage{makecell} % Thicker lines in tables
\usepackage{tabularx, booktabs} % Advanced table layouts
\usepackage{enumitem} % Custom list environments
\usepackage{amsmath} % Advanced math environments
\usepackage{amssymb} % Math symbols
\usepackage{siunitx} % SI units package
\usepackage{listings} % Code listings
\usepackage{natbib} % Bibliography and citations
\usepackage{authblk} % Author and affiliation blocks
\usepackage{tocloft} % ToC, LoF, LoT styling
\usepackage[nottoc]{tocbibind} % Add Bib, Index, etc., to ToC
\usepackage{microtype} % Improved typography (justification, spacing)
\usepackage{url} % URL formatting
\usepackage[breaklinks,linktocpage]{hyperref} % Hyperlinks (must be loaded late)

% Variant-Specific Packages
% Define variant names for comparison
\def\papervariant{paper}
\def\essayvariant{essay}

% Load packages only required for the 'paper' template
\ifx\templatevariant\papervariant
  \usepackage{multicol}     % For multi-column layout
  \usepackage{dblfloatfix}  % Fixes for double-column floats
\fi


% ! ==============================
% ! # MARK: Page Layout & Geometry
% ! ==============================

\ifx\templatevariant\papervariant
  % Paper Template Layout (Two-Column)
  \geometry{
    left=3cm,
    right=3cm,
    top=2.5cm,
    bottom=3cm
  }
  % Settings for two-column layout
  \setlength{\columnseprule}{0pt} % No vertical rule between columns
  \setlength{\columnsep}{0.7cm}   % Space between columns
\else
  % Essay Template Layout (Single-Column, Default)
  \geometry{
    left=4cm,
    right=4cm,
    top=3cm,
    bottom=3.5cm
  }
\fi


% ! ==========================
% ! # MARK: Typography & Fonts
% ! ==========================

% Page Numbering
\renewcommand{\thepage}{\footnotesize\arabic{page}}

% List Spacing
\setlist[itemize]{noitemsep, topsep=7pt, partopsep=0pt, leftmargin=*, label=\textendash}


% ! ================================
% ! # MARK: Floats, Figures & Tables
% ! ================================

% Float Placement Parameters
\renewcommand{\topfraction}{0.9}
\renewcommand{\bottomfraction}{0.8}
\renewcommand{\textfraction}{0.1}
\renewcommand{\floatpagefraction}{0.8}
% Additional float settings for two-column paper template
\ifx\templatevariant\papervariant
  \renewcommand{\dbltopfraction}{0.9}
  \renewcommand{\dblfloatpagefraction}{0.8}
\fi

% Caption Styling
\usepackage[font=footnotesize,skip=7pt,labelfont=bf]{caption}
\captionsetup{justification=raggedright} % Left-align all captions
\newcommand{\floatcaption}[2]{\caption[#1.]{#1~#2.}} % Custom caption command


% ! ================================
% ! # MARK: Bibliography & Citations
% ! ================================

\renewcommand{\bibname}{References} % Change bibliography title
\setlength\bibindent{0pt}           % No indentation for bibliography entries

% Adjust layout and font size in the bibliography environment
\let\oldthebibliography=\thebibliography
\let\endoldthebibliography=\endthebibliography
\renewenvironment{thebibliography}[1]{%
  \begin{oldthebibliography}{#1}%
    \raggedright%
    \footnotesize%
    \setlength{\itemsep}{3pt}%
    \setlength{\parsep}{0pt}%
    \setlength{\parskip}{0pt}%
    }{%
  \end{oldthebibliography}%
}


% ! =====================
% ! # MARK: Code Listings
% ! =====================

% Custom Colors for Code
\definecolor{codegreen}{rgb}{0,0.5,0}
\definecolor{codegray}{rgb}{0.4,0.4,0.4}
\definecolor{codepurple}{rgb}{0.58,0,0.82}
\definecolor{backcolour}{rgb}{0.96,0.96,0.96}
\definecolor{lightgray}{gray}{0.8}

% Language Definition (Example: JavaScript)
\lstdefinelanguage{JavaScript}{
  keywords={break, case, catch, continue, debugger, default, delete, do, else, finally, for, function, if, in, instanceof, new, return, switch, this, throw, try, typeof, var, void, while, with},
  morecomment=[l]{//},
  morecomment=[s]{/*}{*/},
  morestring=[b]',
  morestring=[b]",
  sensitive=true
}

% Listing Style Definition
\lstdefinestyle{mystyle}{
  backgroundcolor=\color{backcolour},
  commentstyle=\color{codegreen},
  keywordstyle=\color{purple},
  numberstyle=\tiny\color{codegray},
  stringstyle=\color{codepurple},
  basicstyle=\ttfamily\footnotesize,
  breakatwhitespace=false,
  breaklines=true,
  captionpos=b,
  frame=tb,
  framerule=0pt,
  framextopmargin=6pt,
  framexbottommargin=6pt,
  keepspaces=true,
  numbers=left,
  numbersep=5pt,
  showspaces=false,
  showstringspaces=false,
  showtabs=false,
  tabsize=2
}
\lstset{style=mystyle} % Apply the defined style globally


% ! ===================================
% ! # MARK: Document Structure & Titles
% ! ===================================

% Table of Contents (ToC)
\renewcommand{\contentsname}{Table of Contents}
\renewcommand\cftsecafterpnum{\vskip8pt}              % Vertical space after section numbers
\renewcommand{\cftsecleader}{\cftdotfill{\cftdotsep}} % Dotted leaders for sections

% List of Listings (LoL)
\renewcommand{\lstlistlistingname}{List of \lstlistingname s}


% ! ==========================
% ! # MARK: Hyperlinks & URLs
% ! ==========================

\hypersetup{
  colorlinks = true,
  urlcolor   = blue,
  linkcolor  = blue,
  citecolor  = blue,
  breaklinks = true
}
% URL Line Breaking
\PassOptionsToPackage{hyphens}{url}
\urlstyle{same}
\def\Urlmuskip{0mu plus 1mu}
\def\UrlBreaks{\do\/\do-}
\def\UrlBigBreaks{\do\/\do-\do:\do.}


% ! =======================
% ! # MARK: Draft Watermark
% ! =======================

% Uncomment the following lines to add a "DRAFT" watermark on every page.
% \usepackage{background}
% \backgroundsetup{contents=DRAFT, opacity=0.25, color=gray}
% Line Spacing
% \doublespacing % Uncomment for review drafts


\begin{document}
\pagenumbering{roman}
\counterwithin{lstlisting}{section}
\counterwithin{figure}{section}
\counterwithin{table}{section}
\setlength{\footskip}{65pt}

% ! ===========================
% ! # MARK: Title, author, etc.
% ! ===========================

\title{\textbf{Death is an Engineering Challenge}}
\author[1]{Daniel Burger}
\affil[1]{\textbf{Eightsix Science Ltd}}
\affil[ ]{\href{mailto:daniel@eightsix.science}{daniel@eightsix.science}}
\author[3]{Gabriel Cunha}
\affil[3]{\textbf{Tufts University}}
\affil[ ]{\href{mailto:gabriel.cunha@tufts.edu}{gabriel.cunha@tufts.edu}}
\author[4]{Izumi Handa}
\affil[4]{\textbf{Panda Lab Inc}}
\affil[ ]{\href{mailto:izumi.handa@pandalab.jp}{izumi.handa@pandalab.jp}}
\author[2]{Masataka Watanabe}
\affil[2]{\textbf{University of Tokyo}}
\affil[ ]{\href{mailto:watanabe@sys.t.u-tokyo.ac.jp}{watanabe@sys.t.u-tokyo.ac.jp}}

\date{\textit{\today}}
\maketitle
\thispagestyle{empty}

\begin{sloppypar}

  \begin{figure}[ht]
    \centering
    \includegraphics[width=\textwidth]{figures/cover.png}
    \label{fig:cover}
    \caption{\% TODO: The 4D process world line of a conscious substrate going through space-time.}
  \end{figure}
  \newpage

  % ! ================
  % ! # MARK: Abstract
  % ! ================


  \begin{abstract}
    We view death as the irreversible destruction of consciousness’s physical and dynamic processes and frame it as a manageable systems problem solvable through engineering. By that, we propose synconetics, a new discipline dedicated to sustaining conscious continuity with current and near-term technologies that we can empirically test. This essay outlines the principles of synconetics and introduces two practical approaches to achieve this goal within the next twenty years.
  \end{abstract}

  \pagebreak
  \pagenumbering{Roman}
  \tableofcontents
  \pagebreak
  \listoffigures
  \pagebreak
  % \listoftables
  % \pagebreak
  % \addcontentsline{toc}{section}{\lstlistlistingname}
  % \lstlistoflistings
  % \pagebreak
  \pagenumbering{arabic}

  % ! ========================
  % ! # MARK: Document content
  % ! ========================

  \section{Introduction}
  \label{sec:introduction}

  Months of collaborative alignment within our working group, involving rigorous debate and scrutiny of prior schools of thought, have shown fundamental shortcomings in conventional approaches to solving death. As a result of this critical process, we introduce synconetics. This essay presents the first formal articulation of this new discipline, outlining a cohesive paradigm born from our shared analysis and convictions. We stress, however, that this represents an initial definition of an emerging field; the written work presented herein, as will become apparent throughout the essay, requires significant refinement through broader engagement and critique. Recognising this necessity, we invite collaborators from academia and industry to join us in challenging, developing, and advancing synconetics.

  At its core, synconetics is a scientific discipline focused on understanding the fundamental aspects of death and overcoming it through the development of synthetic consciousness mechanics—a framework for engineering interventions designed to sustain the physical-dynamical processes of consciousness using resilient substrates. By grounding its approach in empirically testable systems and near-term technologies over speculative and non-falsifiable approaches, synconetics bypasses philosophical speculation about the true nature of consciousness’s cessation (i.e., death) and establishes consciousness continuity preservation as a concrete engineering challenge, forming the discipline’s foundational principles.

  \subsection{First Principles of Death}
  \label{sec:first-principles}

  Death is not an inevitability but a contingent failure of the organised physical system that instantiates consciousness (i.e., the conscious substrate). Physical processes are, in principle, manipulable: no fundamental law prohibits the indefinite persistence of self-maintaining systems under engineered conditions. We, therefore, define dying operationally as the irreversible destruction of the processes sustaining conscious continuity over time.

  \begin{figure}[ht!]
    \centering
    \includegraphics[width=\textwidth]{figures/4D-process-world-line-ending.png}
    \caption{\% TODO: The 4D process world line of a conscious substrate, what it means if it decays (i.e., death, irreversible loss of conscious continuity, the past is immutable), and what it means when we preserve it over another substrate through substrate independence.}
    \label{fig:process-world-line-ending}
  \end{figure}

  The term conscious substrate denotes any physical medium whose organisation and dynamics have been shown to support conscious experience. Biological substrates, such as the central nervous system (CNS), remain the only empirically confirmed examples. This definition does not restrict consciousness to biology but acknowledges current empirical constraints. Individuals possess direct, if subjective, evidence of their own conscious continuity (e.g., “I think, therefore I am”), providing a provisional anchor for engineering objectives.

  Death marks the irreversible termination of the four-dimensional process world-line—the spatiotemporal trajectory of physical states within the conscious substrate that collectively underpins conscious continuity (see \autoref{fig:process-world-line-ending} for a schematic visualisation). Irreversibility reflects thermodynamic reality: entropy increase erodes the recoverability of prior states. Synconetics addresses this through open-system engineering, countering entropy, for example, via energy/matter exchange and error correction. While the exact organisational level essential for consciousness remains unclear (whether cellular, molecular, atomic, etc.), engineering pragmatism prevails: we do not need to fully understand a system to work productively with it and utilise what we have at hand.

  \subsection{Critiquing Conventional Paradigms}
  \label{sec:new-paradigm}

  \begin{figure}[ht!]
    \centering
    \includegraphics[width=\textwidth]{figures/other-approaches.png}
    \caption{\% TODO: Show the line of the 4D process world line compared to how other approaches are doing it.}
    \label{fig:other-approaches}
  \end{figure}

  Current approaches to overcoming death, whether by delaying or reversing substrate degradation through conventional longevity research (e.g. through lifestyle interventions, genetic engineering, etc.) or by pursuing true substrate independence via computational abstraction (for instance, mind uploading or whole brain emulation), fail to address a fundamental issue: the fragility of the four-dimensional process-world-line that maintains continuous consciousness. Synconetics seeks to safeguard the physical-dynamical continuity of this process while potentially enabling measured transitions to synthetic substrates that promise greater resilience than purely biological ones.

  Mind uploading (MU) and Whole Brain Emulation (WBE) assume that consciousness can survive as a computational abstraction. This is an unverified leap that may pose an existential threat if consciousness proves inseparable from its original physical substrate. Synconetics, therefore, adopts a cautious stance in that regard. Synthetic substrates may indeed be more robust, but only with thorough, incremental validation that does not break continuity. By integrating advanced elements such as brain-machine interfaces in ways that preserve the ongoing dynamics of the central nervous system, synconetics aims to maintain an unbroken 4D process-world-line while testing the feasibility of eventually instantiating consciousness in substrates beyond standard biology. Three unresolved challenges exemplify the risks of premature reliance on MU/WBE:

  \newpage

  \begin{enumerate}
    \item \textbf{Philosophical Zombies}: MU/ME’s assumption of substrate independence—that consciousness can “run” on arbitrary hardware—lacks empirical evidence. Computational models simulate neural correlates (e.g., firing rates, connectomes) but cannot verify whether subjective experience persists post-transfer. This leap conflates necessary conditions (physical dynamics like ion gradients and metabolic cycles) with sufficient conditions (unproven for digital substrates). Destructive uploading risks terminating the original process-world-line without verifiable assurance of qualia preservation, effectively gambling one’s existence on unconfirmed substrate equivalency. Even if functional continuity is achieved (à la Chalmers’ ‘gradual replacement’ of biological substrates with digital ones \citep{chalmers_conscious_1998}), absent empirical methods to confirm subjective survival, the replica’s consciousness remains an untestable conjecture—a leap no rigorous discipline should endorse.

    \item \textbf{Scale Separability}: Emulating brains at biologically relevant scales demands energy exceeding Earth’s projected budget for centuries \citep{bostrom_whole_2008}. Modern supercomputers simulate <1\% of a human brain’s synaptic activity at 1,000\(\times\) slowed time—a trivial fraction of real-time biological efficiency (20W vs. megawatts). Crucially, the brain’s dynamics blend deterministic chaos and stochastic noise as quantum-scale fluctuations (e.g., ion channel gating) propagate into macroscopic neural activity. This “web of causality” \citep{watanabe_biological_2022}—though shorter-lived than classical models suggest—defies reduction to deterministic simulations. Worse, fidelity requirements escalate exponentially if consciousness depends on quantum effects (e.g., microtubule coherence as proposed by Orch-OR). To simulate reality, we asymptotically approach infinity by simulating it within itself (i.e. infinite recursion).

    \item \textbf{Teleportation Paradox}: Most MU/WBE approaches rely on destructive scanning or non-destructive copying to transfer consciousness onto a computational substrate. Both methods sever the original’s causal continuity: destructive scans terminate the 4D process-world-line outright, while non-destructive copies spawn a parallel entity divorced from your subjective stream. Even “perfect” replication creates a new 4D trajectory; you do not experience the replica’s existence. Critics dismiss this as philosophical nitpicking—“I don’t care if it’s a copy, as long as it thinks it’s me.” But survival hinges on \emph{your} conscious continuity, not a replica’s beliefs. Even if half of humanity accepted copy-based “survival,” the other half would reject existential roulette. Synconetics prioritises solutions that preserve the 4D process-world-line outright—ensuring no one is forced to gamble their existence on untested metaphysics.
  \end{enumerate}

  \begin{figure}[ht!]
    \centering
    \includegraphics[width=\textwidth]{figures/4d-trajectory-disembodiment.png}
    \caption{\% TODO: Showing a graph similar to the one from the old draft in 4D, but with the disembodiment of the conscious substrate on the y-axis and the time on the x-axis. What it means to disembody it from its bodily substrate.}
    \label{fig:4d-trajectory-disembodiment}
  \end{figure}

  Critics may accuse synconetics of biological bias, but this misunderstands the burden of proof: synthetic substrates must demonstrate conscious continuity before replacing biology. Until then, privileging verified systems is engineering prudence, not chauvinism. The imperative is clear: abandon replication metaphysics and address substrate fragility of the 4D process-world-line.

  \subsection{Principles of Synconetics}
  \label{sec:principles}

  Synconetics emerges as a novel field defined by its foundational principles as shown in \autoref{tab:synconetics-principles}. We deliberately positioned it as a new, transdisciplinary scientific discipline, not confined to being a subdiscipline of any single existing field. This positioning allows it to integrate and innovate across multiple domains, creating a unique framework for addressing the challenges of consciousness continuity. The term synconetics refers to the core scientific discipline, while synthetic consciousness mechanics (SCM) denotes the applied engineering methodologies for interfacing with, stabilising, repairing, or transitioning consciousness-supporting substrates while maintaining the 4D process-world-line.

  \begin{table}[ht!]
    \centering
    \begin{tabular}{p{0.3\textwidth}p{0.65\textwidth}}
      \toprule
      \textbf{Principle}                            & \textbf{Description}                                                                                                                                                                                                               \\
      \midrule
      Process-World-Line Fidelity                   & Prioritises the preservation of the unbroken 4D causal chain of the physical dynamics underlying consciousness.                                                                                                                    \\
      \midrule
      Non-Destructive Transition                    & Employs gradual replacement or repair validated by first-person reportability, explicitly rejecting destructive scanning or copying methods.                                                                                       \\
      \midrule
      Substrate Agnosticism with Physical Grounding & Focuses on necessary physical dynamics rather than material composition (biological vs. synthetic), while requiring substrates to operate as thermodynamically open systems capable of sustaining consciousness-critical dynamics. \\
      \midrule
      Empirical Primacy                             & Prioritises testable interventions and falsifiable claims over philosophical speculation. Validates continuity through physical metrics and first-person reports, avoiding non-falsifiable theories of consciousness.              \\
      \midrule
      Near-Term Engineering Urgency                 & Leverages existing technologies and incremental advancements rather than awaiting speculative breakthroughs. Focuses on iterative progress using 2020s-era tools while planning for future scalability.                            \\
      \bottomrule
    \end{tabular}
    \caption{Core Principles of Synconetics}
    \label{tab:synconetics-principles}
  \end{table}

  \subsection{Nomenclature and Definitions}
  \label{sec:nomenclature}

  To ensure operational rigour and circumvent the conceptual ambiguities often found in related fields or conventional paradigms, synconetics employs an initial lexicon grounded directly in its foundational principles. The following core terms delineate the operational scope and key components:

  \begin{itemize}
    \item \textbf{Synthetic Consciousness Substrate (SCS)}: Any thermodynamically open physical system (biological, biohybrid, or artificial) engineered or verified to be capable of sustaining the critical dynamics underlying conscious experience. Validation requires demonstrating neural-phenomenological isomorphism with biological benchmarks, support for millisecond-scale feedback loops essential for integrated cognitive functions, and confirmation via first-person reporting during phased, non-destructive integration.

    \item \textbf{Synthetic Consciousness Migration (SCM)}: Protocols enabling the non-destructive migration of conscious processes between substrates (e.g., from biological to SCS or between SCSs), ensuring the causal topology of the 4D process-world-line remains intact. Distinct from destructive ‘scan-and-copy’ paradigms associated with conventional mind uploading, SCM focuses exclusively on methods like gradual, plasticity-aligned replacement or seamless BMI-mediated integration. The objective is to facilitate the transition to potentially more resilient or adaptable substrates while guaranteeing that the operational conditions for Continuity (see \autoref{eq:continuity}) are met throughout the process, verified by both continuous physical monitoring and first-person phenomenological reports.

    \item \textbf{Synthetic Consciousness Interfacing (SCI)}: The development and application of robust, bidirectional input/output systems connecting a consciousness-supporting substrate (SCS) to external environments (physical, virtual, or hybrid) or other systems designed to operate without compromising substrate integrity or processual continuity. The primary goal is to provide the conscious entity with effective agency—the means to perceive, interact, and act coherently—while also ensuring the quality and richness of interaction necessary for long-term psychological well-being and a preserved sense of self.

    \item \textbf{Processual Continuity (PC)}: Defined as the uninterrupted persistence of the causal topology—the specific, dynamic physical processes—constituting an individual’s 4D process-world-line, considered essential for maintaining their unique conscious identity and subjective experience. Operationally, this continuity \( \mathcal{C} \) is maintained during an intervention or process \( \mathcal{I} \) over a time interval \( [t_0, t_1] \) if the trajectory of the system’s critical state variables \( S_{\text{crit}}(t) \) evolves within the bounds of its adaptive capacity. This condition can be formalised by requiring that the magnitude of the rate of change of these critical variables does not exceed a state-dependent threshold \( \Lambda_{\text{adapt}} \) at any point during the process:
          \begin{equation}
            \mathcal{C}(\mathcal{I}, [t_0, t_1]) \iff \forall t \in [t_0, t_1] : \left\| \frac{dS_{\text{crit}}}{dt}(t) \right\|_{\mathcal{S}} \le \Lambda_{\text{adapt}}(S_{\text{crit}}(t))
            \label{eq:continuity}
          \end{equation}
          Where:
          \begin{itemize}
            \item \( \mathcal{C}(\mathcal{I}, [t_0, t_1]) \) represents the proposition that Processual Continuity holds for the intervention \( \mathcal{I} \) over the interval \( [t_0, t_1] \).
            \item \( S_{\text{crit}}(t) \in \mathcal{S} \) is the state vector in a high-dimensional state space \( \mathcal{S} \), capturing the critical physical and dynamical variables (e.g., neural firing rates, synaptic weights, metabolic concentrations) necessary for consciousness at time \( t \). Identifying the components of \( S_{\text{crit}} \) and the appropriate state space \( \mathcal{S} \) is a major ongoing challenge for neuroscience and Synconetics.
            \item \( \frac{dS_{\text{crit}}}{dt}(t) \) is the velocity vector in the state space \( \mathcal{S} \) at time \( t \), representing the instantaneous rate of change of the critical state variables. This formulation assumes sufficient smoothness for differentiability, which may be an idealisation of complex biological dynamics.
            \item \( \| \cdot \|_{\mathcal{S}} \) denotes a suitable norm defined on the tangent space of \( \mathcal{S} \), quantifying the magnitude (speed) of the state change. The choice of the norm (e.g., Euclidean \( L^2 \), maximum \( L^\infty \)) depends on which aspects of the state change are most critical to functional integrity and are subject to empirical investigation.
            \item \( \Lambda_{\text{adapt}}: \mathcal{S} \to \mathbb{R}^+ \) is a positive scalar function representing the state-dependent adaptive capacity threshold. It quantifies the maximum speed of change the system can tolerate at state \( S_{\text{crit}}(t) \) by invoking adaptive mechanisms (e.g., neural plasticity, homeostasis) without losing functional integrity or phenomenological continuity. Characterising \( \Lambda_{\text{adapt}} \) empirically for different states and interventions is a key experimental target for SCM protocols.
          \end{itemize}

          This inequality (\autoref{eq:continuity}) thus formalises the core engineering constraint of Synconetics: the rate at which the system is forced to change by an intervention \( \mathcal{I} \) must remain below the rate at which the system can successfully adapt to that change.
  \end{itemize}

  \section{Methods and Approaches}
  \label{sec:methods}

  The principles and definitions outlined earlier require translation into practical engineering applications. This section illustrates how synconetics informs concrete research and development efforts, moving beyond theoretical frameworks. It details two distinct approaches currently being pursued by the co-authors in academia and industry. These approaches serve as initial examples of SCM in action, showcasing verifiable strategies aimed at addressing the significant challenge of preserving the 4D process-world-line.

  \subsection{Ectopic Cognitive Preservation}
  \label{sec:daniel-approach}

  The approach termed ‘Ectopic Cognitive Preservation’ (ECP), under development by Eightsix Science Ltd (a venture co-founded by co-author Daniel Burger), exemplifies a synconetics methodology. As a practical application of synthetic consciousness mechanics, it focuses on ensuring the physical continuity of the biological substrate through gradual, technologically mediated replacement, aiming to create a resilient synthetic consciousness substrate eventually. Its core technical proposal involves the progressive, piecemeal substituting existing biological brain tissue with biohybrid neural grafts. These grafts are intended as constructs of living neural tissue, potentially derived from the patient’s own induced pluripotent stem cells (autologous iPSCs) differentiated into appropriate neural lineages to circumvent immune rejection and integrated during advanced bioprinting with micro- or nano-scale electronic components. These integrated elements serve various functions, such as sensing local activity, providing targeted stimulation, offering structural support, or facilitating metabolic exchange during cell maturation. Achieving and rigorously verifying true functional equivalence between the original tissue and the graft—encompassing not merely basic neuronal firing but complex network dynamics, synaptic plasticity profiles, and the preservation of identity-critical information patterns necessary for maintaining the unique 4D process-world-line—represents a monumental, yet central, challenge for this approach.

  \begin{figure}[ht]
    \centering
    \includegraphics[width=\textwidth]{figures/ecp-brain-replacement.png}
    \label{fig:ecp-replacement}
    \caption{\% TODO: Visualisation of biohybrid brain replacement.}
  \end{figure}

  ECP’s commitment to continuity hinges critically on the principle of gradualism, designed to leverage the brain’s inherent plasticity (\textit{cf.} Principle 4, Physical Realisability via plasticity) and capacity for functional reorganisation, analogous to adaptations observed in response to slow-growing lesions like benign low-grade gliomas (LGGs). The core hypothesis is that by carefully managing the rate of replacement—ensuring the condition for Processual Continuity (\autoref{eq:continuity}) is met by keeping changes within the adaptive threshold \( \Lambda_{\text{adapt}} \). Neural information processing and functional roles can migrate or be re-encoded within the new substrate without disrupting the overall continuity of cognitive processes and, crucially, conscious experience. This reliance on plasticity, while biologically plausible, carries inherent risks regarding the fidelity of information preservation; ensuring that specific memories, learned skills, and personality nuances constituting the individual’s identity are faithfully maintained during such transitions rather than merely enabling functional adaptation remains a key area requiring deep theoretical understanding and empirical validation via first-person reports alongside physical monitoring. Methodologies for precisely controlling gradual silencing and for real-time monitoring of graft integration and functional takeover are, therefore, critical research components.

  \begin{figure}[ht]
    \centering
    \includegraphics[width=\textwidth]{figures/ecp-vr-integration.png}
    \label{fig:ecp-vr-integration}
    \caption{\% TODO: Visualisation of VR integration.}
  \end{figure}

  The initial outcome targeted by ECP is a rejuvenated, potentially enhanced biological or biohybrid brain residing within the original cranium. Composed progressively of the new graft material integrated with embedded electronics, this enhanced substrate aims primarily to halt or reverse age-related degradation within the brain itself, thereby addressing a primary failure mode contributing to the cessation of the process-world-line. The integrated electronics could also offer inherent capabilities for advanced Synthetic Consciousness Interfacing (SCI), enabling seamless integration with virtual or augmented reality environments without requiring separate invasive procedures later.

  The ultimate, more radical goal of ECP involves the surgical explantation of this fully replaced biohybrid brain. Sustained long-term via an advanced, closed-loop whole-brain perfusion system providing a meticulously controlled physiological environment ex vivo, its function would be embedded within sophisticated virtual environments through high-bandwidth SCI channels derived from the integrated electronics. This step aims to achieve complete decoupling from the vulnerabilities of the original biological body, enhancing resilience. Realising stable, long-term ex vivo maintenance presents immense technical hurdles, demanding perfect replication of complex physiological conditions. Furthermore, the profound ethical and psychological implications of explantation and existence within a potentially constrained virtual reality necessitate careful consideration beyond mere technical feasibility, touching upon questions of identity, well-being, and the nature of experience itself.

  \begin{figure}[ht]
    \centering
    \includegraphics[width=\textwidth]{figures/ecp-brain-explant.png}
    \label{fig:ecp-brain-explant}
    \caption{\% TODO: Visualisation of brain explantation.}
  \end{figure}

  Methodologically, ECP aligns directly with the foundational principles of Synconetics. It confronts biological death as a contingent substrate failure problem, addressing it through a tangible engineering methodology grounded in \textbf{Empirical Primacy} (Principle 4) by integrating bioprinting, grafting, BCI, and perfusion systems based on established science rather than speculation. Its defining characteristic is the explicit commitment to \textbf{Non-Destructive Transition} (Principle 2), prioritising Processual Continuity through gradual replacement, meticulously designed to preserve the 4D process-world-line and avoid the existential risks of destructive copying. This approach relies on the brain’s known plasticity to enable the gradual integration required by Principle 2, consistent with the focus on necessary physical dynamics outlined in \textbf{Substrate Agnosticism with Physical Grounding} (Principle 3). Furthermore, ECP adheres to a core Synconetic tenet by focusing squarely on survival and resilience—halting degradation and enabling repair—subordinating potential enhancements to the non-negotiable goal of continuity, even if this specific constraint isn’t separately enumerated in the table. While the projected timelines and the ultimate certainty of guaranteeing continuity face valid scrutiny and require significant empirical validation, ECP serves as a concrete example of the Synconetics paradigm: pursuing ambitious engineering goals while respecting \textbf{Process-World-Line Fidelity} (Principle 1). Its roadmap inherently generates intermediate technologies—advanced neural simulation, high-fidelity graft production, progressive replacement techniques—with significant near-term therapeutic and research value, offering a pragmatic pathway for development and funding aligned with the principle of \textbf{Near-Term Engineering Urgency} (Principle 5).

  \subsection{Continuity via Hemispheric Integration}
  \label{sec:masataka-approach}

  A different methodology within the synconetics framework, termed Continuity via Hemispheric Integration (CHI) and conceptually developed by co-author Professor Masataka Watanabe at the University of Tokyo, leverages insights from split-brain research to achieve non-destructive substrate transition. This approach posits that consciousness, while unified, relies on the coordinated activity of both cerebral hemispheres. Clinical evidence demonstrates that separating the hemispheres can result in two distinct streams of consciousness, which can subsequently re-fuse if reconnected. CHI proposes to exploit this phenomenon by using an advanced Brain-Machine Interface (BMI) to functionally separate the biological hemispheres while simultaneously linking each to a dedicated, initially non-conscious, Artificial Hemispheric Complement (AHC) – a form of synthetic consciousness substrate. This strategy fundamentally differs from classical mind uploading, aiming to migrate conscious processes gradually through functional integration and redundancy rather than destructive scanning and replication.

  The proposed CHI process commences with the surgical implantation of a high-bandwidth, segmented BMI designed to functionally isolate the native hemispheres (akin to a corpus callosotomy) while enabling bidirectional communication between each biological hemisphere and its paired AHC. This initial separation is expected to induce a temporary state of dual consciousness, similar to that observed in split-brain patients, a significant but theoretically manageable challenge. Subsequently, through processes analogous to neural plasticity and potentially guided by targeted stimulation protocols, functional pathways and informational representations (including memories and learned behaviours) are hypothesised to integrate between each biological hemisphere and its corresponding AHC. This ‘neural routing’ aims to establish the AHCs as functional extensions and eventual successors to their biological counterparts.

  Continuity during this critical integration phase is predicated on redundancy. As function integrates into an AHC, the combined bio-artificial system for that hemisphere maintains processing. Crucially, even if one biological hemisphere were to fail or degrade significantly, the overall conscious process is hypothesised to persist uninterrupted within the remaining biological hemisphere and its fully integrated AHC partner, mediated by the inter-hemispheric capabilities of the BMI itself. Memory transfer is envisaged via mechanisms involving both active recall (mirrored and stored in the AHC) and potentially stimulated recall (akin to Penfield’s findings), with protocols allowing for user oversight regarding sensitive memories.

  The final stage occurs once both AHCs have achieved sufficient functional integration to fully support the dynamics previously handled by their biological partners. As the biological hemispheres naturally cease function or are allowed to terminate, the individual’s conscious process-world-line persists entirely within the paired AHCs. These two AHCs are then functionally merged via the BMI or direct inter-AHC connection, restoring a unified conscious experience within a purely synthetic substrate (SCS). This final substrate, potentially operating within a secure physical location, could interact with external reality via advanced Synthetic Consciousness Interfacing (SCI), such as immersive virtual environments or robotic avatars, thereby decoupling from biological fragility.

  Methodologically, CHI aligns with the principles of synconetics. It addresses death as a substrate failure, circumventing it via engineered transition (\textbf{Substrate Agnosticism with Physical Grounding}, Principle 3, requiring AHCs to sustain necessary dynamics). It employs a tangible \textbf{engineering methodology} centred on advanced BMIs and leveraging neuroplasticity (\textbf{Empirical Primacy}, Principle 4, building on split-brain data and testable integration). Its core design is predicated on \textbf{Non-Destructive Transition} (Principle 2), aiming to preserve the 4D process-world-line through redundancy and gradual integration, explicitly avoiding destructive methods. While the transition involves profound alteration, the goal is strict adherence to \textbf{Process-World-Line Fidelity} (Principle 1) by ensuring no complete cessation of conscious processing occurs. The approach leverages foreseeable advancements in BMI technology and neuroscience (\textbf{Near-Term Engineering Urgency}, Principle 5). However, CHI faces immense technical challenges, including the development of BMIs with unprecedented fidelity and bidirectional integration capabilities, the design of AHCs capable of replicating essential neural dynamics, the verification of successful integration and memory transfer, and the profound ethical and phenomenological questions surrounding the induced split-consciousness phase. Despite these hurdles, CHI represents a compelling, continuity-focused alternative within the synconetics paradigm.

  \section{Feasibility and Opportunities}
  \label{sec:feasibility}

  The methodologies outlined in this essay—Ectopic Cognitive Preservation (ECP) and Continuity via Hemispheric Integration (CHI)—represent complementary pathways toward solving death. ECP adopts a biological-first strategy, incrementally transitioning to biohybrid substrates while leveraging existing neurosurgical and regenerative techniques. CHI prioritises synthetic substrates from inception, using split-brain dynamics as a gateway to engineer synthetic consciousness systems. These approaches are not mutually exclusive; instead, they form a strategic duality: ECP’s gradual replacement mitigates near-term risks by preserving biological continuity, while CHI’s synthetic focus probes the boundaries of substrate independence. Critically, advances in one methodology directly inform the other. For instance, CHI’s synthetic substrate validation protocols could eventually enable ECP to transition fully to artificial systems, while ECP’s neurointegration techniques refine CHI’s bio-synthetic interfaces.

  Purely synthetic substrates, as explored in CHI, promise inherent advantages over biological systems: immunity to pathogens, tolerance for extreme environments (e.g., radiation, temperature), and engineered fault tolerance through redundancy. However, biology remains the only empirically validated substrate—a reality that ECP’s conservatism respects. Together, these approaches create a fail-safe continuum: synthetic substrate development proceeds without gambling existing consciousness, while biological augmentation extends survival until synthetic options mature.

  \subsection{Engineering Resilience Through Substrate Design}
  \textbf{Decoupling Agency from Embodiment:} Both methodologies enable the conscious substrate (SCS) to persist independently of its original biological body. Whether stored in a shielded facility (ECP’s ex vivo biohybrid brain) or distributed across synthetic nodes (CHI’s Artificial Hemispheric Complements), the SCS becomes decoupled from localised physical threats. Death of the biological body ceases to equate to termination of consciousness—akin to cloud computing surviving individual server failures.

  \textbf{Co-Location and Redundancy:} Synthetic or biohybrid substrates permit distributed architectures impossible in biology. Neural networks could span geographically isolated secure sites, with real-time synchronisation ensuring continuity. Partial destruction (e.g., 2% loss from a localised attack) would not terminate the 4D process-world-line, as critical dynamics persist across redundant nodes—mirroring RAID arrays in data storage.

  \textbf{Redundancy Through Resolution:} Modern computing achieves fault tolerance through atomic-scale repair (e.g., replacing individual transistors), yet biological brains lack analogous precision. High-bandwidth neural interfaces resolve this asymmetry, enabling synthetic repairs at resolutions matching the brain’s microcircuitry. This transforms the brain from a “black box” into an engineerable substrate, allowing repair and augmentation—enhancing memory density or computational speed while preserving continuity.

  \subsection{Sensory Abstraction as an Engineering Opportunity}
  \textbf{Scale Separation of Sensory Input:} Biological consciousness already operates on abstracted sensory inputs—photons reduced to retinal signals, air vibrations to cochlear frequencies. Synthetic Consciousness Interfacing (SCI) does not replicate raw physics; it only needs sufficient bandwidth to match the brain’s native compression. Virtual environments demonstrate this: triangular meshes evoke visceral fear of heights, and 44.1kHz audio fools humans into perceiving “live” music. By exploiting this abstraction, SCI systems can:
  \begin{itemize}
    \item Reduce engineering complexity (no need to simulate quantum fields),
    \item Enhance safety (virtual avatars avoid physical harm),
    \item Expand experiential range (perceiving infrared or ultrasonic ranges via synthetic transduction).
  \end{itemize}

  Existing sensory prosthetics (cochlear implants restoring hearing, retinal arrays granting vision, etc.) validate this approach. Their success hinges not on replicating biology exactly but on providing functionally equivalent inputs. Synconetics extends this principle, prioritising agency and coherence over perfect fidelity.

  \subsection{The Pragmatic Path Forward}
  History shows that transformative engineering often precedes complete theoretical understanding: steam engines predated thermodynamics, and winged flight emerged before aerodynamic models. Synconetics adopts this pragmatic tradition, focusing on manipulating the known physical substrate (the brain) rather than awaiting a unified theory of consciousness. This strategy leverages rapid advancements across synergistic fields:
  \begin{itemize}
    \item \textbf{Neuroscience:} Mapping neural correlates and plasticity mechanisms.
    \item \textbf{Neuroengineering:} Developing high-resolution BCIs and stimulation protocols.
    \item \textbf{Materials Science:} Creating biocompatible interfaces and synthetic substrates.
  \end{itemize}

  The brain’s plasticity is both a tool and a safety net. Gradual interventions (e.g., ECP’s grafts and CHI’s hemispheric integration) leverage plasticity to maintain continuity during substrate transitions. This mirrors how stroke survivors rewire neural pathways, proving that dynamic systems can adapt to structural change without identity loss.

  \subsection{Beyond Biology: The Jet Engine of Consciousness}
  Nature’s evolutionary constraints yielded biological substrates optimised for survival, not resilience. Just as aeronautics transcended avian flight—first mimicking feathers, then inventing turbines—synconetics aims to surpass biology’s limitations. Current efforts (ECP’s grafts, CHI’s AHCs) represent the “propeller phase” of this trajectory: incremental improvements on natural designs. The end goal, however, is the “jet engine” of consciousness substrates: systems leveraging synthetic materials, fault-tolerant architectures, and physics beyond biology’s thermodynamic niche.

  In this paradigm, death becomes both solvable and physically archaic—a contingency as avoidable as wooden biplanes in modern aviation. Preserving the 4D process-world-line transitions from crisis management to routine maintenance, with synthetic substrates offering near-invulnerability to traditional failure modes (trauma, ageing, disease). The challenge shifts from preventing collapse to optimising persistence across cosmological timescales—a feat requiring biological repair and engineered transcendence.

  \subsection{Near-Term Impact and Iterative Progress}
  Synconetics’ feasibility is amplified by its alignment with near-term technological trajectories. Advances in neural prosthetics, organoid development, and biocompatible materials—already funded for medical applications—directly serve substrate stabilisation and repair. Early milestones (e.g., restoring motor function via BCIs, reversing age-related neural decline) offer tangible societal benefits, ensuring continued investment even before full continuity is achieved.

  The engineering challenge, while immense, decomposes into tractable sub-problems:
  \begin{itemize}
    \item Validating synthetic substrates (CHI’s focus),
    \item Ensuring continuity during the transition (ECP’s gradualism),
    \item Optimising sensory interfaces (shared SCI development).
  \end{itemize}

  This modularity allows parallel progress, mirroring the Apollo program’s phased approach to spaceflight. By prioritising incremental, testable advancements over speculative leaps, synconetics transforms immortality from a philosophical ideal into an engineering roadmap.

  \section{DRAFT, DO NOT READ: Roadmap and Implications}
  \label{sec:roadmap}

  % TODO Merge this chapter together with the next one

  The establishment of Synconetics as a viable discipline demands more than foundational principles and precise nomenclature; it requires a pragmatic research and development roadmap. Such a roadmap must candidly acknowledge the profound technical and conceptual challenges inherent in engineering conscious continuity, whilst simultaneously identifying tractable starting points and strategic pathways forward. A central tenet of this pragmatism is the parallel pursuit of complementary methodologies. These distinct approaches address different facets of the core problem, mitigate different categories of risk, and strategically leverage both current and foreseeable technological capabilities. This section outlines such a strategic approach, illustrating the engineering feasibility central to Synconetics by focusing on the interplay between biologically grounded interventions and the exploration of synthetic substrates.

  Given the persistent uncertainties surrounding the precise physical prerequisites for consciousness and the optimal characteristics of a long-term, resilient substrate, a prudent strategy necessitates pursuing distinct, yet potentially synergistic, research programmes concurrently. We advocate for a dual-pronged approach. The first prong encompasses methodologies focused on preserving continuity by directly augmenting, repairing, or gradually replacing the existing biological substrate. This biologically grounded path, exemplified by the Ectopic Cognitive Preservation (ECP) strategy detailed in Section \ref{sec:daniel-approach}, prioritises working with the known substrate, leveraging established biological mechanisms such as neural plasticity alongside advancements in tissue engineering, biohybrid integration, and neurosurgery. Its initial focus lies squarely on mitigating intrinsic biological failure modes, primarily age-related degeneration, and enhancing the resilience of the existing system. While potentially offering a nearer-term route by sidestepping the challenge of creating consciousness ex nihilo in an entirely novel medium, this approach ultimately retains a substrate with inherent biological vulnerabilities. Even an advanced biohybrid brain, particularly if maintained ex cranio, remains susceptible to physical destruction, lacks intrinsic fault tolerance compared to potentially achievable engineered systems, and may face fundamental biological limitations that constrain indefinite persistence.

  The second prong of our strategy directly confronts the challenge of engineering non-biological or radically different physical systems capable of supporting conscious processes, coupled with developing rigorous methods to verify their functional status and, crucially, their capacity for subjective experience. This synthetic substrate path often relies heavily on the development and application of ultra-high-bandwidth, bidirectional Brain-Machine Interfaces (BMIs). Such interfaces serve not merely as input/output channels but as critical tools for gradual integration, functional mapping, and potentially validation – drawing inspiration from proposals for testable machine consciousness, such as those exploring inter-system integration paradigms (as will be further elaborated in Section \ref{sec:masataka-approach}). This route holds the potential for creating substrates with fundamentally greater robustness, engineered fault tolerance, enhanced resilience against environmental hazards, and perhaps even capabilities beyond biological limits. However, this path faces the ‘hard problem’ of consciousness more directly, depending critically on identifying and successfully implementing the correct physical principles or dynamic properties sufficient for instantiating consciousness in a synthetic medium. Success hinges on significant breakthroughs in substrate engineering, ultra-high-fidelity BMI technology capable of seamless, non-disruptive integration, and the development of reliable methods for verifying conscious presence beyond mere functional mimicry. Furthermore, ensuring the continuity of personal identity during any transition or integration process involving a fundamentally different substrate presents unique and formidable theoretical and technical hurdles.

  Pursuing both paths simultaneously provides crucial strategic hedging and risk mitigation. Should the engineering of verifiable consciousness in synthetic substrates prove unexpectedly intractable, or if current assumptions about the sufficiency of certain physical dynamics (e.g., specific computational architectures) turn out to be incorrect, the biologically grounded path offers an alternative route towards significantly extended persistence and resilience. Conversely, if the inherent limitations or vulnerabilities of biological or biohybrid systems ultimately prove insurmountable for achieving indefinite continuity or sufficient resilience against catastrophic failure, advancements along the synthetic substrate path offer a potential long-term solution. This duality aligns directly with the core Synconetics principle of seeking robust, engineered solutions while honestly acknowledging current scientific unknowns and technological limitations.

  Crucially, these two paths are not entirely independent; significant synergies exist, and they may eventually converge. Advancements in the sophisticated BCIs required for the later stages of ECP (such as embedding within rich virtual environments or enabling enhanced cognitive control) are direct precursors to the ultra-high-fidelity interfaces essential for the synthetic substrate path. Conversely, insights gained from attempting to engineer Synthetic Consciousness Substrates (SCS)—particularly regarding the minimal dynamic complexity or specific organisational principles required—can directly inform the design criteria and functional targets for the biohybrid grafts used in ECP. It is conceivable that biologically grounded approaches like ECP could serve as a vital transitional phase, creating a stabilised, enhanced biological or biohybrid platform from which safer, more gradual, and verifiable integration with future synthetic systems might be achieved.

  The deliberate engineering focus of Synconetics enhances the practical feasibility of this roadmap. The ECP path, with its clearly defined intermediate goals in advanced tissue engineering, regenerative medicine for neurological conditions, and improved BCIs, offers tangible near-term therapeutic and potentially commercial value. This creates opportunities for phased, sustainable funding streams, aligning research with demonstrable benefits. The synthetic substrate path, while perhaps representing a longer-term endeavour, involves fundamental research in neuroscience, materials science, physics, and BMI technology that is attractive to governmental and foundational research funding agencies. Its emphasis on developing testable hypotheses and verifiable outcomes, even if focused initially on intermediate measures of complex dynamics or information integration rather than subjective report, makes it more tractable than purely speculative or philosophical approaches to artificial consciousness. Both paths strategically avoid reliance on unproven fundamental physics or distant science-fiction concepts like atomically precise nanotechnology, focusing instead on integrating and aggressively advancing existing technological frontiers in bioengineering, neurotechnology, and complex systems engineering. Nonetheless, the timelines for achieving the ultimate goals of either path remain highly ambitious, and securing consistent, long-term funding—particularly for the more fundamental aspects of the synthetic substrate research—will undoubtedly be challenging and requires demonstrating consistent, verifiable progress against defined milestones.

  % \section{Socio-Economic and Ethical Implications}
  % \label{sec:economics}

  The potential success of Synconetics methodologies, even within the challenging timeframes we acknowledge, necessitates a departure from purely technical discourse or distant philosophical speculation. If conscious continuity can be reliably engineered, enabling individuals to persist beyond the limitations of their original biological substrate, it precipitates profound socio-economic, political, and ethical questions demanding pragmatic analysis today. The assertion that Synconetics offers a potentially near-term engineering pathway, distinct from indefinite postponement pending future scientific revolutions, compels us to confront these implications not as hypothetical scenarios, but as foreseeable consequences requiring immediate, serious consideration alongside technical research and development.

  The emergence of individuals whose consciousness persists via engineered substrates—whether advanced biohybrids or entirely synthetic systems—fundamentally challenges existing legal and political frameworks, which are entirely unprepared for non-biological personhood. How is legal identity defined for an entity potentially lacking a conventional biological body? Questions of citizenship, property ownership, voting rights, and the very basis of legal standing become acutely problematic. Establishing internationally recognised standards for the personhood, rights (such as substrate autonomy, freedom from non-consensual modification, access to environments) and responsibilities (taxation, legal liability) of Synconetic entities represents a monumental political and philosophical undertaking, fraught with potential for inequality and novel forms of exploitation if not proactively addressed.

  Economically, the advent of potentially vastly long-lived or effectively immortal conscious entities promises radical disruption. Can such entities participate meaningfully in labour markets, particularly alongside accelerating AI automation? Assessing their potential for cognitive, creative, or virtual value creation is complex; their existence may necessitate fundamental shifts in economic models, potentially reinforcing arguments for systems like Universal Basic Income if traditional biological labour diminishes further. Furthermore, the significant, ongoing resource demands—energy, computation, physical security, specialised maintenance—for sustaining consciousness-supporting substrates raise critical questions of allocation. What economic models (e.g., subscription, public utility, private ownership) govern access and upkeep, and how can unprecedented societal stratification between those who can afford continuity and those who cannot be avoided? The potential for cost to exacerbate existing inequalities demands careful forethought.

  The infrastructural and logistical realities of supporting a population of Synconetic entities are equally daunting, involving engineering challenges often vastly underestimated. Robust, secure physical and digital infrastructure is paramount. Where are consciousness-supporting substrates housed? What levels of physical security, redundancy against technical failure or environmental catastrophe, and resilience against malicious attack are achievable and sustainable? Centralised hosting creates single points of failure and control, whilst distributed models present immense logistical hurdles. Provider viability is another critical concern: what happens if a commercial or state entity responsible for hosting becomes insolvent, politically unstable, or technologically obsolete? Without clear standards and protocols guaranteeing substrate or data portability—enabling transfer between providers or substrate types without violating continuity—individuals face extreme vulnerability and vendor lock-in. The sheer energy and computational load, especially if entities interact within rich virtual environments, also poses significant questions about long-term global sustainability.

  Perhaps most profoundly, the successful realisation of Synconetics challenges fundamental societal notions of life, death, identity, and community. How will society perceive these entities—as ‘alive’, ‘post-biological’, or something entirely new? How do existing relationships, inheritance laws, legacy considerations, and social security systems adapt? The psychological well-being of individuals undergoing transition and potentially existing indefinitely, perhaps within environments vastly different from baseline biological reality, presents significant risks and necessitates novel forms of support. Maintaining existential meaning under such conditions is a critical, open question. Ensuring equitable access and mitigating the potential for coercion (e.g., societal pressure to transition) are paramount ethical considerations. Finally, defining ‘death’ for a Synconetic entity and establishing ethical end-of-life protocols—managing substrate failure, irreversible cognitive decline, or respecting an individual’s voluntary wish to cease existence—represents entirely uncharted territory demanding sensitive, cross-disciplinary deliberation.

  The potential near-term feasibility advocated by Synconetics thus transforms these issues from speculative fiction into urgent matters for contemporary policy, ethics, and engineering. The stark contrast between this potential and the current lack of serious planning underscores the imperative for proactive engagement. Addressing the legal, economic, infrastructural, and ethical dimensions cannot be postponed; it must occur in parallel with technical research and development. This proactive, transdisciplinary effort is essential to mitigate the risks of societal disruption, inequality, and catastrophic failure, ensuring that the pursuit of engineered continuity aligns with broadly shared human values. It is a core tenet of the Synconetics approach that responsible engineering necessitates foresight into its societal consequences.

  % ! MARK: Conclusion and Call to Action
  \section{DRAFT, DO NOT READ: Conclusion}
  \label{sec:conclusion}
  This essay has introduced Synconetics, a scientific and engineering discipline founded upon the conviction that biological death, understood fundamentally as a failure of the consciousness-supporting substrate, represents a tractable engineering challenge. We have argued that by rigorously prioritising the uninterrupted physical continuity of the processes underpinning individual consciousness—the preservation of the unique 4D process-world-line—and concentrating on tangible, buildable systems grounded in established science, Synconetics charts a more robust, ethically defensible, and ultimately achievable course than paradigms reliant on destructive replication or unverified philosophical assumptions, such as strong computationalism. Its framework offers a pragmatic pathway towards ensuring the persistence of conscious existence.

  The methodologies currently being developed within the Synconetics framework, exemplified by the Ectopic Cognitive Preservation strategy (Section \ref{sec:daniel-approach}) focused on gradual biohybrid replacement, and complemented by approaches centred on advanced Brain-Machine Interface integration for probing and potentially validating synthetic substrates (as anticipated in Section \ref{sec:masataka-approach}), serve as initial, concrete demonstrations of this potential. They affirm that research and development aligned with Synconetics principles can commence immediately, strategically leveraging existing and foreseeable advancements across neuroscience, bioengineering, materials science, and related fields. This potential feasibility, suggesting meaningful progress within decades rather than indefinite centuries, transforms the profound socio-economic, legal, and ethical questions accompanying engineered conscious continuity from distant speculations into urgent matters demanding immediate, serious consideration. Proactive, transdisciplinary planning and societal dialogue are not merely advisable; they are imperative to navigate the immense societal shifts this technology could precipitate.

  We contend, therefore, that a significant redirection of focus and resources is necessary within the broader constellation of research aiming to overcome biological limitations. A paradigm shift is required, moving decisively towards the direct engineering of continuity and substrate resilience. This involves embracing the complexities of physical instantiation and continuous process dynamics, rather than pursuing potentially flawed or existentially risky shortcuts predicated on abstract information patterns or destructive scanning alone.

  The advancement of Synconetics, however, cannot be the work of isolated groups; it demands a concerted, collaborative, and deeply transdisciplinary effort. We issue a call to action to researchers, engineers, clinicians, ethicists, policymakers, entrepreneurs, and funders worldwide to engage actively with this nascent field. We invite scientists and engineers to \textit{advance the research frontier} by pursuing fundamental research and targeted engineering development aligned with Synconetics principles; opportunities exist for postgraduate research exploring consciousness mechanisms and BMI integration (e.g., with Prof. Watanabe’s group at the University of Tokyo) and for applied RnD within dedicated ventures. We encourage engagement with, and support for, organisations \textit{translating Synconetics principles into practice}, such as Eightsix Science (currently seeking technical collaborators, funding, and grant support for its ECP approach). We urge innovators to \textit{foster diversity and progress} by launching new research projects or companies exploring alternative continuity-preserving strategies; a healthy ecosystem of complementary approaches will strengthen the entire field.

  Furthermore, we call upon the community to \textit{engage in critical dialogue}: connect with the authors and other researchers to discuss these concepts, rigorously challenge assumptions, and collaboratively refine the Synconetics framework. Join the nascent community discussions (e.g., via the established Discord server) to share insights and foster collaboration. Help \textit{disseminate and develop knowledge} by sharing this whitepaper and engaging peers in substantive discussion. Contribute to future knowledge-building efforts, such as the planned comprehensive book, \textit{Synthetic Consciousness}; we actively seek co-authors from diverse disciplines, particularly medicine, law, economics, and ethics, to ensure a truly comprehensive and transdisciplinary perspective. Finally, support or participate in initiatives designed to \textit{convene the community}, potentially including a dedicated Synthetic Consciousness Conference, to consolidate research findings and catalyse interdisciplinary exchange.

  Synconetics represents more than a theoretical exercise or a distant dream; it is a call to apply the full power of rigorous engineering principles, tempered by ethical foresight, to one of humanity’s oldest and most profound challenges. By maintaining an unwavering focus on verifiable physical continuity and the development of buildable, reliable systems, we can begin to move beyond speculative fiction towards tangible progress in ensuring the persistence of human consciousness. We welcome all who share this vision and commitment to join us in building this critical field.

  % TODO: Add call to action appendix chapter

  % izumi’s notes/version:
  % This essay has introduced Synconetics, a scientific and engineering discipline founded upon the conviction that biological death, understood fundamentally as a failure of the consciousness-supporting substrate, represents a tractable engineering challenge.
  % The methodologies currently being developed within the Synconetics framework, exemplified by the Ectopic Cognitive Preservation strategy (Section 4.1) or “Masa’s approach name” serve as initial, concrete demonstrations of this potential. They affirm that research and development aligned with Synconetics principles can commence immediately, strategically leveraging existing and foreseeable advancements across neuroscience, bioengineering, materials science, and related fields. The emerging feasibility of this challenge —suggesting tangible progress not in some indefinite future, but within the coming decades—elevates the associated socio-economic, legal, and ethical considerations from distant speculation to urgent, real-world challenges requiring immediate attention (Section 6).
  % We invite scientists and engineers to advance the research frontier through rigorous inquiry into consciousness mechanisms, substrate stability, and information-preserving transformation processes, and to translate such insights into functional architectures—both computational and biological. Opportunities exist for postgraduate research (e.g., with Prof. Watanabe’s group at the University of Tokyo), for applied R/D within ventures developing real-world platforms grounded in Synconetics principles, and for direct collaboration with organisations actively pursuing Synconetics-aligned implementation—such as Eightsix Science, which is currently seeking technical collaborators, funding, and grant support for its Ectopic Cognitive Preservation (ECP) strategy.
  % More broadly, as Synconetics is entering a stage where early implementation is becoming viable— it can no longer and shall no longer remain only in the exclusive and isolated domain of neuroscientists, imperative to involve a diverse constellation of expertise. It demands a concerted, collaborative, and deeply transdisciplinary effort, bringing together foundational science, applied engineering, cognitive interface design, and sustainable systems thinking. We issue a call to action to clinicians, ethicists, entrepreneurs, policymakers, and funders worldwide to engage actively with this nascent and upcoming field. To that end, we call upon:
  % Ethicists and legal scholars
  % – to engage deeply with the evolving boundaries of identity, personhood, and agency—particularly in contexts involving biohybrid substrates, embedded memory loops, or longitudinal re-instantiations of self—that may challenge existing moral and legal frameworks.
  % Policymakers and institutional stakeholders
  % – to consider the societal, infrastructural, and temporal implications of Synconetics-based interventions, especially within domains such as end-of-life care, long-term cognitive augmentation, and anticipatory biosecurity, and to help shape responsive, future-oriented governance structures.
  % Product strategists and venture builders
  % – to articulate viable use cases and service models—particularly in healthcare, education, age-related cognitive support, and climate-responsive human systems—anchored in real human need and technical feasibility, and to guide the emergence of Synconetics-aligned products into coherent go-to-market trajectories.


  % UX researchers and interaction designers
  % – to address not only interface-level usability but also deeper questions of trust, comprehension, affective alignment, and cognitive asymmetry between human users and systems operating on non-standard substrates or asynchronous timelines, thereby shaping how such systems are understood, accessed, and integrated into lived experience.
  % Entrepreneurs and funders
  % – not only to accelerate experimentation and translational pathways, but also to envision and establish the long-lived institutional and business architectures necessary to sustain cognitive systems over timeframes that may exceed current market expectations or assumptions.


  % Editors, publishers, and intellectual curators
  % – to help structure, translate, and amplify emerging discourse in this space—through books, essays, and public-facing media—and to convene collaborations between technical and humanistic voices capable of holding the full conceptual and ethical complexity Synconetics entails.

  % Furthermore, we call upon the community to engage in critical dialogue: connect with the authors and other researchers to discuss these concepts, rigorously challenge assumptions, and collaboratively refine the Synconetics framework. Join the nascent community discussions (e.g., via the established Discord server) to share insights and foster collaboration. Help disseminate and develop knowledge by sharing this whitepaper and engaging peers in substantive discussion. Contribute to future knowledge-building efforts, such as the planned comprehensive book, Synthetic Consciousness; we actively seek co-authors from diverse disciplines, particularly medicine, law, economics, and ethics, to ensure a truly comprehensive and transdisciplinary perspective. Finally, support or participate in initiatives designed to convene the community, potentially including a dedicated Synthetic Consciousness Conference, to consolidate research findings and catalyse interdisciplinary exchange.
  % Synconetics represents more than a theoretical exercise or a distant dream; it is a call to apply the full power of rigorous engineering principles, tempered by ethical foresight, to one of humanity’s oldest and most profound challenges. By maintaining an unwavering focus on verifiable physical continuity and the development of buildable, reliable systems, we can begin to move beyond speculative fiction towards tangible progress in ensuring the persistence of human consciousness. We welcome all who share this vision and commitment to join us in building this critical field.

  % ! ========================
  % ! # MARK: References, etc.
  % ! ========================

  \pagebreak
  \bibliographystyle{../../templates/custom-apa}
  \bibliography{references/bibliography}
  \nocite{*}

\end{sloppypar}
\end{document}